%!TEX TS-program = xelatex
%!TEX encoding = UTF-8 Unicode

\documentclass[12pt,a4paper]{extarticle}
% extarticle is like article but can handle 8pt, 9pt, 10pt, 11pt, 12pt, 14pt, 17pt, and 20pt text

\def \ititle {Joint Action Nijmegen Talk:}
\def \isubtitle { Shared Intentions and Collective Goals}
\def \iauthor {Stephen A. Butterfill}
\def \iemail{s.butterfill@warwick.ac.uk}
%\date{}

\input{$HOME/Documents/submissions/preamble_steve_paper}

\begin{document}

\setlength\footnotesep{1em}

\bibliographystyle{newapa} %apalike

\maketitle
%\tableofcontents

\begin{abstract}
A joint action is a goal-directed action, or something resembling one, comprising two or more agents' activities.  What is the relation between a joint action and the goal (or goals) to which it is directed?  Characterising this relation is necessary for saying what joint actions are and for understanding in what senses, if any, they are actions.  Nearly all existing accounts of the relation between joint actions and their goals hinge on a notion of shared intention.  Such accounts are at best incomplete if, as argued in this talk, there are significant cases of joint action without shared intention.  The talk will briefly provide background on joint action and what shared intention is, followed by an argument that not all joint actions involve shared intention and, in the concluding part, an attempt to explain the relation between joint actions and their goals without invoking shared intentions.  
*Consequenceg
\end{abstract}


\section{Nijmegen questions and comments}

SEE PRESENTATION SLIDES FOR UPDATES TO FINAL SECTION OF TALK ON COLLECTIVE GOALS.

Ulf 1: instead of thinking of this as joint action, why not think of it as individual action with some of the features of joint action?  [*Didn't properly get this question; it has something to do with people having said that if you can possibly think of a case as involving individual action then you should.]

Marc Slors: Emergent coordination. Your first example (Ayesha/Beatrice) is one where coordination emerges with no central coordination.  Might compare Andy Clark's example of the cockroach legs.


Guenther: there are cases of joint action without shared intention which are not successful where I can nevertheless identify the goal of an action.  For example, this would seem to be possible in the first, Ayesha \& Beatrice case.  The account you have given only applies where the coordination causes the outcome and so there can only be collective goals for successful actions.  It also seems odd to leave open whether the action has a collective goal until it has happened.

Reply: Yes, I need to modify the third clause (see slides).  It isn't that the coordination has to cause or partly constitute the outcome; it is that that it should normally do so.  (Might also consider the possibility that, as the activities progress, the coordination must be bringing the outcome close; but I'm not sure that sort of line would work for every case of joint action.)


Unknown2: you need to be more specific about coordination

Reply: Agree.  Consider the Ayesha and Beatrice example.  There are three kinds of coordination.  There is coordination of the time they both start pulling; there is coordination by means of the rope which ensures that the force they exert brings the block upwards; and there is the ongoing need to balance the force you are exerting against the force that the other exerts.  There's a link to research on coordination through haptic channels here.


Ulf 2: What are the difference consequences of shared intentions and collective goals?  For example, Michael Bratman stipulates that helpfulness is a necessary condition for shared cooperative activity.  This is clearly not a necessary condition of there being a collective goal.  In fact, Bratman's mafia-New-York case would count as joint action on your view.

Reply: Reply to the question of detail first.  Yes, the Mafia-new-york example counts as joint action because there is a distributive goal (each agent intends to get to New York), there is coordination (the fact that it is a consequence of coercion doesn't disqualify it) and the coordination facilitates the outcome.  So I agree.  But I don't think this is a problem for the account.  You can coerce someone to perform an individual action, e.g. to take the rubbish out, without interfering with the relation between their action and its goal.  Similarly, I don't think a history of coercion is relevant to assessing whether something is joint action.  Of course there might be good reasons for special interest in cases with more advanced features.  But it doesn't follow that these features are necessary for joint action generally.

Idea: start by stressing that the aims of Bratman's account are unclear.  In one place he talks about `a notion we care about [check]', in another place (MSDI) it's a conceptual framework to support theorising.


Unknown: I think that cases of common effect without coordination might still be cases of goal-directed joint action.  

Reply: E.g. consider the hoodlum-bus case.


Marc Slors 3: connect this to theory of mind!  

Reply: Introduce shared goals; say Joint Action and Knowing Others' Minds.

Marc Slors Reply: sounds like this has a lot in common with Gallagher's interaction theory.  



\section{Introduction}
A joint action is a goal-directed action, or something resembling one, comprising two or more agents' goal-directed activities. 

Philosophers' paradigm cases of joint action include painting the house together (Michael Bratman), lifting a heavy sofa together (David Velleman), preparing a hollandaise sauce together (John Searle), going to Chicago together (Christopher Kutz), and walking together (Margaret Gilbert).

The question for my talk is, \textbf{What is the relation between a joint action and the goal (or goals) to which it is directed?}  
%(Here and throughout we ignore for simplicity the possibility that joint actions may be directed to multiple goals.  The answers considered are consistent with this possibility.)

To illustrate, suppose Ayesha and Beatrice between them lift a table thereby releasing a cat whose paws were trapped and,  simultaneously, breaking a glass.  On standard theories of events, the event which initiated the cat's release is identical to the event which initiated the glass' destruction \citep{Davidson:1969ie}.  But observing this joint action we might wonder whether its goal was the release of the cat or the destruction of the glass (or both); and if we are in doubt about this then in an important sense we don't yet know which action Ayesha and Beatrice performed.  To identify their action we need not only a concrete event, something with temporal and spatial properties, but also an abstract outcome such as that of the cat's release \citep{Davidson:1971fz}.

Ayesha, asked about the table lifting episode mentioned above, might insist, `The goal of our intervention wasn't to smash the glass but to free the cat.'  (Note that this concerns the goal of the joint action and not---or not directly---the agents' intention.)  In terms of this example, we could put the question like this.  \textbf{What makes it true that the goal of Ayesha and Beatrice's action is that of releasing the cat rather than that of breaking the glass?}

Answering this question is necessary for saying what joint actions are and for understanding in what senses, if any, they are actions.
%*better sentence on significance of question?

The term `goal' has been used both for outcomes (as in `the goal of our struggles') and, perhaps improperly, for psychological states (it is in this sense that agents' goals might cause their actions).  The terms `goal-outcome' and `goal-state' make the distinction explicit.  Our question is about how joint actions are related to their goal-outcomes.

For individual (that is, not joint) action, the counterpart of this question has a standard answer.  In barest outline, an individual action consists in an agent acting on an intention (a goal-state).  The intention has a content which specifies an outcome.
%, namely the outcome whose occurrence would consist in the propositional content's being true.
This is the goal-outcome of the action.
It is the goal-outcome of the action in virtue of being specified by the intention (or goal-state) the agent is acting on. 

In essence, then, on the standard theory, an individual action is related to its goal-outcome in virtue of the agent's acting on one or more intentions or goal-states.


How do things stand in the case of joint action?  Can we extend the standard theory from individual to joint action?  


The usual way of thinking about joint action starts with the premise that all significant cases of joint action involve shared intention (dissenters are mentioned below).  For instance:  
\begin{quote} 
`I take a collective action to involve a collective intention.'  \citep[p.\ 5]{Gilbert:2006wr}.
\end{quote}
\begin{quote}
`the key property of joint action lies in its internal component [...] in the participants’ having a “collective” or “shared” intention.' \citep[pp. 444-5]{alonso_shared_2009}.
\end{quote}


But what is shared intention?

In barest outline, shared intentions are supposed to do for multiple agents some of what ordinary intentions do for individuals.  So, like an ordinary intention, a shared intention coordinates plans and goal-directed activities---with the difference that these are the plans and goal-directed activities of multiple agents performing a joint action \citep{Bratman:1993je}.

The notion of shared intention provides a straightforward way to extend the standard theory about how actions are related to their goal-outcomes to cases of joint action.

In brief, the standard theory is extended by substituting shared for individual intentions and otherwise unchanged.  

I want to argue, not that this extension of the standard theory is incorrect, but just that it is not a fully account of the relation between joint actions and their goals.

For I claim that there are significant cases of joint action without shared intention.

So I think the following propositions can both be true:

\begin{enumerate}

\item The goal of Ayesha and Beatrice's action is to free the cat.  [That is, their joint action has a goal-outcome.]

\item It is not true that Ayesha and Beatrice have a shared intention to free the cat. 

\end{enumerate}

There are more and less radical versions of my thesis.  The less radical thesis says that, although \emph{shared intentions} are not necessary for joint action, it is necessary that the agents have shared goal-states of some kind.  The more radical thesis, the one that I will defend, says that in some cases the link between a joint action and its goal-outcome is not provided by a shared goal-state of any kind.

So I think the first proposition above is also consistent with this one:

\begin{enumerate}[resume]

\item It is not true that Ayesha and Beatrice have any kind of  shared goal-state.  

\end{enumerate}

\marginpar{ \tiny \flushleft
[*slide: change shared intention to any kind of goal-state]
}

It seems to me that the consistency of these propositions shouldn't be very controversial.  Compare the case of individual action---it is quite widely accept that individual actions can be goal-directed without involving an intention or any other goal-states.  
And there are accounts of the relation between actions and their goals not involving intentions or goal-states of any kind 
(e.g.
	\citealp{Bennett:1976rg};
	\citealp{Butterfill:2001kc};
	\citealp{Schueler:2003fk};
	\citealp{Taylor:1964tr}).
If we don't think that goal-states are essential for there to be goal-outcomes associated with actions in the individual case, why think this must be so in the joint case?

Nevertheless, I have found that philosophers think this is consistency of these claims controversial.  And as we saw, quite a few philosophers insist that joint action necessarily involves shared intention.  

So in what follows I shall first argue for the consistency of these propositions by giving you some examples of joint actions without shared intentions.  I then offer an account of how joint actions are related to their goal-outcomes which does not involve anything like shared intention.  


\section{Preliminary: Shared Intention}

But before any of that I need to fill in some background on shared intention.

My immediate aim, as I mentioned, is to show that there are cases of joint action without shared intention.  An immediate obstacle is \textbf{lack of agreement on what shared intentions are}.  

Some hold that shared intentions differ from individual intentions with respect to the attitude involved (\citealp{Kutz:2000si}; \citealp{Searle:1990em}). 
Others have explored the notion that shared intentions differ with respect to their subjects, which are plural \citep{Gilbert:1992rs}, 
or that they differ from individual intentions in the way they arise, namely through team reasoning \citep{Gold:2007zd}, 
or that shared intentions involve distinctive obligations or commitments to others (\citealp{Gilbert:1992rs}; \citealp{Roth:2004ki}).
Opposing all such views, \citet{Bratman:1992mi,Bratman:2009lv} argues that shared intentions can be realised by multiple ordinary individual intentions and other attitudes whose contents interlock in a distinctive way. 

Given that almost no two philosophers agree on what shared intentions are, how can we say that shared intention is not involved in any given case?  

On all or most leading accounts of shared intention, each of the following is a necessary condition:

\begin{idescription}
\label{conditions-for-shared-intention}

\item[awareness of joint-ness] at least one of the agents knows that they are not acting individually; she or they have `a conception of themselves as contributors to a collective end.'\footnote{
	\citet[p.\ 10]{Kutz:2000si}.  Compare \citet[p.\ 361]{Roth:2004ki}: `each participant ... can answer the question of what he is doing or will be doing by saying for example ``We are walking together'' or ``We will/intend to walk together.''' 
Relatedly, \citet[p. 56]{miller_social_2001} requires that each agent believes her actions are interdependent with the other agent's.
}

\item[awareness of others' agency]  at least one of the agents is aware of at least one of the others as an intentional agent.\footnote{
	Compare \citet[p.\ 333]{Bratman:1992mi}: `Cooperation ... is cooperation between intentional agents each of whom sees and treats the other as such'.  See also \citet[p.\ 105]{Searle:1990em}: `The biologically primitive sense of the other person as a candidate for shared intentionality is a necessary condition of all collective behavior' 
}
\item[awareness of others' states or commitments] at least one of the agents who are F-ing is aware of, or has individuating beliefs about, some of the others' intentions, beliefs or commitments concerning F.\footnote{
This condition is necessary for shared intention even on what \citet[p.\ 40]{tuomela_collective_2000} calls `the weakest kind of collective intention'.  But it may not be necessary if, as \citet{Gold:2007zd} suggest, shared intentions are constitutively intentions formed by a certain kind of reasoning.
% "if the distinctive feature of collective intentions is to be found in the reasoning by which they were formed, then an analysis that focuses on the intentions themselves will miss the feature that makes collective intentions collective. " 
}

\end{idescription}

There are philosophers who deny that shared intention is necessary for joint action,\footnote
{
\citet[p.\ 407]{Roth:2004ki} and \citet{Searle:1990em}  hold that the intentions required for joint action need not be shared; \citet{miller_social_2001} also denies that shared intention is necessary for joint action.
*Should say something about Bratman and others.
*Should possibly also mention Kutz on participatory intentions.
}
but even they hold that one or more of these conditions is individually necessary for joint action (see footnotes above).

What follows assumes that where one or more of these three conditions is not met, there is no shared intention. 

Some of what follows also makes use of the further assumption that these conditions express causal conditions on shared intention.  That is, where joint action involves shared intention, the agents act in part \emph{because} they have awareness of joint-ness, of others' agency or of others' states or commitments.




\section{Preliminary: Joint Action}

Now we have necessary conditions for shared intention.  
Since I want to give examples of joint action without shared intention, I also need to give some sufficient conditions for joint action.  

The sufficient condition is already implicit in my project.

For there to be an interesting question about how a joint action is related to its goal-outcome, there has to be a sense in which all the agents' activities taken together have a goal where this isn’t simply a matter of each agent's activities individually having that goal. 

So the condition is this: 
\begin{quote}
Multiple agents' activities taken together have a goal-outcome which isn't just a matter of each agent's activities individually having that goal-outcome.
\end{quote}
I'm not saying that this a necessary condition for joint action.  I'm only saying that where this condition is met, there is an interesting question about how joint actions are related to their goals.

To see what sort of case this rules out, suppose that a large, brightly lit package is dropped from a helicopter.  Several people on the ground below attempt to catch the package.  These people are unaware of each other because there the helicopter has generated a mini dust storm and there is no coordination of their activities. 
Each person's goal is not that \emph{she} should catch the package but only that the package should be caught (perhaps they want to avoid its contents from breaking).  The package is large enough so that they could all have a hand in catching it, so it's possible for them all to succeed; but the package is also light enough so that one person could catch it alone, so the others are not necessary.   

As it happens, though, only one person does catch it.

In this case it is true that the people's activities were directed to the goal of catching the package.  But there is nothing more to this than the fact that each person's activities were individually directed to the goal of catching the package.



\section{Case Studies: Joint Action Without Shared Intention}


\subsection{Case Study---the environment coordinates joint actions}

Two ropes hanging over either side of a high wall are connected to a heavy block via a system of pulleys.  Ayesha and Beatrice each individually intend to raise the block.  
The positions of the walls mean that they can each see the block but they can't see each other.
Ayesha and Beatrice each know that, in addition to the rope they can pull, there is another rope and that force has to be exerted on that as well.  
But they have no idea about what will exert this force; for all they know it might be something mechanical or a fluke of nature rather than another agent.
They hold their own rope so that it's possible to feel when additional force is exerted on the other.  As soon as they feel such force, they attempt to lift the block.\footnote{
Compare the `Me plus X' notion from Vesper et al *ref
} 
This ensures that Ayesha and Beatrice pull the ropes simultaneously, causing the heavy block to rise as a common effect of their actions. 

None of the above necessary conditions for shared intention are met---there is no awareness of joint-ness, no awareness of others' agency and no awareness of others' states or commitments either.  
But intuitively this is joint action, perhaps because the ropes and pulleys bind the agents' actions together and ensure a common effect.  

We could make this example less bizarre by allowing that Ayesha and Beatrice can see each other.  As long as they are not aware of each other as agents, this still won't count as a case of shared intention.  

It is tempting to dismiss environmentally-provided coordination like that illustrated by Ayesha and Beatrice's raising the block as irrelevant to joint action; certainly philosophers tend to focus on activities, such as walking together, where acting together rather than in parallel could only be a matter of agents' attitudes or commitments \citep[e.g.][]{gilbert_walking_1990}.  But it is plausible that some coordination in joint action is provided by environmental structures rather than psychological mechanisms and, further, that humans perceive joint affordances when joint action is possible \citep{richardson_judging_2007}.  Even where shared intentions are present, how we perceive objects and events may be as important for effective joint action as what we know of other minds.

This artificial case indicates that in joint action much of the coordination can be taken care of by what objects afford multiple agents rather than by intentions. 



\subsection{Case Study---Kissing}
Initiating a kiss seems to be a good candidate for joint action because, like salsa dancing, it's not the sort of thing you can do alone.

But if the kiss is sufficiently spontaneous, initiating it might not involve awareness of jointness or awareness of others' states or commitments in advance of success.  In this case where, instead of shared intentions providing coordination, there are likely to be chemical and emotional means of coordination.



\subsection{Case Study---Motor Simulation}

Sam and Ahmed are sitting before a pile of wooden cubes.  They each individually intend that a stack be created from all the cubes.  They are indifferent to each other's presence and activities; they will be satisfied if the cubes all end up in a single stack in a way consistent with the fulfilment of their individual intentions.  Acting on their individual intentions, they rapidly pile the cubes into a single stack.

Necessary conditions for shared intention are not met: Sam and Ahmed do not know in advance whether they are acting individually or jointly [*first condition], and they need not be aware of each other's intentions or commitments regarding the activity [*third condition].  Yet their activity counts as a joint action in the minimal sense that their individual goals are fulfilled only as a common effect of both of their purposive actions.  

As so far described, this is not a compelling case of joint action.  To turn it into one we need to invoke motor simulation.  This calls for a little background.

It is now well established that some of the motor representations involved in planning and executing an action are also involved in observing that action.\footnote{
Some of the most direct evidence for this claim comes from \citet{Gangitano:2001ft} who artificially stimulated the motor cortices of subjects observing actions.  They found motor-evoked potentials related to the very muscles used in performing the observed action at the very times those muscles were needed for the task \citep[see further][]{Fadiga:2005gq}.  
}
The role of motor cognition in action observation appears to extend beyond matching a currently observed motor action to predicting subsequent motor actions based on the context of action \citep[e.g.][]{Iacoboni:2005ww,hamilton_action_2008}.  Among other functions, this is thought to enable agents to predict others' actions and their immediate outcomes \citep{Wolpert:2003mg,Wilson:2005qu}.  Such predictions in turn influence attention to action.  To take an example relevant to the present case study, \citet{Flanagan:2003lm} had subjects observe an agent stacking blocks.  They found that observers' gaze patterns were similar to those of the agent performing the action and dissimilar to those of control subjects who saw the blocks being stacked without seeing any actions.  In particular, observers tended to anticipate actions by gazing at blocks to be grasped and at the sites they were to be placed, just as they would if they themselves were performing the actions \citep[see further][]{Rotman:2006xf}.  
%Note that there is no reason to suppose that the observers in these experiments shared intentions with the agents they observed; observers were not asked to take part in the action.

How is this relevant to the present case study?  \textbf{Sam and Ahmed have to coordinate their actions because they are rapidly adding cubes to the same stack: they have to time their actions to avoid colliding with each other, and to avoid delay they have to anticipate where the other will place a cube when planning their own next move.}  Such coordination occurs on a timescale too short to be served by shared intentions.  Rather, as the above research demonstrates, it is motor simulation that makes this coordination possible.  \textbf{Sam and Ahmed can coordinate their actions with each other because, speaking loosely, each engages in motor planning for the other's actions as well as for his own}.  

In this case there is no shared intention because Sam and Ahmed need not think of themselves as contributing to a collaborative end.  Sam's and Ahmed's activities are coordinated thanks to meshing of their motor cognition rather than of their intentions.  This meshing ensures that the \textbf{two agents' actions resemble in some respects those of a single agent performing an action with two hands}.  In these respects Sam and Ahmed are acting as one, which shows that this is a significant case of joint action without shared intention.  

The first case study illustrated how joint action sometimes relies on shared affordances.  This case study illustrates how, in other cases, joint action relies on meshing motor cognition.  To insist that joint action always involves shared intention would mean neglecting other psychological mechanisms involved in the coordination of joint action.



\section{Collective Goals}
\label{section_collective}

So far I have argued that there are goal-directed joint actions without shared intentions.  
In the examples I offered, necessary conditions for shared intention are not met---there is no awareness of joint-ness, no awareness of others' agency or no beliefs about others' states or commitments.  But the examples do involve joint action.  For there is a sense in which multiple agents' activities taken collectively are directed to a goal, where this is not just a matter of each agent's activities being individually directed to that goal.
My question is, How can we characterise the relation between joint actions and their goal-outcomes without invoking shared intentions?  

\subsection{[skip: Aggregate Agents]}
In principle, one possibility would be to appeal to aggregate agents.
\emph{Aggregate agents} are agents among whose parts are two or more agents 
%this is to rule out `Russian doll' cases
none of which are parts of the others. 
(`Aggregate' is used here  because in zoology an aggregate animal is one consisting of distinct animals; the notion is related to Gilbert (\citeyear{Gilbert:1992rs})'s \emph{plural subject} *cite Helm too.)  
Suppose that in a joint action the agents involved were to constitute a single aggregate agent.  
Then the standard story about how individual actions relate to their goal-outcomes would apply without modification to these joint actions too.  On this view, the only difference between joint and individual actions is that the agents of joint actions are aggregate.  Special features of joint actions would become special features of their agents.

Applied to this case I do not believe that introducing aggregate agents will help.  As far as I can see there is no justification for supposing that aggregate agents are involved in the cases I have discussed.  While I don't have any objection to them in principle, it seems unlikely that the theoretical notion of an aggregate agent will be useful here.


\subsection{Collective Goals}
So how else could we characterise the relation between joint actions and their goal-outcomes without either invoking shared intentions or aggregate agents?  

I'm going to  answer this question slowly and starting from quite far back.

One necessary condition for joint action is that there must be a single goal-outcome to which the agents' activities are all directed; or at least there must be a single outcome such that
(a) the outcome's occurrence is contingent on some or all of the agents' activities,
(b) for each agent, the outcome's obtaining is either a goal to which her activities are individually directed or else it is a means to her achieving one of these goals,
and
(c) no agent achieves the goals to which these activities are individually directed unless the outcome obtains.
Let a \emph{distributive goal} of a joint action, or of any collection of goal-directed activities, be an outcome with these properties.  Then a basic necessary condition is that joint actions have at least one distributive goal.\footnote{
Note that this does not imply that all of the agents' activities must be directed to goals that can all obtain together.
The above condition concerns only those of their  activities which are part of  the joint action.  To illustrate,  Shireen and Jamel are stranded on a small island with a boat that will carry just one of them.  Only by leaving alone can either survive.  But the boat is too far from the sea for either of them to launch it alone.  They carry the boat to the sea together, each intending to kill the other in order to sail away.  Carrying the boat to the sea is a joint action.  The goal-directed activities which constitute this joint action are such that their goal-outcomes---to get the boat to the sea---can all obtain together.  It is not relevant here that the agents have further goals which cannot all obtain together.  
}

\marginpar{ \tiny \flushleft
[*Slide: include the term `distributive goal']
}

This condition is met where, at a village festival, many revellers each jump into a single boat with the goal of sinking it.  This condition is also met where Sal and Terry break into the town bank together, Sal with the goal of stealing its money, Terry with that of stealing some documents.  Here the distributive goal is breach of the bank's defences.\footnote
{
You may hold that Sal and Terry must each individually have the goal of breaching the bank's perimeter so that this example is not interestingly different from the preceding example.  
The example is included at the risk of redundancy in order to make the paper neutral on this issue.
}  
But the condition is not met by virtue of multiple agents acting on similar goals.  In an example from Searle (\citeyear[p.\ 92]{Searle:1990em}), rain causes park visitors simultaneously to take cover under a central shelter.  Each is acting independently of the others.  Their activities do not have a distributive goal as defined above.  The visitors' activities are all directed to different goal-outcomes---each visitor's own arrival at the shelter---which can occur independently of each other.  No outcome satisfies (a)--(c) above.


Our aim is to explain the relation between joint actions and their goals without appeal to shared intention.  The notion of distributive goal is not sufficient for this purpose.  
Why not?
Interesting cases of joint action are cases where multiple agents' activities non-distributively have a goal.  It's not sufficient that there is one-goal outcome around which each agent's activities are individually organised.  Rather, there has to be a sense in which all the activities taken together have a goal where this isn't simply a matter of each activity having that goal.
The notion of distributive goal isn't sufficient for our purposes.

Maybe you can remember the example I gave earlier where a helicopter drops a brightly lit package which several people on the ground below try to catch.  In this case there is a distributive goal.  But there is no interesting sense in which these people's activities taken together have a goal.



\marginpar{ \tiny \flushleft
Alternative structure is possible.  Idea would be to have case where there is no common effect as pure distributive goals.  Hoodlum-bus example introduces common effect.  Then other examples include coordination as well.
}

\begin{comment}
%cut this bit --- the example I need is the helicopter package
To illustrate, suppose some hoodlums each individually set about destroying a bus.  They are not members of a group; the coincident timing and target of their actions is due only to the bus' salience and the time of day.  Each hoodlum's goal is the bus' destruction.  They are unconcerned about, and largely unaware of, each other's involvement.  Their activities are uncoordinated, each rendering others' activities less effective and increasing risk of injury.  The bus' destruction is a distributive goal of their activities.  
Is there a sense in which their activities taken together have a goal which isn't just a matter of each agent's activities individually having that goal?
Perhaps there is because, as it happens, their distributive goal is achieved as a common effect of all of their activities.
But this is either not a joint action or else a very minimal form of joint action.  The examples of joint action without shared intention which I gave above are joint actions in a slightly richer sense.
\end{comment}

\reversemarginpar
\marginpar{ \tiny \flushright
*TODO (in paper version): it's wrong to give an illustration here because the point is narrowly logical.  The notion of a distributive goal just requires a distributive reading and what I am saying here is that we are interested in cases where there is a non-distributive sense in which multiple agents' activities have a goal-outcome.
} 
\normalmarginpar

What, in addition to a distributive goal, could support the ascription of a goal to two or more agents' activities taken together?

Consider two conditions which are met in some cases of joint action.  
I don't think these conditions are met in every case of joint action---they are not supposed to be necessary conditions.
\marginpar{ \tiny \flushleft
*Should we remove this condition for collective goals?  Doesn't seem to be required for the argument.  Also removes objections to seeing common goals as a generalisation of shared intentions.     
}
First, the distributive goal obtains, or would normally obtain, as a common effect of all of the agents' goal-directed activities; or else the distributive goal's obtaining is, or would normally be, constituted by these activities.  
%possible generalization:
%	for each of the agent's activities, there is, or was at the outset, a chance that these activities would cause the outcome
Second, the agents' activities are coordinated and the distributive goal occurs, or would normally occur, partly as a consequence of this coordination.  Let a \emph{collective goal} of a joint action, or of any collection of goal-directed activities, be a distributive goal meeting these two conditions.

Examples of activities that would typically have collective goals include uprooting a small tree together and kissing a baby together.
By contrast,
in helicopter-package example above there is no collective goal.  This is because the agents' activities are not coordinated and the outcome does not occur as a common effect of all of their activities.  Neither of the conditions on common goals is met but the second is not.  

The word `collective' in `collective goal' should not be understood to imply that the agents  involved constitute a collective in any social sense, nor that the agents must think of themselves as having a collective goal.  The use of `collective' is by analogy with literature on plural quantification where it contrasts with `distributive'.  The point is that for multiple agents' goal-directed activities to have a certain collective goal is not equivalent to each of their activities separately having that goal.


Where a joint action or collection of goal-directed activities has a collective goal there is a sense in which, taken together, the activities are directed to their collective goal.  It is not just that each agent individually pursues the collective goal (or some appropriately related goal---see the definition of \emph{distributive goal} above); in addition, there is coordination among their activities which plays a role in bringing about the collective goal.  We can put this in terms of the direction metaphor.  Any structure or mechanism providing this coordination is directing the agents' activities to the collective goal.  The notion of a collective goal therefore already provides one way of making sense of the idea that joint actions are goal-directed actions.

The approach to explaining the relation between a joint action and its goal-outcome can now be stated.  A joint action is related to the goal-outcome to which it is directed in virtue of this goal-outcome being the collective goal of the joint action.  The goal-outcome of a joint action is its collective goal.\footnote{  
Just as it is possible that some joint actions involve multiple shared intentions, so also might some involve involve multiple collective goals.  This minor complication will be ignored for ease of exposition.
}


\subsection{How collective goals differ from shared intentions}
Note that collective goals are primarily properties of activities; they are not states of agents.  

As this implies, collective goals are conceptually distinct from shared intentions.  A shared intention is something that resembles an intention in two respects.  It is, or resembles, a state of the agents of a joint action.  And it coordinates multiple agents' activities.  Appeal to a shared intention is therefore potentially a way of explaining how agents coordinate their activities.  By contrast, collective goals (as defined here) are not the sort of thing that can coordinate anything.  In addition, shared intentions are generally supposed to be shared by agents in something resembling the sense in which conspirators can share a secret, not only in the sense in which two people can share a name.  It would be a distortion to claim that the notion of a collective goal is the notion of something agents share.

We should not infer from this alone that there are, or even that there might be, collective goals without shared intentions.  That collective goals and shared intentions are conceptually distinct does not imply that one can exist without the other.


\subsection{[*skip: shared intentions without collective goals]}
But there do seem to be cases where there are shared intentions but not collective goals.

\marginpar{ \tiny \flushleft
*I'm no longer sure this is right.  Why make common effect a condition for collective goals?
}

Suppose that three knights tasked with finding a sacred cup agree to search different continents and one of them finds it.
They have a shared intention to find the cup.  
In this case their activities have a distributive goal, namely that of finding the cup.
There is also coordination among the knights.
But the outcome does not occur as a common effect of all of the knights' activities and so it is not a collective goal of these activities taken together.\footnote
{
It is a stipulation of this example that each knight is caused to restrict his activities to a particular domain by the intentions of the others but not by their activities.  This ensures that the others' activities are not, even indirectly, causes of finding the cup.
}  

In many cases the collective goal of a joint action will correspond to the content of a shared intention.  But, as the three-knights example shows, the fact that  agents are acting on a shared intention does not entail that their activities have a collective goal.




\subsection{joint actions with neither shared intentions nor collective goals}

[*MODIFY CONCLUSION OF THIS SUBSECTION: either the contestants do have a shared intention, in which case it's possible to have shared intentions without collective goals; or else they don't have a shared intention in which case sometimes joint actions have non-distributive goals which are neither collective goals nor shared intentions.]

Consider a team of three quiz show contestants who together win a round.
That their team won the round was a goal-outcome of each player's activity, so this is a distributive goal.  
Furthermore,  and the winning is a common effect of all of their activities.
But their activities are not coordinated.  Each player simply attempts to answer whatever questions she thinks she might be able to answer with no regard to other players' behaviour or expertise.
This lack of coordination prevents the team's winning from being a collective goal.
It may also prevent the players from acting on the shared intention that they win the round.  Because, on at least one account, shared intention requires intentions to succeed by way of meshing subplans.  The disposition of each player to answer questions without regard to the others shows that there are no meshing subplans.
So here there seems to be neither a collective goal nor a shared intention.

Despite this, it seems appropriate to describe the team as having attempted to win and succeeded in doing so.  In what way could this be appropriate?

Aggregate agents: ascription is supported by existence of team and structure of the quiz, which treats the team as a unit.

 


\subsection{Are all activities with collective goals joint actions?}
Some collections of activities have a collective goal but are not joint actions.  For example, two strangers walk towards each other down the middle of a street.  In order to pass, each coordinates their sideways movements with the other's.  Their activities have a distributive goal, that of passing without collision.  If the agents simulate each other's motor activities then there may be a sense in which their activities are coordinated and this coordination may be among the causes of their passing each other.  In that case their activities have a collective goal.






\section{Conclusion}
The notion of joint action involves extending theories from the individual, single-agent case to cases involving multiple agents.  There is no straightforward way to so extend these theories, any extension will be truer to some features of individual action at the expense of abandoning others.  Further, no single extension distinguishes itself as uniquely correct.

 

\bibliography{$HOME/endnote/phd_biblio}

\end{document}