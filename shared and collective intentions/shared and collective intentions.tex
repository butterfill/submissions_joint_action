%!TEX TS-program = xelatex
%!TEX encoding = UTF-8 Unicode

\documentclass[12pt,a4paper]{extarticle}
% extarticle is like article but can handle 8pt, 9pt, 10pt, 11pt, 12pt, 14pt, 17pt, and 20pt text

\def \ititle {Joint Action:}
\def \isubtitle { Shared Intentions and Collective Goals}
\def \iauthor {Stephen A. Butterfill}
\def \iemail{s.butterfill@warwick.ac.uk}
%\date{}

\input{$HOME/Documents/submissions/preamble_steve_paper}

\begin{document}

\setlength\footnotesep{1em}

\bibliographystyle{newapa} %apalike

\maketitle
%\tableofcontents

\begin{abstract}
%***OUTOFDATE
A joint action is a goal-directed action comprising two or more agents' goal directed activities.  What is the relation between a joint action and the goal or goals to which it is directed?  
Much research on joint action is guided by the assumption that this relation constitutively involves shared intentions or other shared states.  
This paper shows that assumption to be false.
The relation between joint actions and the goals to which they are directed can be  explicated without appeal to shared intention.  
*Consequence
\end{abstract}


\section{Introduction}
A joint action is a goal-directed action, or something resembling one, comprising two or more agents' goal-directed activities.  What is the relation between a joint action and the goal (or goals) to which it is directed?  
Answering this question is necessary for saying what joint actions are and for understanding in what senses, if any, they are actions.
%*better sentence on significance of question?

%(Here and throughout we ignore for simplicity the possibility that joint actions may be directed to multiple goals.  The answers considered are consistent with this possibility.)

To illustrate, suppose Ayesha and Beatrice between them lift a table thereby releasing a cat whose paws were trapped and,  simultaneously, breaking a glass.  On standard theories of events, the event which initiated the cat's release is identical to the event which initiated the glass' destruction \citep{Davidson:1969ie}.  But observing this joint action we might wonder whether its goal was the release of the cat or the destruction of the glass (or both); and if we are in doubt about this then in an important sense we don't yet know which action Ayesha and Beatrice performed.  To identify their action we need not only a concrete event, something with temporal and spatial properties, but also an abstract outcome such as that of the cat's release \citep{Davidson:1971fz}.

Ayesha, asked about the table lifting episode, might insist, `The goal of our intervention wasn't to smash the glass but to free the cat.'  (Note that this statement concerns the goal of the joint action and not explicitly the agents' intention.)  Applied to this example, the question can be put like this.  What is it for the goal of Ayesha and Beatrice's action to be that of releasing the cat rather than that of breaking the glass?

The term `joint action' might be used in different ways, much as the term `action' sometimes encompasses habitual actions or motor behaviours but in other cases is used more narrowly to refer to intentional actions only.  We are presently interested in how actions relate to goals and, specifically, in the possibility that the involvement of multiple agents complicates a true account of this relation.  Accordingly we stipulate a sufficient condition for joint action.  For multiple agents' purposive activities to constitute a joint action it is sufficient that these activities are directed to a goal where this is not, or not only, a matter of each agent's activities individually being directed to that goal.   

The term `goal' has been used both for outcomes (as in `the goal of our struggles') and, perhaps improperly, for psychological states (it is in this sense that agents' goals might cause their actions).  In this paper the term `goal' is used in the former sense only, and the terms `goal-outcome' and `goal-state' are occasionally used to stress the distinction.  Our question is about how joint actions are related to their goal-outcomes.  

For individual (that is, not joint) action, the counterpart of this question has a standard answer.  In barest outline, an individual action consists in an agent acting on an intention (a goal-state).  The intention has a content which specifies an outcome.
%, namely the outcome whose occurrence would consist in the propositional content's being true.
This outcome is a goal of the action.
It is a goal of the action in virtue of being specified by an intention the agent is acting on. 

%In essence, then, on the standard theory, an individual action is related to its goal-outcome in virtue of the agent's acting on one or more intentions.

How do things stand in the case of joint action?  Can we extend the standard theory from individual to joint action?  

The usual way of thinking about joint action starts with the premise that all significant cases of joint action involve shared intention (dissenters are mentioned below).  This is taken for granted in a recent paper by Alonso:
\begin{quote}
`the key property of joint action lies in its internal component [...] in the participants’ having a “collective” or “shared” intention.' \citep[pp. 444-5]{alonso_shared_2009}
\end{quote}
And Gilbert is explicit on this point:  
\begin{quote} 
`I take a collective action to involve a collective intention.'  \citep[p.\ 5]{Gilbert:2006wr}\footnote{
For the purposes of this paper we can treat `collective intention' as synonymous with `shared intention'.  In introducing the term `collective action', Gilbert stipulates that collective action involves more than multiple agents performing concurrent actions while leaving open what more is involved (p.\ 4).   As I use the term `joint action', this is true of all joint actions and so all joint actions are collective actions in this sense. 
}
\end{quote}
But what is shared intention?  
In barest outline, shared intentions are supposed to do for multiple agents some of what ordinary intentions do for individuals.  So, like an ordinary intention, a shared intention coordinates plans and activities---with the difference that these are the plans and activities of multiple agents performing a joint action \citep{Bratman:1993je}.


Notions of shared intention provide a straightforward way to extend the standard theory about the relation between individual actions and their goal-outcomes to the case of joint action.  
Given a notion of shared intention and an account of what it is for agents to act on a shared intention, 
the standard theory is extended by substituting shared for individual intentions.  The shared intention specifies an outcome.  This outcome is a goal to which the joint action is directed in virtue of being the outcome specified by a shared intention on which the agents are acting.

While there is debate on what shared intention is (e.g. 
	\citealp{Kutz:2000si}; 
	\citealp{Tollefsen:2005vh}; 
	\citealp{gilbert_walking_1990};
	\citealp[pp.\ 74--81]{miller_social_2001};
	\citealp{tuomela_collective_2000}), 
there have been few attempts to explain the relation between joint actions and their goals which involve nothing like shared intention 
(one exception, which differs from the present work in motivation and substance, is \citealp{Roth:2004ki}).

There is a need for explanations of the relation between joint actions and their goals which do not invoke anything like shared intentions.  
How do we know this?
There is a basic necessary condition on shared intention.  
For agents to share an intention it is necessary that they each believe or know that they are not acting individually or have knowledge, beliefs or intentions from which this can be inferred.
Despite disagreements about the nature of shared intention,  most or all leading accounts agree in treating this as a necessary condition for shared intention   
 (e.g. \citealp[p.\ 106]{Bratman:1993je}; \citealp[p.\ 10]{Kutz:2000si}).  
Where this condition is not met, nothing like shared intention is present.
But there do seem to be cases of joint action where this condition is not met.
That is, there seem to be cases where this necessary condition on shared intention is not met but the above sufficient condition for joint action is met---multiple agents' activities are directed to goal and this is not only a matter of each agent's activities individually being directed to that goal.
The existence of such cases is suggested by
a range of scientific research about joint action in which shared intention appears to play no role 
	\citep[as reviewed in][]{%
		vesper_minimal_2010,
		Knoblich:2010fk,
		Sebanz:2006yq%
	},
as well as by a priori arguments for the possibility of joint action without shared intention \citep{petersson_collectivity_2007}.  This is why an explanation of the relation between joint actions and their goals that does not invoke shared intention is needed.

That such an explanation is needed does not entail that all explanations involving shared intention are incorrect.
Returning briefly to the case of individual action, there are accounts of the relation between actions and their goals not involving intentions or goal-states of any kind 
(e.g.
	\citealp{Bennett:1976rg};
	\citealp{Butterfill:2001kc};
	\citealp{Schueler:2003fk};
	\citealp{Taylor:1964tr}).
These accounts are not invariably motivated by scepticism about intention or about the standard theory.  Rather, the motivating claim is sometimes that, while many goal-directed actions do involve intentions, there are also goal-directed actions not involving intentions.  If this were correct it would arguably be necessary to develop multiple approaches to characterising the relation between individual actions and their goals, some involving intentions and some not involving intentions.  Independently of whether or not this is correct in the individual case, something similar may be true in the case of joint action.  

This paper provides an account of how joint actions relate to their goals without invoking anything like shared intentions.  
Its purpose is to provide a conceptual framework for research on joint actions without shared intentions, just as an account of shared intention provides a framework for investigation of joint action involving shared intention \citep{Bratman:2009lv}.  
%Our central innovation is the definition and discussion of \emph{collective goal} in section \vref{section_collective}.  

%Aside from this motivation, exploration of different ways in which joint actions could be related to goals may have an intrinsic interest, for in advance of such exploration it cannot be known whether it is coherent to suppose that joint actions could be goal-directed without involving shared intentions.\footnote{
%*cite people saying all joint action involves shared intention without argument
%}

%The strategy of this paper is informed by the conjecture that joint action does not resemble a natural kind in the sense of being a single category already out there to which claims about its nature are answerable.  Instead philosophers start with a partial, intuitive notion plus some fruitful scientific research (see the reviews cited above and \citealp{schmidt_understanding_2010}).  Our task is to investigate different conceptual accounts of what joint action is and to distinguish those which are merely coherent from those which are also empirically motivated.  Since there is no advance justification for assuming that there is only one theoretically coherent and empirically motivated notion of joint action, we should explore multiple approaches to answering the question about the relation between a joint action and its goal.  

%Too ambitious:
%This paper argues that there are two or more ways of explaining the relation between a joint action and the goal to which it is directed, each useful in its own right.  



\section{Collective Goals}
\label{section_collective}

How can we explain the relation between some joint actions and  their goals without invoking shared intentions?  

In many (and perhaps all) cases of joint action there is a single outcome to which each of the agents' activities are individually directed.  Let a \emph{distributive goal} of a joint action, or of any collection of goal-directed activities, be any such outcome.  

To illustrate, suppose that many revellers at a village festival jump into a single boat each with the aim of sinking it, where each reveller's aim would be met if the boat were to sink under their collective weight.  
That the boat sink is a distributive goal of their activities.

For a slightly more complicated case, suppose that Sal and Terry break into the town bank together, Sal with the aim of stealing its money, Terry with that of stealing some documents.  Here the distributive goal is breach of the bank's defences.\footnote
{
You may hold that Sal's and Terry's activities must individually have been directed to the goal of breaching the bank's perimeter so that this example is not interestingly different from the preceding example.  
The example is included at the risk of redundancy in order to remain neutral on this issue.
}  



But the condition is not met just by virtue of multiple agents acting on similar goals.  In an example from Searle (\citeyear[p.\ 92]{Searle:1990em}), rain causes park visitors simultaneously to take cover under a central shelter.  Each is acting independently of the others.  Their activities do not have a distributive goal as defined above.  The visitors' acitvities are all directed to different goal-outcomes---each visitor's own arrival at the shelter---which can occur independently of each other.  No outcome satisfies (a)--(c) above.


Our aim is to explain the relation between joint actions and their goals without appeal to shared intention.  The notion of distributive goal is clearly not sufficient for this purpose.  Too see why, suppose some hoodlums each individually set about destroying a bus.  They are not members of a group; the coincident timing and target of their actions is due only to the bus' salience and the time of day.  Each hoodlum's goal is the bus' destruction.  They are unconcerned about, and largely unaware of, each other's involvement.  Their activities are uncoordinated, each rendering others' activities less effective and increasing risk of injury.  The bus' destruction is a distributive goal of their activities but we should not suppose that this outcome is a goal of their activities taken together.  As described, there is no significant sense in which, taken together, their activities are directed to a goal.

What, in addition to a distributive goal, could support the ascription of a goal to two or more agents' activities taken together?

Consider two conditions which are met in some cases of joint action.  At this point we leave open whether these conditions are necessary or sufficient conditions for joint action: for now they are simply conditions which sometimes obtain in cases of joint action.
First, the distributive goal obtains, or would normally obtain, as a common effect of all of the agents' goal-directed activities; or else the distributive goal is, or would normally be, constituted by these activities.  
%possible generalization:
%	for each of the agent's activities, there is, or was at the outset, a chance that these activities would cause the outcome
Second, the agents' activities are coordinated and the distributive goal occurs, or would normally occur, partly as a consequence of this coordination.  Let a \emph{collective goal} of a joint action, or of any collection of goal-directed activities, be a distributive goal meeting these two conditions.

Examples of activities that would typically have collective goals include uprooting a small tree together and kissing a baby together.
By contrast,
in hoodlum-bus example above there is no collective goal.  This is because the hoodlums' activities are not coordinated.  The first of the two conditions on common goals is met but the second is not.  

There are also cases where the first but not the second of these conditions is met.  Suppose that three knights tasked with finding a sacred cup agree to search different continents and one of them finds it.  A distributive goal of their activities is the finding of the cup.  In this case there is coordination but the outcome does not occur as a common effect of all of the knights' activities and so it is not a collective goal of these activities taken together.\footnote
{
It is a stipulation of this example that each knight is caused to restrict his activities to a particular domain by the intentions of the others but not by their activities.  This ensures that the others' activities are not, even indirectly, causes of finding the cup.
}  

In many cases the collective goal of a joint action will correspond to the content of a shared intention.  But this is not invariably so.  For example, the three knights might be acting on the shared intention that they find the cup.  The fact that  agents are acting on a shared intention does not entail that their activities have a collective goal.
%?discuss converse?

The word `collective' in `collective goal' should not be understood to imply that the agents  involved constitute a collective in any social sense, nor that the agents must think of themselves as having a collective goal.  The use of `collective' is by analogy with literature on plural quantification where it contrasts with `distributive'.  The point is that for multiple agents' goal-directed activities to have a certain collective goal is not equivalent to each of their activities separately having that goal.


Where a joint action or collection of goal-directed activities has a collective goal there is a sense in which, taken together, the activities are directed to their collective goal.  It is not just that each agent individually pursues the collective goal (or some appropriately related goal---see the definition of \emph{distributive goal} above); in addition, there is coordination among their activities which plays a role in bringing about the collective goal.  We can put this in terms of the direction metaphor.  Any structure or mechanism providing this coordination is directing the agents' activities to the collective goal.  The notion of a collective goal therefore already provides one way of making sense of the idea that joint actions are goal-directed actions.

The third approach to explaining the relation between a joint action and its goal-outcome can now be stated.  A joint action is related to the goal-outcome to which it is directed in virtue of this goal-outcome being the collective goal of the joint action.  The goal-outcome of a joint action is its collective goal.\footnote{  
Just as it is possible that some joint actions involve multiple shared intentions, so also might some involve involve multiple collective goals.  This minor complication will be ignored for ease of exposition.
}

Collective goals are conceptually distinct from shared intentions.  As mentioned above, a shared intention is something that resembles an intention insofar as it coordinates multiple agents' activities.  Appeal to a shared intention is therefore potentially a way of explaining how agents coordinate their activities.  By contrast, collective goals (as defined here) are not the sort of thing that can coordinate anything.  In addition, shared intentions are generally supposed to be shared by agents in something resembling the sense in which conspirators can share a secret, not only in the sense in which two people can share a name.  It would be a distortion to claim that the notion of a collective goals is the notion of something agents share.

We should not infer from this alone that there are, or even that there might be, collective goals without shared intentions.  That collective goals and shared intentions are conceptually distinct does not imply that one can exist without the other.

*Have left it open that there are joint actions without collective goals; if this turns out to be so this account may lack scope.  Suspect that this will be the case: shared intentions can bind together otherwise disparate activities.

*Have left it open that some collections of activities have a collective goal but are not joint actions.  For example, two strangers walk towards each other down the middle of a street.  In order to pass, each coordinates their sideways movements with the other's.  Their activities have a distributive goal, that of passing without collision.  This distributive goal is a collective goal because it depends on both agents' activities and on their coordination.







There is reason, though, to suppose that some cases of joint action involve shared intentions without collective goals; and, conversely, that some cases involve collective goals without shared intentions.




Suppose a joint action does involve a shared intention.  How does the shared intention relate to its collective goal?  Let us make some relatively uncontroversial assumptions about shared intentions.  
First, where a shared intention involves a goal-outcome, each agent who has this shared intention also individually intends that the goal-outcome occur.  
Second, agents act on this individual intention when they act on a shared intention.  
These assumptions entail that the goal-outcome specified by a shared intention will be a distributive goal of the joint action (\emph{distributive goal} is defined above).
This leaves open two issues.  
First, it is apparently possible that agents could could succeed in acting on a shared intention without the associated goal-outcome being a common effect of all of their activities.  For example, *defenders take position but not all positions are attacked (and wouldn't have been attacked even if the defender had been absent).  Complications would arise if shared intention required causal interdependence among agents' activities, but it is unclear why shared intention should require this (as opposed to causal dependence among the agents' individual intentions).  In this case there is no collective goal or the collective goal is not the goal-outcome specified by the shared intention.
Second, it is apparently possible that agents could succeed in acting on a shared intention without coordination among their activities being a cause of the outcome.

***HERE.  Proposal is that we should think of collective goals and shared intentions as neither necessary nor sufficient for each other.


***gameshow constentants (one might not answer any questions; their efforts might be entirely uncoordinated).

*This is a generalisation of the story about shared intention!  Wrong to see it as an alternative.





*This allows for separation of coordinative and individuative mechanisms.  One can be performed by overlapping goals, the other can be performed by motor cognition (e.g.).

*Why not say that this is a case of shared intentions?  Because co-incident is not sharing (except in the trivial sense that we can share a name---perhaps it's important to distinguish in what sense the goals/intentions are shared rather than to say categorically that it is not shared); because intentions coordinate; because none of the three necessary conditions on shared intention need to be met.




\subsection{joint actions with neither shared intentions nor collective goals}
Consider a team of three quiz show contestants who together win a round.
That their team won the round was a goal-outcome of each player's activity, so this is a distributive goal.  
Furthermore,  and the winning is a common effect of all of their activities.
But their activities are not coordinated.  Each player simply attempts to answer whatever questions she thinks she might be able to answer with no regard to other players' behaviour or expertise.
This lack of coordination prevents the team's winning from being a collective goal.
It may also prevent the players from acting on the shared intention that they win the round.  Because, on at least one account, shared intention requires intentions to succeed by way of meshing subplans.  The disposition of each player to answer questions without regard to the others shows that there are no meshing subplans.
So here there seems to be neither a collective goal nor a shared intention.

Despite this, it seems appropriate to describe the team as having attempted to win and succeeded in doing so.  In what way could this be appropriate?

Aggregate agents: ascription is supported by existence of team and structure of the quiz, which treats the team as a unit.

 

\section{Alternative Approach---Aggregate Agents}
%aggregate in OED: 4. Zool. Consisting of distinct animals united into a common organism.

Suppose there were \emph{aggregate agents}, that is agents among whose parts are two or more agents 
%this is to rule out `Russian doll' cases
none of which are parts of the others. 
(`Aggregate' is used here  because in zoology an aggregate animal is one consisting of distinct animals; the notion is related to Gilbert (\citeyear{Gilbert:1992rs})'s \emph{plural subject} *cite Helm too.)  
Suppose, further, that, in every joint action, the agents involved were to constitute a single aggregate agent.  
Then it would be possible to have a notion of joint action which is, conceptually, exactly like individual action except that the agent is aggregate.  Special features of joint actions would become special features of their agents.

On this approach, the standard theory about the relation between actions and their intentions does not need modification: it already applies to joint actions.  A joint action is related to its goal by virtue of the aggregate agent's acting on an intention whose content specifies that goal.  

This approach amounts to taking at face value things people sometimes say about their own joint actions.  Ayesha, asked about the table lifting episode mentioned above, might insist, `Our intention wasn't to smash the glass but to free the cat.'  If she and Beatrice constituted an aggregate agent we could take this ascription of intention at face value.

Opposing this approach some may deny the existence of aggregate agents consisting of primates.  Or, taking a different line, opponents may assert that introducing aggregate agents postpones or even obscures the distinctive empirical and philosophical problems associated with joint action.  These lines of objection raise deep and difficult questions about agency which are beyond the scope of this paper.  Fortunately the central claims of this paper also permit us to be almost neutral on the correctness or not of these objections.  For our purposes, the potential challenge is the possibility that appeal to aggregate agents provides a single, unified approach to understanding how joint actions relate to their goals and so renders alternatives unnecessary.  

To rule out this possibility we need not show that no aggregate agents exist.  It is sufficient to establish that there are some joint actions where the agents do not constitute a single aggregate agent whose intention fixes the goal of the joint action.  Consider the case of an audience clapping together after a performance \citep{Bratman:2009lv}.  As we will see, the members of the audience do not always constitute an aggregate agent whose intention fixes the goal of their joint action.

How can this claim be established?  In some cases joint actions happen in part because of, and in accordance with, individual agents' beliefs, desires and intentions.  In the case of the audience clapping, suppose that individuals clap more or less depending in part on how much overall clapping they think there is in proportion to how good they felt the performance was.  (The stipulation does not exclude other causes of the individuals' clapping; perhaps shared emotional experience is also important.)  Accordingly, the way their joint action unfolds is causally and rationally related to individual agents' psychological states.  If introducing an aggregate agent is to adequately explain joint actions of this kind, the aggregate agent's psychological states (or at least those on which it acts) must be related to the constituent agents' psychological states.  For example, shifts in how much a constituent agent wants to clap must cause or constitute a corresponding shift in how much the aggregate agent wants to applaud.  This is a first requirement on aggregate agents.


Note that this first requirement is not offered as a general requirement on aggregate agents.  
%Presumably motivation for introducing aggregate agents includes the thought that there may be cases of joint action which seem to involve an aggregate agent with a mind of its own, one whose beliefs and desires are not related in any simple, lawlike way to the desires of the individual agents  composing it.\footnote{
%While Sugden's views do not commit him to aggregate agents, he does argue that multiple agents' can have collective preferences that are not reducible the agents' individual preferences \citep{Sugden:2000mw}.  
%}
%The first requirement does not conflict with this.  
The first requirement is just that \emph{if} an aggregate agent is postulated in the audience clapping case as described above (or any relevantly similar case), then changes in a constituent agent's beliefs, desires or intentions relevant to the joint action must cause or constitute corresponding shifts in  \emph{this} aggregate agent's states.

A second requirement on aggregate agents is that their introduction should not *.  Suppose that an individual audience member gradually raises their estimation of the performance so that (because all other things are equal) there is a corresponding increase in the volume of applause.  This increase reflects a change in the aggregate agent's behaviour.  The explanation of this change should not involve factors other than the change in the aggregate agent's desires that were triggered or constituted by the change in the constituent agent's desires.  In other words, the mere existence of an aggregate does not prevent changes in constituent agents from causing simple additive changes in the behaviour of the group.

A third requirement on aggregate agents is more general.  All agents have intentions or goal-states of some kind.  In order to have goal-states an agent must also have related beliefs, desires and perhaps other psychological states as well.  These states must be capable of interacting with each other in order to cause and rationalise actions which the agent performs.  So while an aggregate agent, like an individual agent, will not necessarily satisfy norms of rationality, it must at least be capable of doing so. 

These two requirements on aggregate agents cannot both be satisfied in cases such as the audience's clapping described above.  To see why, consider an artificially simple elaboration.  
For half the audience, each individual's beliefs cause and justify loud applause even for average performances; the other half reject this liberal attitude and are more stringent in how loudly they will clap.  
Initially those in the liberal half of the audience each rate the performance as poor whereas those in the other, more stringent half each rate the performance as average.  
The upshot is restrained applause.  
But as the audience members recall different aspects of the performance, those in the stringent half gradually lower their estimation of its quality while those in the liberal half do the opposite.  
These changes result in a steadily increasing volume of applause.  The aggregate agent is now applauding more than it was before.  What explains this shift?  
There has not been a shift in how good the audience as a whole thinks the performance was, nor has any individual's attitude towards how much applause performances of a given standard merit.


%Suppose there are two respects in which the performance was particularly brilliant where most of the audience only recognises one of these, with about half recognising each.  For each individual, we stipulate that her limited appreciation of the performance together with her other beliefs and desires causes and justifies her clapping activities.  To make the aggregate agent similarly rational is not straightforward, but we might attempt this by saying that the aggregate agent is dimly aware of both brilliant aspects of the performance.

On the one hand, changes in any component agent's states trigger or constitute corresponding changes in the aggregate agent's states.  

entail that there must be rational coordination among the psychological states of the agents constituting an aggregate agent.  But in the case of an audience clapping (for example), there need be no such coordination.  So there are at least some cases of joint action to which the present approach, which hinges on aggregate agents' intentions, does not apply.

*General point: hard theoretical limit on what aggregate agents can do.

\section{Second Approach---Shared Intentions}
*Start by saying that if we don't have aggregate agents we can't straightforwardly extend theory from individual to joint action.  This opens up the possibility that there are multiple theoretically coherent ways to extend the individual theory each of which is also empirically motivated. [This might go into the previous section]

*The shared intention approach involves extending the model by identifying a state which stands to the joint action in many of the ways that an individual intention stands to an individual action.  Having said what shared intentions are and what it means for agents to act on a shared intention, it is then possible to follow the standard model and say that the goal of a joint action is the goal specified by the shared intention the agents are acting on.

*Goal-states play individuative and coordinative roles; can either retain both or drop the coordinative role. [*This might go into the next section]

*Shared intention: disagreement about what it is but two ways to think about it---fundamentally, it is something that plays a coordinative role; less fundamentally, at least one of the three necessary conditions holds when agents share an intention (awareness of joint-ness etc). 

*Too much already written about this notion for it to benefit from  extended discussion here

*[Possible line] the notion of shared intention, being neither shared nor an intention, is doubly metaphorical.  Diversity in the attempts to explain this metaphor may reflect the fact that the metaphors can be cashed out in different ways.  In any case, it is tricky, in explaining one metaphor (joint action) to appeal to another, potentially more thorny metaphor.





\section{Xth Approach---Instrumentalism}
*This is the application of a model other than the standard model of individual action mentioned at the outset to the joint case; it's application is straightforward.

*Of course if one thought that the Dennettian twist to instrumentalism about intentions and other psychological states (which is not actually instrumentalism) were correct, this approach would collapse into the first approach (Aggregate Agents).  Note that the argument against the claim that this approach could apply to all joint actions apply also in this case; those arguments are, then, arguments against the combination of instrumentalism about joint action and instrumentalism about psychological states.


\section{Fourth Approach---Teleological Goals}
*Goals as properties of the action rather than as properties of an agent.  This is a second way (the first was Aggregate Agents, First Approach) in which the transition from individual to joint action is, conceptually, entirely straightforward.

*Here we are appealing to a model other than the standard model of individual action mentioned at the outset.

*Useful for insects (ants, bees illustrate) partly because doesn't involve commitments to the idea that individual agents have goals.



\section{Conclusion}
The notion of joint action involves extending theories from the individual, single-agent case to cases involving multiple agents.  There is no straightforward way to so extend these theories, any extension will be truer to some features of individual action at the expense of abandoning others.  Further, no single extension distinguishes itself as uniquely correct.

 


\section{Cuts---still useful}



.  This answer cannot straightforwardly be extended to the case of joint action because, in this case, no goal-state is related to the joint action in just the way that . 


Exposition is simplified by assuming that goal-states have a propositional content although this is not strictly required for the truth of claims defended here. 






As background for what follows this section rehearses some elementary points about individual action.  It also explains why there is a question about the relation between joint actions and their goals that can't be answered by a straightforward extension of the answer given to the corresponding question about how individual actions relate to their goals.

%A goal-state specifies a unique goal-outcome.  Exactly how this works depends on general issues about the nature of psychological states.  But it is harmless for present purposes to assume that goal-states have propositional contents.  The specified outcome is then the outcome whose occurrence would consist in the propositional content's being true. (*think!)


Of course the agents of a joint action may themselves be acting on goal-states, but as none of their individual
 but this is not the same thing as there  being 



*Why can't we give the same answer for joint action?  Because not all joint actions are such that all the agents involved in the joint action constitute a single aggregate agent.


As already mentioned, an individual action's occurrence consists in an agent acting on a goal-state.  The goal-states plays (at least) two roles.

One role is individuative.  To say which action an agent performed we identify an event and a description of it (*Davidson).  The content of the goal-state on which the agent acted is this description.  So to know which goal-state an agent is acting on is to know which action she is performing (*ref).

Another role is coordinative.  An action may involve multiple activities whose execution is subject to ordinal and temporal constraints.  The goal-state coordinates these activities (*ref).

In joint action, counterparts of goal-states must play individuative and coordinative roles.  However, while one counterpart plays the coordinative role, another does not.  This, in barest outline, is why there are significant differences among the ways in which the relation between a joint action and its goal-outcome can be construed.





\section{}
We have seen that in individual action goal-states play both individuative and coordinative roles.  In joint action there is no goal-state because there is no agent in the ordinary sense.  

There is only something which is metaphorically described as a goal-state.

Because joint action is not literally a 

\bibliography{$HOME/endnote/phd_biblio}

\end{document}