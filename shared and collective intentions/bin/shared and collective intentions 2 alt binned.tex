%!TEX TS-program = xelatex
%!TEX encoding = UTF-8 Unicode

\documentclass[12pt,a4paper]{extarticle}
% extarticle is like article but can handle 8pt, 9pt, 10pt, 11pt, 12pt, 14pt, 17pt, and 20pt text

\def \ititle {Joint Action:}
\def \isubtitle { Shared Intentions and Collective Goals}
\def \iauthor {Stephen A. Butterfill}
\def \iemail{s.butterfill@warwick.ac.uk}
%\date{}

\input{$HOME/Documents/submissions/preamble_steve_paper}

\begin{document}

\setlength\footnotesep{1em}

\bibliographystyle{newapa} %apalike

\maketitle
%\tableofcontents

\begin{abstract}
[*redo]
A joint action is a goal-directed action, or something resembling one, comprising two or more agents' goal-directed activities.  
What is the relation between a joint action and the goal (or goals) to which it is directed?  
Standard answers to this question take for granted that  this relation invariably involves shared intention.  
Yet research in both psychology and philosophy reveals the existence of goal-directed joint actions without shared intentions.
Making sense of this research requires an account of the relation between joint actions and their goals invoking nothing like shared intention.
This paper provides such an account by introducing the notion of a collective goal.
Collective goals are a necessary but neglected building block for research, both theoretical and empirical, on joint action.



\end{abstract}


\section{Introduction}


  
Philosophers' approaches to joint action stand in stark contrast to psychological research on how joint action is possible.
Philosophers have focused almost exclusively on joint actions which involve shared intention, whereas this notion is largely absent from the psychological research.  

What is shared intention?  
Minimally, shared intentions are supposed to do for multiple agents some of what ordinary intentions do for individuals.  
So, like an ordinary intention, a shared intention's functions include coordinating plans and activities---with the difference that these are the plans and activities of multiple agents performing a joint action \citep{Bratman:1993je}.
And a shared intention involves one or more representations whose contents specify the goal or end to which a joint action is directed.
This means that for agents to share an intention to lift a block, it is not sufficient that each agent individually intends to lift the block.  
For the fact that each agent intends this is not, all by itself, sufficient to coordinate their activities.
So whatever exactly shared intentions are, two or more agents' sharing an intention cannot be a matter of each agent individually having an intention.

Beyond this there is deep disagreement on what shared intention is.
***


Despite the disagreements, it is common to start with the premise that all significant cases of joint action involve shared intention.  
This is taken for granted by Alonso:
%
\begin{quote}
`the key property of joint action lies in its internal component [...] in the participants’ having a “collective” or “shared” intention.' \citep[pp. 444-5]{alonso_shared_2009}
\end{quote}
%
Gilbert is also explicit on this point:  
%
\begin{quote} 
`I take a collective action to involve a collective intention.'  \citep[p.\ 5]{Gilbert:2006wr}\footnote{
For the purposes of this paper we can treat `collective intention' as synonymous with `shared intention'.  
In introducing the term `collective action', Gilbert stipulates that collective action involves more than multiple agents performing concurrent actions while leaving open what more is involved (p.\ 4).   
As this matches the sufficient condition for joint action we gave above, we assume that joint actions of the sort we are concerned with are collective actions.
}
\end{quote}
%
And the same view is taken by some developmental psychologists:
%
\begin{quote}
`The sine qua non of collaborative action is a joint goal [shared intention] and a joint commitment’ 
\citep[p.\ 181]{tomasello:2008origins}\footnote{
The context makes it clear that a `joint goal' involves a shared intention in approximately Bratman's (\citeyear{Bratman:1993je}) sense.
}
\end{quote}
%


To illustrate, Schmidt and colleagues assert in passing that `many joint actions occur spontaneously or automatically without the participants being consciously aware of their coordination with each other' (\citeyear[p. 7]{schmidt_understanding_2010}).
Similarly, Vesper and colleagues, in offering a `minimal architecture' for joint action, are able to take for granted that shared intention is a feature only of sophisticated forms of joint action \citep{vesper_minimal_2010}.
This view is implicit in a range of scientific research about joint action in which shared intention plays no role 
	\citep[as reviewed in][]{%
		Knoblich:2010fk,
		Sebanz:2006yq%
	}.
This research focuses on mechanisms coordinating activities including entrainment, motor emulation and task co-representation.  
While it is sometimes acknowledged that shared intention could also play a role in coordinating some joint actions \citep[e.g.][]{Knoblich:2008hy}, this is never held to be essential.


But what is shared intention?  
In barest outline, shared intentions are supposed to do for multiple agents some of what ordinary intentions do for individuals.  So, like an ordinary intention, a shared intention's function is to coordinate plans and activities---with the difference that these are the plans and activities of multiple agents performing a joint action \citep{Bratman:1993je}.


Notions of shared intention provide a straightforward way to extend the standard theory about the relation between individual actions and their goal-outcomes to the case of joint action.  
Given a notion of shared intention and an account of what it is for agents to act on a shared intention, 
the standard theory is extended by substituting shared for individual intentions.  The shared intention specifies an outcome.  This outcome is a goal to which the joint action is directed in virtue of being the outcome specified by a shared intention on which the agents are acting.

While there is debate on what shared intention is (e.g. 
	\citealp{Kutz:2000si}; 
	\citealp{Tollefsen:2005vh}; 
	\citealp{gilbert_walking_1990};
	\citealp[pp.\ 74--81]{miller_social_2001};
	\citealp{tuomela_collective_2000}), 
there have been few attempts to explain the relation between joint actions and their goals which  involve nothing like shared intention.\footnote{ 
One exception, which differs from the present work in motivation and substance, is \citet{Roth:2004ki}.
}
For our purposes, something is `like shared intention' if it both plays a role in coordinating a joint action and  involves a content which specifies a goal to which the joint action is directed.




On the surface claims that there are joint actions without shared intentions are puzzling.  
The puzzle is not that, as we have seen, some philosophers stipulate that joint action involves shared intention; this might be incorrect or  only terminological.
Rather, the puzzle is this.
It is agreed on all sides that the joint actions of interest are, or resemble, goal-directed actions.\footnote{
Just as the term `action' is sometimes used to encompass events which are not goal-directed \citep[e.g.][]{Hursthouse:1991rd}, 
we allow that there might be forms of joint action which are not goal-directed.
The present point is only that some of the joint actions studied in the psychological research cited above are goal-directed.
}
In particular, it is agreed that for several agents' activities to constitute a joint action there must be a goal to which these activities taken together are directed.
And, trivial cases aside, their activities being directed to this goal cannot consist entirely in each agent's activities being individually directed to the goal.
The puzzle is to understand how two or more agents' activities could be collectively directed to a goal without the agents sharing an intention.
We need to solve this puzzle in order to properly understand the claim that some joint actions do not involve anything like shared intention.  
This puzzle is what motivates investigating relations between joint actions and their goals not involving shared intention.

%Note that the existence of the puzzle is independent of whether  the actions in question are really joint actions.  
%The puzzle requires only that there be goal-directed actions or things resembling them comprising two or more agents' goal-directed activities, and that in some cases such events do not involve shared intention.
%*There is still an issue about whether the puzzle is relevant to joint action.

A surprisingly common response to the puzzle is to insist that it has no bearing on philosophical questions about joint action   on the grounds that the psychologists are not investigating joint action in the philosophers' sense and that their research is not relevant to philosophical issues concerning joint action.  
While it is probably true that the term `joint action' is used in a variety of senses (even between narrowly philosophical discussions), there are several reasons to reject this response.  
One is that, contrary to what is often assumed, philosophers invoke several notions of joint action and have raised a variety of questions about them (contrast
	\citealp{Bratman:2009lv},
	\citealp{Gold:2007zd},
	\citealp{miller_social_2001},
	\citealp{Pacherie:2010fk},
	and \citealp{tuomela_we-intentions_2005}%
%	
).  
There is no reason to assume that all of these notions and questions must be unrelated to  psychological discoveries.
There is also a form of interdependence between the philosophical and scientific research.
The scientific research is often aimed at understanding forms of joint action including the most sophisticated as characterised by philosophers.  
Conversely, the philosophical research is usually aimed at characterising forms of joint action that are in some way important to ordinary humans \citep[e.g.][p.\ 327]{Bratman:1992mi}, which means that accounts of what joint action is need to be anchored in empirical phenomena.
To illustrate, it is relatively uncontroversial that shared intentions play a role in coordinating agents' activities.  
As mentioned above, several other mechanisms are also involved in coordinating agents' activities.  
It is unlikely that shared intentions can be fully understood if  treated in isolation from other factors responsible for coordination (the same is probably true for ordinary, non-shared intentions and other factors in the control of non-joint action).  
Instead we need to understand how the several sources of coordination in joint action combine in making effective joint action possible.
This is not to deny that joint action can usefully studied by philosophers in isolation.  Our claim is only that fully understanding both what it is and how it happens will involve correctly interpreting current and future empirical discoveries.

Difficulties in grasping the relevance of psychological to philosophical research on joint action (and conversely) reflect deficiencies in understanding the phenomena.
Removing these difficulties requires analysing forms of joint action without shared intention.

There is also non-scientific motivation for investigating how joint actions might be related to their goals other than by shared intention.  
Petersson offers an a priori argument for the possibility of joint action without shared intention \citep{petersson_collectivity_2007}, but his argument concerns only forms of activity so primitive that they require agency in at most a highly attenuated sense.
It is perhaps an interesting question in its own right whether we can make sense of the possibility that there are richer forms of joint action without shared intention.
Answering this question may be relevant to understanding the distinctive value of joint actions involving shared intention.
Further, it is plausible that characterising joint action involving shared intention will require appeal to forms of joint action not involving shared intention.
In a groundbreaking study, Searle (\citeyear{Searle:1990em}) notes that all shared intentions must be about an activity which in fact involves multiple agents whereas, on pain of circularity or violation of constraints on well-foundedness, not of these activities can be  identified in ways that presuppose shared intention.
This does not require the existence of joint actions without shared intentions, but it does require ways of describing what are in fact joint actions without referring to shared intention.
Consequently it is possible that analysing joint action without shared intention will support claims concerning what shared intentions are about.

We have seen that there are several reasons why an explanation of how joint actions relate to their goals not involving shared intention is needed.  
That such an explanation is needed does not entail that an explanation involving shared intention is incorrect.  For all that we have said here, different joint actions may be related to their goals in different ways.  
Returning briefly to the case of individual action for comparison, there are accounts of the relation between actions and their goals which do not involve intentions or any other kind of goal-specifying psychological states
(e.g.
	\citealp{Bennett:1976rg};
	\citealp{Butterfill:2001kc};
	\citealp{Schueler:2003fk};
	\citealp{Taylor:1964tr}).
These accounts are not invariably motivated by scepticism about intention or about the standard theory.  Rather, the motivating claim is sometimes that, while many goal-directed actions do involve intentions, there are also goal-directed actions not involving intentions.  If this were correct it would arguably be necessary to develop multiple approaches to characterising the relation between individual actions and their goals, some involving intentions and some not involving intentions.  Independently of whether or not this is correct in the individual case, something similar may be true in the case of joint action.  For this reason we do not claim that accounts of the relation between joint actions and their goals are incorrect.  Our claim is just that such accounts are either incorrect or incomplete.

This paper provides an account of how joint actions relate to their goals without invoking anything like shared intentions.  
Its purpose is to provide a conceptual framework for research on joint actions without shared intentions, just as an account of shared intention provides a framework for investigation of joint action involving shared intention \citep[p.\ 150]{Bratman:2009lv}.  
%Our central innovation is the definition and discussion of \emph{collective goal} in section \vref{section_collective}.  

%Aside from this motivation, exploration of different ways in which joint actions could be related to goals may have an intrinsic interest, for in advance of such exploration it cannot be known whether it is coherent to suppose that joint actions could be goal-directed without involving shared intentions.\footnote{
%*cite people saying all joint action involves shared intention without argument
%}

%The strategy of this paper is informed by the conjecture that joint action does not resemble a natural kind in the sense of being a single category already out there to which claims about its nature are answerable.  Instead philosophers start with a partial, intuitive notion plus some fruitful scientific research (see the reviews cited above and \citealp{schmidt_understanding_2010}).  Our task is to investigate different conceptual accounts of what joint action is and to distinguish those which are merely coherent from those which are also empirically motivated.  Since there is no advance justification for assuming that there is only one theoretically coherent and empirically motivated notion of joint action, we should explore multiple approaches to answering the question about the relation between a joint action and its goal.  

%Too ambitious:
%This paper argues that there are two or more ways of explaining the relation between a joint action and the goal to which it is directed, each useful in its own right.  



\section{Distributive Goals}
\label{section_distributive}

How can we explain the relation between some joint actions and  their goals without invoking shared intentions?  In this section we take a first step; the question is answered in the next section.

In many (perhaps all) cases of joint action there is a single outcome to which each agent's activities are individually directed such that the following is possible: all of the agents' activities succeed.  
Let a \emph{distributive goal} of a joint action, or of a collection of goal-directed activities, be any such outcome.  
To illustrate, suppose that many revellers at a village festival jump into a single boat each with the aim of sinking it, where each reveller's aim would be met if the boat were to sink under their collective weight.  
That the boat sink is a distributive goal of their activities.

The notion of a distributive goal is not by itself enough to explain how joint actions relate to their goals.  
It is only one ingredient in the explanation we offer (see section \vref{section_collective}).  
The rest of this section highlights features of distributive goals which are  straightforward  but have sometimes occasioned misunderstanding.

Where agents act on a shared intention, their activities often or always (depending on what shared intentions are) have a distributive goal.  
But the notion of a distributive goal is much broader; in many cases there are distributive goals but no shared intention.
Winfred works in the library of an embassy currently under attack.  
He is setting fire to papers with the intention of destroying all the secret documents before they can be captured.
With all the smoke and  noise of the attack Winfred has no idea whether he is working alone or whether colleagues are acting on the same goal, nor does he have time to reflect on this.
It happens that several of his colleagues have had the same idea and are destroying documents nearby with the same goal as Winfred.
In this case there is no shared intention because none of the agents are aware of any of the others and none know that they are not acting alone.
In fact Winfred and his colleagues' activities arguably do not constitute a joint action.
But their activities do have a distributive goal.


Note that multiple agents' activities do not have a distributive goal just in virtue of the agents' actions being directed to similar outcomes or to outcomes of the same type.  In an example from Searle (\citeyear[p.\ 92]{Searle:1990em}), rain causes park visitors simultaneously to take cover under a central shelter.  Suppose that each visitor's activity is directed to a similar but distinct outcome, namely her own arrival at the shelter.  This is not sufficient for the visitors' activities to have a distributive goal.

The notion of a distributive goal has applications in observation of action where observers have limited knowledge  of the agents' mental states.
In everyday life many people readily assign goals to the activities of pluralities.  
Tyrone and Anne are looking out of the window in a city to whose customs they are both strangers.  Anne asks what \emph{those people} with the brushes are doing and Tyrone replies by saying that \emph{they} are clearing the street of glass.
Tyrone's reply is naturally interpreted as identifying a distributive goal of their activities rather than (say) the contents of any agents' intentions.
On this interpretation, the plural prediction involves almost no commitments beyond those that would be incurred by talking about each agent individually.

As specified in the above definition, 
	for two or more agents' activities to have a distributive goal 
	the relation between each agent's activities and the goal must be such that it is possible for all agents' activities to succeed relative to this goal.
To illustrate, take Sam and Marnie who are among the revellers at the village festival mentioned above.  There is a boat which Sam and Marnie each individually intend to sink.  
For all each of them knows, it is possible that he or she could sink it alone.
Arguably there are two ways of further specifying this situation.  It may be that Sam's and Marine's intentions are incompatible in the sense that they cannot both be realised; each would have to be the sole sinker of this particular boat to succeed.   Alternatively, it may be that Sam's and Marnie's intentions are compatible in the sense that it is possible for both to be be fulfilled; they could both succeed by sinking the boat together.  
(Note that `together'  does not have direct psychological or deontological significance here; compare `the two wings of a biplane lift it together'.)
Only in the latter case, where they could succeed together, 
 would Sam's and Marnie's boat sinking activities have a distributive goal.


Where an agent or her activities are related to a goal in such a way that the agent could succeed relative to this goal in some situations where other agents were also successful relative to the same goal, let us say that this goal relation is \emph{compatible with others' success}.   
We have just noted that distributive goals by definition require compatibility with others' success.

How is this compatibility achieved?  How could the relation between Sam's activities and the goal be such as to allow both Sam and Marnie to succeed in sinking the boat?
One possibility involves Sam having an intention which is explicitly neutral concerning others' success.  
For example, Sam might intend that he sink the boat either alone or with any other revellers.
A second, more interesting possibility is that Sam's intention is simply an intention to sink the boat.
This intention does not specify who will sink the boat; it is not the intention that he, Sam, sink the boat.
Of course, the fact that this intention is Sam's intention means that Sam will have to be among the sinkers of the boat to succeed.  
So Sam's overall relation to the goal is such that his involvement is required for its fulfilment, even though this is not specified in the content of his intention.
It does not follow that Sam's intention or any other aspect of his relation to the goal requires him to be the sole sinker of the boat in order to succeed.
We can allow that there may be ways of elaborating the story about Sam on which his success requires him to sink the boat alone.  
But there are also ways of elaborating the story such that this is not a requirement---being well into the festive spirit, Sam has given no thought at all to whether others will sink the boat, and there are no relevant conventions or norms.
The existence of these elaborations shows that to be related to a goal in a way that is compatible with others' success does not require having intentions whose contents explicitly involve agents.
In some conditions, acting on an intention whose content does not  involve agency is sufficient for being related to a goal in a way that is compatible with others' success.

On standard accounts of shared intention, compatibility with others' success involves the first of these two possibilities, that is, an intention whose contents  explicitly involves other agents.  
For instance, on Bratman's account of shared intention our sharing an intention to paint a house involves us each intending that \emph{we} paint the house \citep[p.\ 333]{Bratman:1992mi}.
We have just been arguing that compatibility with  others' success does not require intentions with contents that explicitly involve agency.
This matters because we are concerned with  forms of joint action that require little conceptual sophistication.  
Eventually we shall show that inability to represent other agents as agents is no bar to engaging in some forms of joint action (that is, to engaging in some goal-directed actions comprising two or more agents' goal-directed activities).
For now the point is just that distributive goals, despite requiring compatibility with others' success, do not require abilities to represent other agents as agents.

For readers who doubt that compatibility with others' success is really possible unless the contents of agents' intentions explicitly allow for it, we offer two considerations.
The first is an imaginative exercise.  
Luke's goal is to destroy the brick wall of a neighbour's front garden.  
He passes this wall every day.  
In order to escape detection he removes a little of the pointing from the wall every time he passes it.  
Over time his activities destabilise the wall and it collapses.  
As we have described this situation, it is of course possible to elaborate it further, adding features which render Luke's relation to the goal incompatible with others' success.
But no such features need be present and we shall stipulate that they are absent.
Now consider the possibility that after the wall collapses, Luke discovers to his surprise that an acquaintance had the same goal and, with no knowledge of Luke's intention or activities, has been working to the same end as Luke.
Whether this possibility is actual seems to have no bearing on Luke's success relative to his goal; to know that Luke has succeeded we do not have to know that no other agents were involved.
This is so even though Luke's goal was simply to destroy the wall and did not explicitly involve agency.
The gradual nature of these activities makes the possibility vivid, but many less gradual actions such as getting a car back onto the road or catching a particular throw are no different in principle.

The second consideration is that intentions can be neutral on a wide variety of issues concerning how a goal is achieved.  
To return to Luke's destruction of the wall, our description is consistent with neutrality on a wide variety of contributing factors.
Luke's success in destroying the wall is independent of whether design faults also contribute to its fall, and of contributions from bad weather, moss or bacteria.
(Such contributions may lessen his responsibility but, given our description of the situation, they would not prevent his success.)
Luke's success is also clearly independent of whether other agents with different goals are involved in the wall's destruction.  
One person sits on the wall to rest her legs, another displays bravado by kicking it in front of her friends while a third removes some pointing in order to conceal chewing gum.
All three contribute to the wall's destruction, but not in a way that prevents Luke from fulfilling his intention to destroy the wall.
Given that Luke's relation to the goal can be compatible with potential contributions such as these, it seems clear that it can also be compatible with others' success even where what the others intend is the same as what Luke intends.

Some goals do not allow for compatibility with others'  success.  For instance, if Andrew intends to win a certain prize and the competition is structured in such a way that ties are impossible, then compatibility with others' success is impossible.  
But for many goals---eating a certain pizza, lifting a heavy block, and so on---compatibility with respect to others' success is possible.

Where multiple agents' activities have a distributive goal there is a sense in which their activities are directed to a goal.  But this may amount only to each agent's activities being individually directed to that goal.  In that case, the question about how joint actions relate to their goals would not be interestingly different from the parallel question about individual actions.
To make the joint version of the question interestingly different, we need cases where multiple agents' activities being directed to a goal is not, or not only, a matter of each agent's activities individually being directed to that goal.
What, in addition to a distributive goal, could be involved in two or more agents' activities being directed to a goal?



\section{Collective Goals}
\label{section_collective}

Let an outcome, possible or actual, be a \emph{collective goal \label{df_collective_goal}} of a joint action, or of any collection of goal-directed activities, where three conditions are met: 
	(a) this outcome is a distributive goal of the activities; 
	(b) the activities are coordinated; and 
	(c)  coordination of this type would normally  facilitate occurrences of outcomes of this type.  
Examples of activities that typically have collective goals include uprooting a small tree together and tickling a baby together to make it laugh.

The notion of a collective goal assumes that of coordination.  This should be understood in a very broad sense.  If some knights tasked with finding a sacred cup agree that each will search on a different continent, their activities count as coordinated.  Equally, when two agents between them lift a heavy block by means of each agent pulling on either end of a rope connected to the block via a system of pulleys, their pullings count as coordinated just because the rope relates the force each exerts on the block to the force exerted by the other.  In this second case, the agents' activities are coordinated by a mechanism in their environment, the rope, rather than by any psychological mechanism (of course lifting the block may well require other, perhaps psychological, forms of coordination).  
By invoking a broad notion of coordination 
and invoking coordination of activities rather than of agents,
the definition of collective goal avoids direct appeal to psychological states.

In characterising collective goals we have appealed to facts about what would \emph{normally} happen (in the third clause, (c), above).  
The relevant notion of normal paradigmatically features in statements like \emph{Birds can normally fly}.  
This notion is arguably teleological; certainly it is not  straightforwardly statistical or normative.\footnote{
Detailed discussion of the nature of the relevant notion of  \emph{normal} would take us too far from the present topic.
For teleological accounts, see 
	\citet[p.\ 33ff.]{Millikan:1984ib} and 
	\citet[p.\ 48ff.]{Price:2001hs}.
}
Conceptually it would be simpler to characterise collective goals by appeal only to what actually happens---that is, to replace the third clause with the requirement that the coordination actually facilitate the occurrence of the goal-outcome.  
Why is the appeal to what would normally happen necessary? 
Consider a case in which two agents' activities do have a collective goal and coordination of their activities actually facilitates the goal-outcome's occurrence: John and Anika fell a tree using a two-handled saw.  
Now imagine a case which is as similar as possible to this one except that John becomes exhausted and they have to give up half way through.  
In this modified case the coordination of John's and Anika's activities does not facilitate the occurrence of the outcome (the felling of the tree).
This is simply because the outcome does not occur.  
But the differences between the two cases are not the sorts of difference that generally determine facts about which goals an activity is directed to.  
Whether activities succeed or fail does not generally play any role in determining what their goals were.
So an adequate account of collective goals must allow that agents' activities can have collective goals even where they fail.  
This is one reason for appealing to what would normally happen in characterising collective goals.  
A second, more direct but less obvious reason involves external factors which render coordination inefficacious.  For an illustration, suppose that Isabel and Rudi are in the habit of lifting heavy blocks by each pulling on a handle which is linked to the block by an intricate system of ropes and pulleys.  
Normally and on nearly all occasions either could lift any of the blocks alone but, providing their pullings are coordinated, the task is easier when done jointly;
and no matter how uncoordinated they are, the way the ropes are arranged means that it is never normally harder for them to lift a block jointly than alone.  
Normally, then, coordination facilitates the blocks being lifted.  
On one exceptional occasion John, a third person, intervenes.  John dislikes coordination between people and so, seeing the coordination of Isabel's and Rudi's activities, he grabs a rope and attempts to prevent the block being lifted.  Although John fails, he does make the joint lifting harder than it would have been for either Isabel or Rudi to lift the block alone.  
So in this exceptional case the coordination of their activities actually hinders rather than facilitates the blocks being lifted:
had their activities not been coordinated, it would have been easier for them to lift the block.
But this case, where John intervenes, does not differ from the normal cases in ways that are relevant to facts about the goals of the agents' activities.  
For this reason it would be a mistake to ascribe a collective goal to Isabel's and Rudi's activities in the normal cases but not in the case where John intervenes. 
This is why appeal to normal conditions is a necessary in characterising collective goals.



The word `collective' in `collective goal' should not be understood to imply that the agents  involved constitute a collective in any social sense.  Nor does having a collective goal imply that the agents think of themselves as having a collective goal.  The use of `collective' and `distributive' reflects  (but does not exactly match) the use of these terms in literature on plural quantification.\footnote{
See Linnebo (\citeyear{Linnebo:2005ig}) for an introduction to plural quantification.  On some views, the predicate in `The goal of their activities was to lift this block' could be interpreted as either distributive and collective.  On the distributive reading, the truth of the sentence is entailed by the truth of `For each of their activities, the goal of that activity was to lift this block'.  On the collective reading this entailment might not hold.
}
The point is that for multiple agents' goal-directed activities to have a certain collective goal is not equivalent to each of their activities separately having that goal.


Where a joint action or collection of goal-directed activities has a collective goal there is a sense in which, taken together, the activities are directed to their collective goal.  It is not just that each agent individually pursues the collective goal; in addition, there is coordination among their activities which plays a role in bringing about the collective goal.  We can put this in terms of the direction metaphor.  Any structure or mechanism providing this coordination is directing the agents' activities to the collective goal.  The notion of a collective goal therefore already provides one way of making sense of the idea that joint actions are goal-directed actions.

We can now explain how joint actions could be related to their goals without invoking shared intention.  A joint action is related to the goal (or, if there is more than one, to a goal) to which it is directed in virtue of this goal  being the collective goal of the joint action.  The goal of a joint action is its collective goal.

The aim of this section, and the motivation for introducing collective goals, was to understand how joint actions could be related to their goals in the absence of shared intention.  
Our claim is that the notion of a collective goal can be used to identify one way in which joint actions are related to their goals.  
This is compatible with maintaining that there are other ways in which joint actions can be related to their goals and that the relation we have identified does not hold in every case of joint action.  
We return to this issue in section \vref{section_collective_vs_shared}.

To sum up so far, understanding how joint actions relate to their goals requires the notion of a collective goal in addition to that of a shared intention.

The notion of a collective goal involves too little to provide dramatic new insights into the nature of joint action.
Why devote a whole paper to such a simple notion?
One reason is the mystique surrounding joint action, which is sometimes introduced by reference to elusive notions of togetherness or jointness (compare 
	\citealp{gilbert_walking_1990} and
	\citealp[p.\ 150]{Bratman:2009lv}%
	%\citep[p.\ 150]{Bratman:2009lv}: What distinguishes your and my relation to each other, in our modest sociality, from each of our relations to the Stranger?
).
It is important that the puzzle about how joint actions could be related to their goals in the absence of shared intention can be solved with appeal only to minimal, uncontroversial ingredients.
The existence of collective goals shows that we can coherently think about joint action without starting from underspecified notions of jointness and without invoking ideas of togetherness which are somehow distinct from any involved in the three legs of a tripod supporting a camera together.
A second reason for focusing on collective goals 
	is that this notion identifies a common starting point for different psychological approaches to joint action.
There is currently much debate between researchers who focus on behavioural dynamics 
(e.g.\ 
	\citealp{marsh_social_2009} and
	\citealp{schmidt_richardons:_2008}%
)
and those whose approach to explaining how joint action is possible primarily involves motor cognition and representation (e.g.\ 
	\citealp{Sebanz:2005fk} and
	\citealp{Knoblich:2006bn}%
).
As things stand, it is sometimes hard to identify the common ground necessary for the two approaches to be in genuine disagreement.
The account of joint action in terms of collective goals identifies what both approaches can agree on. 
The substantial questions concern which types of coordination exist in joint action, 
	and to what extent characterising any given type of coordination requires appeal to general dynamical principles or to motor cognition and representational control structures.
The notion of a collective goal, then, is offered  not as something which already provides deep insight into what joint action is but as part of a minimal framework for the construction of more substantial accounts.


\section{Collective Goals Are Not Shared Intentions 
	\label{section_collective_vs_shared}
}
Collective goals are conceptually distinct from shared intentions.  Shared intentions are, or resemble, states of agents.  By contrast a collective goal is primarily an attribute of activities, not of agents.  Furthermore, a shared intention plays a role in the coordination of multiple agents' activities.  Appeal to a shared intention is therefore potentially a way of explaining how agents coordinate their activities.  By contrast, collective goals are not the sort of thing that can coordinate anything; their existence presupposes coordination.  Finally, shared intentions are supposed to be shared by agents in something resembling the sense in which conspirators can share a secret and not only in the sense in which two people can share a name.  
For two people to share a name it is sufficient that each is individually  so named.  
Collective goals are shared in this sense, but not in the stronger (although rarely explicated) senses associated with `shared intention'.
It would be a distortion to claim that the notion of a collective goal is the notion of something agents share.

There is a potentially more fundamental contrast between shared intentions and collective goals.  A shared intention integrates two functions: it determines what the action is by specifying its goal and it plays a role in the coordination of agents' activities.  By contrast, no single element provides both functions in the account of the relation between joint actions and their goals which appeals to collective goals.  

We should not infer from this alone that there are, or even that there might be, shared intentions without collective goals.  That collective goals and shared intentions are conceptually distinct does not imply that one can exist without the other.

Even so, it may be true that agents sometimes share intentions without their activities having any corresponding collective goal.  To illustrate, suppose that four merchants get together and agree to fix their prices with the shared intention that they will each become rich enough to retire by the end of the year.  
The year turns out to be good for trade and the merchants'  shared intention is realised.
As it happens these merchants are excellent traders but poor strategists.
Their profits would have been even larger if they had not colluded in fixing prices.
In fact their price-fixing strategy was so flawed that coordination of this type could not normally have a positive effect on profits.
So their attempts at coordination hindered rather than facilitated the realisation of their shared intention.  
Because the existence of a collective goal requires coordination of agents' activities to be of a type instances of which would normally facilitate occurrences of  goals of this type, no collective goal of the merchants' activities as here described corresponds to their shared intention.

The possibility that shared intentions exist without there being corresponding collective goals is not a superficial feature of the way we have defined collective goal.  
Where there are shared intentions, one or more propositional attitudes serve to link the agents and their goals with any  means of coordination.  
For example, the merchants described above believe, falsely, that coordinating their pricing will facilitate fulfilment of their shared intention.  
Apart from their beliefs and other propositional attitudes, nothing appropriately connects their coordination to their activities and their goal.
In characterising collective goals our aim is to better understand how joint actions are related to their goals when such propositional attitudes are absent.  
In the case of collective goals, then, the link between a goal and a means of coordination can only involve facts, not beliefs or expectations, about its efficacy.
This is why, on standard accounts of shared intention, it is possible for agents to act on shared intentions without there being any corresponding collective goal.  

We began this paper with a preliminary identification of a broad class of joint actions.  
Our concern is with joint actions 
	which involve two or more agents' goal-directed activities 
		where the activities taken together are directed to a goal and this is not, or not only, a matter of each agent's activities being individually directed to that goal.
Characterising how joint actions relate to their goals is essential for more fully understanding what joint actions are.
Given the premise that shared intentions exist, we have shown that there are at least two ways in which joint actions relate to their goals.
Sometimes joint actions are related to their goals by virtue of the agents' shared intention, and sometimes joint actions are related to their goals by virtue of the collective goal of the activities comprising them.  
These relations are empirically distinct.  
It is possible for joint actions to involve shared intentions with no corresponding collective goals and it is possible for joint actions to involve collective goals without any shared intentions.
So there are at least two kinds of goal-directed joint action.


\section{An Objection
	\label{section_objection}
}
The requirements for agents' activities to have a collective goal can be met without the agents being aware that they are met 
(these requirements are given \vpageref{df_collective_goal}).  
This is because two agents' activities can be coordinated without either agent intending to coordinate, and even without the agents being aware that their activities are coordinated \citep{Sebanz:2003kf, schmidt_understanding_2010}.
It follows that two agents' activities might have a collective goal 
and that the agents might be engaged in a joint action together
even though each agent's overall knowledge state is consistent with the possibility that she is acting alone.
But (runs the objection) it is surely true that the agents involved in a joint action must each be in a position to know that she or he is not acting individually.
So the account we have offered of how joint actions relate to their goals is mistaken, or so the objection entails.


There are two possible responses to this objection.  One would be to modify the definition of collective goal.
For instance, we could add this requirement: concerning the outcome which is the collective goal, each agent must know both that 
	this outcome is a distributive goal of some (or all) of the activities comprising the joint action 
	and also that 
	some of these activities are her own activities.
This further stipulation would avoid the objection by entailing that where some agents' activities have a collective goal, each agent is in a position to know she is not acting individually.
Note that this modification requires only that the agents of a joint action are able to represent distributive goals and to identify activities as their own.
The modified account retains consistency with the possibility that agents can engage in joint action without being able to represent intentions or other propositional attitudes.

This first response is only acceptable if there are grounds to think that the objection is correct.  
For it would be a mistake to complicate the definition of collective goal  without decisive reasons for doing so.
A second possible response to the objection is to reject its second premise, 
	that is,
	to reject that claim that each agent of a joint action must be in a position to know that she is not acting alone.
	
Several philosophers have  implied that this claim is true.  
Kutz asserts that participants in a joint action have 
`a conception of themselves as contributors to a collective end' (\citeyear[p.\ 10]{Kutz:2000si}).
Similarly, Roth asserts that in joint action `each participant \ldots \ can answer the question of what he is doing or will be doing by saying for example ``We are walking together'' or ``We will/intend to walk together''' 
(\citeyear[p.\ 361]{Roth:2004ki}).
Relatedly, \citet[p. 56]{miller_social_2001} asserts that each agent of a joint action believes that her actions are interdependent with the others'.
If any of these assertions are true then each agent of any joint action is in a position to know she is not acting alone.
As far as we know, no arguments have yet been given for these assertions.\footnote{
For instance, Miller (whose approach perhaps most closely resembles our own, although we disagree on some important points) writes `we need to distinguish between joint action and various kinds of interdependent individual action that are closely related' (\citeyear[p. 56]{miller_social_2001}).
We agree that the distinctions Miller wants to make are important.
But he does not explain (as far as we can tell) why such distinctions are distinctions between joint and individual actions rather than distinctions between two kinds of joint action.
}

The lack of argument suggests that common sense or intuition may be what grounds these assertions.
Is this plausible?
Start by distinguishing the claim at issue from related claims which are consistent with the unmodified account of collective goals.
The unmodified account entails the possibility of joint actions with agents who are not all individually in a position to know that they are not acting alone.
It does not follow (and it is no part of our view) that such cases are either teleologically normal or theoretically significant.
It is possible that coordination mechanisms on which effective joint action hinges are such that it is teleologically normal for them to facilitate joint action only when agents intend or expect to coordinate their actions.
If so, there is a sense in which joint actions without some minimal awareness of jointness are not normal, and they may also lack theoretical significance.
This would be consistent with the unmodified account.
So any grounds which support the objection have to distinguish the claim at issue, which concerns conceptual or logical possibility, from similar claims about what is teleologically normal or theoretically significant.  
Common sense and intuition are not generally reliable when it comes to such fine distinctions, nor when it comes to deciding borderline or limiting cases.
Further, it is plausible that different people have inconsistent intuitions about what counts as joint action.
These considerations indicate that the objection cannot be grounded on intuition or common sense alone.

As things stand, we lack decisive grounds both for accepting the objection and for rejecting it.
For this reason we tentatively endorse the second response and stick with the unmodified definition of collective goals.
Even if this were a mistake, the first response above shows that the objection can be avoided by adding to the definition; this addition would not substantially alter our conclusions.


\section{Conclusion}

The question was how joint actions are related to their goals.
On most contemporary accounts of joint action the answers hinge on shared intention.
Shared intentions do  for joint actions what non-shared intentions do for non-joint actions: they play a role in the coordination of activities and specify the goals to which these activities are directed.
Given the notion of shared intention, whether an action is individual or joint makes almost no difference to an explanation of  the relation between actions and their goals.
This is why the question about how  joint actions relate to their goals has not been considered.  
The question only becomes pressing because, as psychological and philosophical research on joint action reveals, there are joint actions without shared intentions.

The existence of joint actions without shared intentions is at least superficially puzzling.
It is puzzling because, 
given existing accounts of what joint action is, 
it is unclear how a joint action could be directed to a goal except by virtue of something like a shared intention.

Our solution, which may not be the only possible solution, involved two steps.  
The first step concerned distributive goals.  
An outcome is a \emph{distributive goal} of multiple agents' activities just if this outcome is a goal to which each agent's activities are individually directed where it is possible for all agents  (not just any agent, all of them together) to succeed relative to this goal.
%One dark night two communists  each independently intend to paint a large bridge red.   Because the bridge is large and they start from different ends, they have no idea of the other's involvement in their project until they meet in the middle.  Although their intentions were simply to paint the bridge and did not explicitly involve agency at all, they both succeed in painting the bridge. As this illustrates,  
Distributive goals  do not require intentions or thought about others' agency.

The notion of distributive goal is not sufficient to explain how joint actions are related to goals, as we saw.

The second step in solving the puzzle was to introduce the notion of a collective goal.  An outcome is by definition a collective goal of multiple agents' activities when (a) this outcome is a distributive goal of the activities; (b) the activities are coordinated; and (c)  occurrences of an outcome of this type would normally be facilitated by coordination of this type.  Where there is a collective goal there is a sense in which the agents' activities are directed to this goal, and this is not only a matter of each agent's activities being individually so directed.
So the notion of a collective goal provides one way of making sense of the claim that there are joint actions which are goal-directed but do not involve shared intentions.

The point of elucidating the notion of a collective goal is not that this notion by itself identifies a theoretically significant form of joint action.  Its value lies in showing how there could be goal-directed joint actions without shared intention, and in providing the foundation on which more significant forms of jointness in action are built.


How should we think of joint action in the light of this discussion? 
Differences in the ways that actions (joint or individual) relate to their goals can demarcate different kinds of action.
Accordingly, we saw that the notions of collective goal and shared intention are associated with two kinds of joint action.
There is no straightforward hierarchical relation between these kinds of joint action.  
For, as we saw, some joint actions involve collective goals but no shared intentions while others involve shared intentions without corresponding collective goals.  
There are, however, several ways in which these two kinds of joint action may be related.
First, the contents of propositional attitudes which are or comprise shared intentions can refer to joint actions involving collective goals; because collective goals do not constitutively involve intentions, this involves no threat of circularity and raises no issues about well-foundedness.
Second, joint actions involving only collective goals may be proper parts of larger structures which do involve shared intention, much as (on some views) merely purposive activities can be components of actions that are intentional in a stronger sense.
Third, collective goals and the associated form of joint action may be a precursor, in evolution or development (or both), to the potentially more cognitively and conceptually demanding forms of joint action associated with shared intention.

It is no part of our view that there are just two forms of joint action.  
One reason for suspecting there may be greater heterogeneity in the notion of joint action turns on background facts about shared intention.
There are different philosophical accounts of what shared intention is.  
At least some of these are inconsistent if interpreted as competing accounts of a single phenomenon.
Despite this, each account appears to be conceptually coherent and there  do not appear to be arguments which decisively favour one account over its apparent competitors.
This diversity may be partly due to the fact that the term `shared intention' is metaphorical, for (given some plausible premises) shared intentions are neither literally shared nor literally intentions.  
It is possible, then, that not all of the different ways of developing the metaphor are in fact competing accounts of a single phenomenon.
Instead there may be multiple targets of analysis, and these might be linked to further distinctions among kinds of joint action.

Forms of joint action which are goal-directed but involve nothing like shared intention have been neglected by philosophers, perhaps 
% partly because on some views it is tempting to assume that this combination of features is impossible, and mainly
partly because they are, or are thought to be, too simple to present conceptual puzzles.
This is a mistake.  
To understand the cognitive bases of abilities to engage in joint action or their evolution or development, the fact that a conception of joint action presents conceptual puzzles is no virtue.
It may be better to start with the simplest possible notions, such as those of distributive and collective goal, and add the minimum required to further demarcate the category of interest in any given inquiry.
Fully understanding joint action may require shared intention and other philosophically problematic notions; but
understanding goal-directed joint action at its most basic requires nothing like shared intention, and no more than the notion of a collective goal.

\bibliography{$HOME/endnote/phd_biblio}

\end{document}