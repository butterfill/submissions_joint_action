%!TEX TS-program = xelatex
%!TEX encoding = UTF-8 Unicode

\documentclass[12pt,a4paper]{extarticle}
% extarticle is like article but can handle 8pt, 9pt, 10pt, 11pt, 12pt, 14pt, 17pt, and 20pt text

\def \ititle {Observation of Joint Action:}
\def \isubtitle {Shared Intentions and Collective Goals}
\def \iauthor {Stephen A. Butterfill}
\def \iemail{s.butterfill@warwick.ac.uk}
%\date{}

\input{$HOME/Documents/submissions/preamble_steve_paper}

\begin{document}

\setlength\footnotesep{1em}

\bibliographystyle{newapa} %apalike

\maketitle
%\tableofcontents

\begin{abstract}
There are two approaches to defining joint action.
On one, joint action constitutively involves shared intention; the main challenge for this approach is then to say what shared intention is.
On the second approach, a joint action is an action with two or more agents.  
The main challenge for this approach is that, given standard views about action, what are usually offered as paradigm cases of joint action turn out not to be joint actions at all.
This paper investigates problems and prospects for the second approach.

%[***] An alternative approach to joint action starts from the claim, motivated by semantic considerations, that a joint action is an action with two or more agents (Ludwig 2007).  An immediate objection to this claim is that, given any of several widely held views about action, events standardly offered as paradigm joint actions turn out not to be joint actions at all.  This objection can be overcome by refining the claim that a joint action is an action with two or more agents.  This alternative approach is necessary for characterising the joint actions that could explain how sophisticated* forms of cognition emerge in development or evolution.


\end{abstract}




\section{The Question}
To sing a song, to saw through a log and to ride a bike---these and many other things can be done individually by a single agent or jointly by several agents, or so it seems. 
In everyday life we regularly think and talk about actions involving two or more agents.
Asked about the concert, you might report that after some solo numbers, Emily %Haines%
sang a duet with Lyra. %Lyra Brown.%
Catching site of wedding celebrations outside a church, we might notice that the bride and groom are sawing through a log with a two-handled saw.
%I suggest that she is not pulling her weight but you have seen that her dress is coming under pressure from her efforts.
This duet and this sawing are distinct from otherwise similar cases in which the same agents are each acting alone---cases in which, say, the singers are competing for attention or the bride and groom are each individually sawing through two logs.
Everyday thinking about action appears to respect this distinction. 
So we can think and talk about actions involving two or more agents as well as actions involving just one.
This paper partially answers two questions.
First, in what ways (if any) can an action involve more than one agent as agents of the action?
Second, for those who already have number concepts, does an ability to think and talk about actions involving more than one agent necessarily require conceptual sophistication not already required for thinking and talking about actions involving just one agent?


To illustrate the second question, consider actions with a feline agent or an artificial agent.
%It is possible that thinking about such actions does not necessarily require conceptual sophistication not already required for thinking about the actions of human adults.
Perhaps there is a concept of action with this feature: anyone who knows what an action is and what a cat is already knows what a feline action is.  
Nothing would be lost (and things might be gained) by having a concept of action which applies irrespective of agents' species and aeteology.
Our second question is whether the same is true concerning the number of agents.
Is there a concept of action such that anyone who knows what an action is and can count already understands what it would be for there to be an action with 

Is it possible that thinking about actions involving two or more agents demands nothing in the way of conceptual sophistication not already demanded for thinking about the actions involving just one agent?





Most research on joint action concerns shared intentions. ...




There is a phenomenon, call it \emph{joint action}, paradigm cases of which are held to involve two people 
	painting a house together \citep{Bratman:1992mi}, 
	lifting a heavy sofa together \citep{Velleman:1997oo}, 
	preparing a hollandaise sauce together \citep{Searle:1990em}, 
	going to Chicago together \citep{Kutz:2000si}, 
	and walking together \citep{gilbert_walking_1990}.
The same phenomenon is held to be paradigmatically exemplified by cases from developmental psychology such as two people 
	tidying up the toys together \citep{Behne:2005qh},
	cooperatively pulling handles in sequence to make a dog-puppet sing \citep{Brownell:2006gu},
	bouncing a ball on a large trampoline together \citep{Tomasello:2007gl},
	and pretending to row a boat together.
Other paradigm cases of what is thought to be the same phenomenon  from research in cognitive psychology include two people
	lifting a two-handled basket  \citep{Knoblich:2008hy},
	putting a stick through a ring \citep{ramenzoni_joint_2011},
	and swinging their legs in phase \citep[p. 284]{schmidt_richardons:_2008}.

These examples are not supposed to be merely cases of joint action, whatever that is.
They are also supposed to be paradigm cases, 
and they are supposed to be cases reflection on which could provide an intuitive understanding of what joint action is.
Given the premise that there is some notion (at least one but not necessarily only one) an intuitive understanding of which could be gained by reflection on these supposedly paradigm cases, what is this notion---or, if there is more than one, what are those notions?
Our aim is to partially answer this question.

Here is a scenario we shall use throughout.
Nora and Olive killed Fred.  
Each fired a shot.
Neither shot was individually fatal but together they were deadly.
An ambulance arrived on the scene almost at once but Fred didn't make it to the hospital.
What would it take for Nora and Olive's killing of Fred to be a joint action?

The question can be further constrained by noting that joint action phenomena are the focus of a tangle of scientific and philosophical questions.  Psychologically we want to know which mechanisms make it possible to engage in and understand different sorts of joint action
\citep{vesper_minimal_2010}.  
Developmentally we want to know when joint action emerges, what it presupposes and whether abilities to engage in it somehow facilitate socio-cognitive, pragmatic or symbolic development \citep{Moll:2007gu,Hughes:2004zj,Brownell:2006gu}.  
Conceptually, we want a principled way of distinguishing joint from individual actions which supports investigation of mechanisms and development \citep{Bratman:2009lv}, plus a formal account of how practical reasoning for joint action differs (if at all) from individual practical reasoning \citep{Sugden:2000mw,Gold:2007zd}.  
Phenomenologically we want to characterise what (if anything) is special about experiences of action and agency when the actions are joint actions \citep{Pacherie:2010fk}.  
Metaphysically we want to know what kinds of entities and structures are implied by the recognition that some actions are joint actions \citep{Gilbert:1992rs,Searle:1994lb}.  
And normatively we want to know what kinds of commitments (if any) are imposed on participants in joint actions or how these commitments arise \citep{Roth:2004ki}.

These questions provide a motive for asking our question about the nature of joint action.
Perhaps better understanding the nature of joint action will facilitate  progress with the tangle of philosophical and scientific questions about it.
Accordingly we impose a further constraint on acceptable answers to our question about the nature of joint action.
Any answer has not only to conform to an intuitive notion captured by all or most of the examples given earlier; 
it must also be at least potentially relevant to the tangle of scientific and philosophical questions commonly taken to be questions about joint action, whatever that is.
As we shall see (in section \vref{section_first_objection}), 
this constraint distinguishes our project from narrowly semantic and conceptual projects (which is not to say that there is anything wrong with those projects, only that they are not subject to the same constraints).

Our question is not new.
According to Gilbert:
%
\begin{quote}
`The key question in the philosophy of collective action is simply ... under what conditions are two or more people doing something together?' \citep[p.\ 67]{Gilbert:2010fk}
\end{quote}
%
While acknowledging Gilbert's sustained attempt to elaborate and answer this question, we shall distance ourselves from two potentially puzzling features of this passage, features which seem to reflect an attitude widespread among researchers in this area.
First, Gilbert takes the question to be a question in `the philosophy of collective action' (since differences in terminology do not appear to mark explicit theoretical differences, we shall assume that `collective action' is another term for joint action).
In ordinary thought and talk about action, there does not seem to be anything special about actions with not just one agent but two or more.
For instance, in many contexts answering the question `What are they doing?' does not obviously raise difficulties that the question `What is she doing?' does not.
Given recent developments in the logic of plural quantification \citep[see][for an overview]{Linnebo:2005ig}, we should not assume without argument that actions with multiple agents (if there are any) are philosophically or theoretically problematic in a way that actions with tall agents (say) are not.
Nor should we accept without argument that there is any such thing as the `philosophy of collective action'.
To illustrate, it is rightly controversial whether there are philosophically significant differences between actions with animal agents and actions with non-animal agents.
It should be similarly controversial, at least at the outset, whether there are philosophically significant differences between actions with single agents and actions with multiple agents.
It is true, of course, that philosophical claims about action tend to apply directly only to actions with one agent and cannot straightforwardly be generalised to actions with two or more agents.  
But since this historical fact may reflect a blind-spot rather than an insight, it alone should not persuade us that collective action is among the categories deserving a philosophy of their own.
So while our question is Gilbert's, we are neutral on whether, as she seems to suppose, the relevant philosophical category is joint action as opposed to simply action.

The second potentially puzzling feature in Gilbert's formulation of the question is reliance on togetherness.
The term `together' can be used in contexts which do not involve actions or agents.
The three legs of a tripod support it together,
and the many bee stings together cause an allergic reaction.
One crude conjecture about its meaning is that `together' functions to force collective rather than distributive predication \citep[on collective and distributive, see][p.\ 322]{oliver_modest_2006}.
If we think of `together' in this crude way, 
or even if we adopt a more sophisticated theory of what `together' means  \citep[e.g.][]{moltmann_semantics_2004},
there are many cases in which people do things together that clearly do not fit any intuitive notion linked to the paradigm cases above.
For instance, suppose that before work one morning three strangers each discover a corner of an extremely large but very light polystyrene triangle in their gardens and that, entirely by chance and without any knowledge of the others' efforts, each lifts her corner in order to inspect it.
It seems plausible that the strangers lift the triangle together.
But it seems unlikely that this case falls under any notion of joint action linked to the paradigm cases mentioned above;
certainly this is one type of case that philosophers typically contrast with joint action \citep[e.g.][p.\ 149]{Bratman:2009lv}.
We should be cautious, then, of supposing that our question can be put exactly as Gilbert does, by appealing to a notion of acting together.
More generally, there do not seem to be any simple semantic markers of joint action---at least not if, as we are assuming, joint action is the notion or notions an implicit understanding of which is available through reflection on the paradigm cases mentioned above.

To sum up, we assume that there is a notion (at least one, but not necessarily only one) which is both 
	(a) central to a tangle of philosophical and scientific questions  commonly taken to be questions about joint action
	and also 
	(b) such that an implicit conception of it is available through reflection on a collection of paradigm cases including all or many of those listed in our opening paragraph.
By stipulation, and in line with much of the literature, we call any such notion a notion of \emph{joint action}.
Our question is, What is that notion or, in case there is more than one, what are those notions?  
Aside from these questions, (a), and paradigm cases, (b), we have not identified any constraints on what the notion or notions could be.



\section{Standard Answers}
The usual approach to answering this question is to appeal to some notion of shared intention (we shall use `shard intention' broadly to encompass notions of collective intention and what are sometimes called we-intentions).
For instance:  
%
\begin{quote} 
`I take a collective action to involve a collective intention.'  \citep[p.\ 5]{Gilbert:2006wr}
\end{quote}
%
\begin{quote} 
`The sine qua non of collaborative action is a joint goal [shared intention].’ 
(Tomasello 2008, p. 181)
***Tomasello says collaborative!
\end{quote} 
%
Many further examples could be given  (%
	\citealp[p.\ 381]{Carpenter:2009wq}; 
	\citealp[p.\ 369]{Call:2009fk};
	\citealp{Kutz:2000si}; 
	\citealp[p.\ 117]{rakoczy_pretend_2006}; 
	\citealp{Tollefsen:2005vh}%
	).

On this approach the problem of characterising joint action becomes the problem of characterising shared intention.
Why is it necessary to characterise shared intention?
Intentions can literally be shared in the sense in which siblings can share genes, but this sense of sharing is almost universally regarded as too weak.
And intentions are not literally shared in the sense in which siblings can share a parent; or if some intentions are shared in this sense (as some, including \citealp{Velleman:1997oo}, suggest), this form of sharing is not general enough to characterise the full range of paradigm examples that anchor notions of joint action.
This is perhaps why, on almost any account, to share an intention does not require there to be a single mental state with two or more subjects.

The term `shared intention' is doubly metaphorical.
For on almost any account, shared intentions are neither shared nor intentions.
This may be why there is {much divergence on what shared intentions are}. 
Some hold that shared intentions differ from individual intentions with respect to the attitude involved (\citealp{Searle:1990em}). 
Others have explored the notion that shared intentions differ with respect to their subjects, which are plural \citep{Gilbert:1992rs,helm_plural_2008}, 
or that they differ from individual intentions in the way they arise, namely through team reasoning \citep{Gold:2007zd}, 
or that shared intentions involve distinctive obligations or commitments to others (\citealp{Gilbert:1992rs}; \citealp{Roth:2004ki}).
Opposing all such views, \citet{Bratman:1992mi,Bratman:2009lv} argues that shared intentions can be realised by multiple ordinary individual intentions and other attitudes whose contents interlock in a distinctive way. 

Each notion of shared intention could give rise to a different characterisation of joint action.
Published research to date provides no decisive argument for supposing that there is only one notion of joint action, nor that there is only one notion of shared intention.
Perhaps one or more of these notions of shared intention gives rise to a partial answer to our question about joint action.


Opposing the broad consensus, several researchers have proposed that there are notions of joint action on which shared intention is not necessary for joint action.\footnote{
See \citet[p.\ 330]{Bratman:1992mi} on `cooperatively neutral joint-act-types': `There is ... a clear sense in which we can ... paint the house together without our activity being cooperative.'  Bratman does not discuss such activities in any detail, and his position requires only the weaker assumption that some cases of joint activity can be `understood in a way that is neutral with respect to shared intentionality' (\citeyear[p.\ 147]{Bratman:1999fr}).  \citet{chant_unintentional_2007}, \citet{ludwig_collective_2007}, and \citet[p. 7]{schmidt_understanding_2010} also hold views on which not all joint action involves shared intention.
%\citet[p. 7]{schmidt_understanding_2010}: `many joint actions occur spontaneously or automatically without the participants being consciously aware of their coordination with each other.'
Finally, \citet[p. 448 fn. 17]{alonso_shared_2009} agrees with Bratman in stating that joint action does not require shared intention but, puzzlingly, also claims that `what distinguishes joint action from other kinds of aggregated phenomena ... lies in the participants' having a ... ``shared'' intention' (pp. 444-5).
} 
Understanding what this proposal amounts to requires specifying more about shared intention than we have done so far.
Given divergent views on what shared intention is, what could it mean to  say that shared intention is not involved in any given case?  
On all or most leading accounts of shared intention, at least one of the following is a necessary condition:

\begin{description}
\label{shared_intention_conditions}

\item[awareness of joint-ness] Agents acting on a shared intention know that they are not acting individually; they have `a conception of themselves as contributors to a collective end.'\footnote{
	\citet[p.\ 10]{Kutz:2000si}.  Compare \citet[p.\ 361]{Roth:2004ki}: `each participant ... can answer the question of what he is doing or will be doing by saying for example ``We are walking together'' or ``We will/intend to walk together.''' 
Relatedly, \citet[p. 56]{miller_social_2001} requires that each agent believes her actions are interdependent with the other agent's.
}

\item[awareness of others' agency]  When agents act on a shared intention, each is aware of at least one of the others as an intentional agent.\footnote{
	Compare \citet[p.\ 333]{Bratman:1992mi}: `Cooperation ... is cooperation between intentional agents each of whom sees and treats the other as such'.  See also \citet[p.\ 105]{Searle:1990em}: `The biologically primitive sense of the other person as a candidate for shared intentionality is a necessary condition of all collective behavior' 
}

\item[awareness of others' states or commitments] When two agents share an intention that they F, each is aware of, or has individuating beliefs about, some of the other's intentions, beliefs or commitments concerning F.\footnote{
This condition is necessary for shared intention even on what \citet[p.\ 40]{tuomela_collective_2000} calls `the weakest kind of collective intention'.  But it may not be necessary if, as \citet{Gold:2007zd} suggest, shared intentions are constitutively intentions formed by a certain kind of reasoning.
% "if the distinctive feature of collective intentions is to be found in the reasoning by which they were formed, then an analysis that focuses on the intentions themselves will miss the feature that makes collective intentions collective. " 
}

\end{description}
The proposal under discussion is that there is a notion of joint action on which shared intention is not necessary for joint action.
To establish this proposal it is sufficient (but perhaps not necessary) to show that none of the three conditions above is met in every case of joint action.

As far as we know, no decisive argument has been given for this proposal.\footnote{
\citet{petersson_collectivity_2007} offers an extended argument against the claim that all joint actions (`collective activities') involve shared (`collective') intentions.  But his argument establishes only that joint action very broadly understood need not involve intentions: the notion of non-intentional joint action Petersson characterises  applies even to the behaviours of non-animal and inanimate  substances (see, e.g., p. 149).  The present paper is primarily concerned with more elaborate cases of joint action without shared intention.
} 
But nor has anyone published a decisive objection to it.
In our view, a good argument would be a constructive one.
Such an argument would take the form of the construction of a notion that is recognisable as joint action and which clearly allows for joint action without shared intention.
Our aim in what follows is to provide such a construction, or at least to explore some of the pitfalls in providing one.

\section{Motivation
	\label{section_motivation}
}
We have just explained that, by contrast with the usual approach, our aim in what follows is to characterise a notion (at least one) of joint action which does not necessarily involve shared intention.
Aside from any intrinsic interest in questions about what joint actions are,
why attempt this?

One possible reason is to better understand shared intention.
On some accounts, shared intentions involve individual intentions about a joint action \citep[e.g.][]{Bratman:1993je}.  Since the contents of these individual intentions cannot without circularity all concern shared intentional activities as such \citep[p. 95]{Searle:1990em}, characterising shared intention would, on these accounts, require ways of construing joint action without appeal to shared intention.  Understanding shared intention would therefore require understanding which construals of joint action individual intentions can be about (\citealp{petersson_collectivity_2007}; \citealp[p. 163]{Bratman:2009lv}).  The investigation of cases of joint action without shared intention is a step towards meeting this requirement.

Another reason for wanting to characterise notions of joint action which do not involve shared intention arises from some  scientific research on joint action.
Philosophers' approaches to joint action stand in stark contrast to psychological research on how joint action is possible.
To illustrate, Schmidt and colleagues assert in passing that `many joint actions occur spontaneously or automatically without the participants being consciously aware of their coordination with each other' (\citeyear[p. 7]{schmidt_understanding_2010}).
Similarly, Vesper and colleagues, in offering a `minimal architecture' for joint action, are explicitly open to the possibility that shared intention is a feature only of sophisticated forms of joint action \citep{vesper_minimal_2010}.
This view is implicit in a range of scientific research about joint action in which shared intention plays no direct role 
	\citep[as reviewed in][]{%
		Knoblich:2010fk,
		Sebanz:2006yq%
	}.
This research focuses on mechanisms coordinating action including entrainment, motor emulation and task co-representation.  
Since these mechanisms are not directly related to shared intentions \citep{Knoblich:2008hy, vesper_minimal_2010},
it is plausible that the phenomena studied do not fall under notions of joint action which require shared intention.
If they are joint actions at all, they fall under some other notion of joint action.
It might be useful to know whether there is any such notion and, if so, what it is.


\section{The Simple Definition}
To recap, we assume that a notion is a notion of joint action if it is both central to the tangle of philosophical and scientific questions commonly taken to be questions about joint action and also such that an implicit conception of it is available through reflection on a collection of paradigm cases including many or all of those listed in our opening paragraph above. 
Our aim is to identify a notion of joint action which does not involve shared intention.
More precisely, we aim to identify a notion of joint action such that none of the three conditions given \vpageref{shared_intention_conditions} are met in every case of joint action. 


Our starting point we call  \emph{the simple definition}.
On this definition, a \emph{joint action} is an action with two or more agents, as contrasted with an \emph{individual action} which is an action with a single agent \citep[p.\ 366]{ludwig_collective_2007}.
We do not claim that this definition is adequate to our aim.
Instead we shall offer a series of objections and refinements.

Does the simple definition allow for the possibility of joint action without shared intention?
For all we have said so far, it may turn out that the existence of an action with two or more agents requires the agents to have a shared intention.
We shall not explore this issue because there is a more straightforward objection to the simple definition.


\section{First Objection
	\label{section_first_objection}
}

The first objection to the simple definition is that, given standard views of action, what are taken to be paradigm cases of joint action would not be joint actions at all.
In this section we explain the objection.

According to Davidson, whose views have shaped discussion,
%
\begin{quote}
`our primitive actions, the ones we do not by doing something else, ... these are all the actions there are.'
\citep[p.\ 59]{Davidson:1971fz}.
\end{quote}
%
On Davidson's definition, a \emph{primitive action} is one that `cannot be analysed in terms of [its] causal relations to acts of the same agent' \citep[p.\ 49]{Davidson:1971fz}.
Unless the existence of joint action can be ruled out in advance, this definition is insufficiently general because it applies only to actions with one agent.
We can avoid or at least postpone the tricky question of how this definition should be generalised by noting that Davidson also holds that the only actions which are primitive are `mere movements of the body' \citep[p.\ 59]{Davidson:1971fz}.
So on his view movements of the body are all the actions there are.
To illustrate, suppose that Ahmed unlocks a door by turning a key, which he in turn achieves by moving his fingers.
On Davidson's view, Ahmed's door unlocking action is his finger moving action.
To those unfamiliar with this idea it may seem baffling but it is in some ways an intuitive idea.
Ahmed's action, the difference that his agency makes in the world,  is not constituted by the lock's movement but only by the movements of his own body
(or, on developments of Davidson's view, by Ahmed's trying to move his body---see below).
Whether the lock unlocks or not depends on things which Ahmed cannot directly influence; it is at most the movements of his body which are under his control when he acts.
Since his action causes the door to unlock, we can \emph{describe} Ahmed's action by saying that he unlocks the door.  
But this doesn't mean that changes in the lock are any part of his action, only that they are among its effects.

This  view, together with some plausible assumptions, implies that in the scenario described above, Nora and Olive's killing of Fred, no action has two agents and so there is no joint action in that sense.
Briefly, this is because there is no bodily movement with more than one agent.
Nora and Olive each move their separate fingers on the two triggers.
(Perhaps we could imagine a case where Nora and Olive were both involved in each firing, but that is not the case described above.)
Furthermore, no other bodily movement involving these 
agents is among the causes of Fred's death.
Suppose that all actions are primitive and that all primitive actions are  bodily movements, as Davidson claims.
Then among the actions which caused Fred's death there is none of which Nora and Olive are both agents.
Given the simple definition, it follows that Nora and Olive's killing of Fred was not a joint action.

Is this too quick?  
Suppose that the composite of Nora's and  Olive's bodily movements is an event.
This event clearly involves two agents.
If this event were an action, there would be an action of which Nora and Olive are agents.
And this event is clearly distinct both from Nora's firing and from Olive's firing.
But could there really be a third action, one distinct from both Nora's firing and Olive's firing?
Consider a parallel scenario, one that is as similar as possible to the killing of Fred except that a single agent, Coralie, fires both shots (she has a gun in each hand).
Let us stipulate that each of Coralie's two firings is an action; this is surely possible even if there are other possible cases in which Coralie's two firings would jointly comprise a single action without either individually being an action.
Unless there are three actions in this scenario, Coralie's two firings plus their composite, we should not allow that there are three actions in the original scenario either.
Now on  some views of action it is possible that there are indeed three (or more) actions which involve  Coralie shooting.
But, given Davidson's claims, the composite of Coralie's two firings is not an action.
This is because the event is not primitive: it consists of Coralie's two firings and nothing else and so, trivially, it can be analysed in terms of these actions.
We should therefore draw the same conclusion about the composite of Nora's and Olive's firings: it is not an action.
So on Davidson's view, the events (if any) which involve both Nora and Olive as agents are not actions.

To avoid possible confusion about the nature of this claim, consider  what `Nora and Olive's killing of Fred' might refer to.
This, like other phrases of this form, is arguably ambiguous \citep[p.\ 84]{pietroski_actions_1998}. 
It might refer to an event which includes, among other things, Nora's and Olive's firings, the subsequent movements of the bullets, and Fred's death.
Call this event \textsc{the episode}.
Alternatively, `Nora and Olive's killing of Fred' might refer to an event comprising the actions which caused Fred's death and nothing else.
Call this event \textsc{the actions}.
(We could describe this event as comprising Nora's firing of a shot and Olive's firing of a shot but for the fact that `Nora's firing of a shot' exhibits the very form of ambiguity we are discussing.)
These are different events if, as seems plausible to many, \textsc{the actions} is over some time before Fred's death whereas \textsc{the episode} is not.
Now the event of which Nora and Olive are apparently both agents, \textsc{the episode}, is not an action;
and the event which is entirely constituted by actions, \textsc{the actions}, does not include any action with more than one agent.
This is why, given Davidson's claims, Nora and Olive's killing of Fred is not a joint action where `joint action' means `action with two or more agents'.  

Note that  this argument works irrespectively of whether Nora and Olive have a shared intention.
We could have stipulated that they engage in team reasoning, that they jointly decided to kill Fred, that each intends to do her part; we could even have stipulated that they have common knowledge of each other's intentions that they kill Fred in accordance with, and because of, meshing subplans of their  intentions that they kill Fred.
None of this would have made  any difference as far as the above argument is concerned.

The argument, appropriately modified, applies to cases that are widely taken to be paradigmatic joint actions.
In making the sauce, you stir while I pour;
in painting the house, you cover the outside while I do the inside; 
and in walking together you move your legs while I move mine.
In each of these supposedly paradigm cases of joint action, no bodily movements have two or more agents---or, if they do, they are not actions.
\begin{comment}
\footnote{
The qualification `irreducibly' may be necessary if composite bodily movements may themselves be bodily movements.
What does `irreducible' mean?
We stipulate that an individual is \emph{irreducibly} an agent of a bodily movement if for every part of that bodily movement which is itself a bodily movement, the individual is an agent of the part.
%that bodily movement has no proper parts which are bodily movements and of which the individual is not an agent.
A bodily movement \emph{irreducibly} has exactly $n$ agents if exactly $n$ individuals are irreducibly agents of that bodily movement.
%(except possibly those exhaustively comprised by several bodily movements each of which has exactly one agent)
}
\end{comment}
Given the premises above, it follows that there are no actions with more than one agent in any of these cases.
And on the definition of joint action under consideration this means that these supposedly paradigm cases of joint action are not in fact joint actions.

In outline, this is the objection to the simple definition of joint action:
%
\begin{quote}
1. A joint action is an action with two or more agents.
\end{quote}
%
%
\begin{quote}
\label{objection_1_premise_2}
2. Bodily movements `are all the actions there are.'
\citep[p.\ 59]{Davidson:1971fz}.
\end{quote}
%
%
\begin{quote}
3. In what are taken to be paradigm cases of joint action, no  bodily movements with more than one agent are actions.
\end{quote}
%
Therefore:
%
\begin{quote}
4. What are taken to be paradigm cases of joint action are not actually joint actions.
\end{quote}
%
Why is this argument, even assuming that it is sound, an objection to the simple definition?
Why not just bite the bullet and accept that supposedly paradigm cases are not actually joint actions?
This might be worth considering if our project were to provide a semantic theory.
But our project is to identify a notion of joint action that supports philosophical and scientific inquiry.
And a notion on which few or no supposedly paradigm cases turn out actually to be joint actions is unlikely to serve that purpose.
After all, the paradigm cases are among the things which anchor the phenomenon to be defined.
The argument, if sound, shows that the simple definition is too narrow.

The second premise of the argument can be weakened.
Some philosophers broadly in agreement with Davidson hold that primitive actions are tryings rather than bodily movements \citep[e.g.][]{hornsby_actions_1980}.  
If the above argument works given Davidson's position it also works given this position.
In fact, the argument works given any position which respects two constraints: first, no action involves the movements of things other than the agent's or agents' bodies; and, second, an event comprising two actions with distinct agents and nothing else is not an action.
Whether Nora's firing was a bodily movement, a trying or anything else which stops short of including a bullet's movement, the action was hers alone.

Note that the above argument does not depend on the assumption that all bodily movements (or tryings) have at most one agent.
It may be that some joint actions are literally actions with two or more agents.\footnote{
Several arguments point in this direction.
Roth (\citeyear{Roth:2004ki}) argues that one agent can literally act on another's intention,
Sebanz et al (\citeyear{Sebanz:2005fk}) argue that an individual agent's motor system routinely plans not only that agent's actions but also the actions of another,
and 
Ramenzoni et al (\citeyear{ramenzoni_joint_2011}) argue that two people's bodies can be coupled in ways that resemble  couplings within a body. 
}
Our point is that \emph{Nora's firing}, this particular event, has only one agent  (and Olive's likewise); and that the same is true for many cases which are widely taken to be paradigms of joint action.

This, then, is the first object for the simple definition.
Given standard views about action, the simple definition implies that many supposedly paradigm cases of joint action do not involve joint action at all.



\section{Avoiding the First Objection}
One response to the first objection would be to provide grounds for rejecting the standard views about action which cause the problems (i.e. rejecting premise 2 of the argument \vpageref{objection_1_premise_2}).
%*Tuebingen
We shall not consider this response here \citep[but see][]{chant_unintentional_2007}.
%We shall consider this response presently.
Instead we shall pursue a response which does not require rejecting standard views about action.
This response draws on the idea that there is an attenuated sense of agency in which individuals may be agents of events other than actions.

Pietroski proposes a sense in which individuals can be agents of  events which are not necessarily  actions.  
His proposal has two parts.  
First, there is a relation among events, \emph{grounding}.  
He stipulates that 
%
\begin{quote}
\textbf{singular grounding} 
`event $D$ \emph{grounds} $E$, if: $D$and $E$ occur; 
$D$ is a (perhaps improper) part of $E$; and 
$D$ causes every event that is a proper part of $E$ but is not a part of $D$.'
\citep[p.\ 81]{pietroski_actions_1998}
\end{quote}
%
Pietroski's intention is that the toppling of a line of ten dominoes should be grounded by the toppling of the first domino \citep[p.\ 81]{pietroski_actions_1998}.
The definition of grounding may need modification if this intention is to be fulfilled.
For suppose that the toppling of the first two dominoes is an event, call it $F$.
Then, since $F$ is a proper part of the toppling of the whole line and not a part of the toppling of the first domino,
the above definition entails that
the toppling of the first domino can only ground the toppling of the whole line if the toppling of the first domino causes $F$.
But since the toppling of the first domino is a part of $F$, on many standard accounts of causation the former will not cause the latter.
To see why, consider two principles.  
First, no event causes itself.  
Second, where one event causes another, the first event also causes any event which is part of the second (this principle may be restricted in ways which do not affect its application here).  
These principles jointly imply that the toppling of the first domino does not cause $F$.
Given that $F$ exists and these principles are correct, the above definition entails that the toppling of the first domino does not ground the toppling of whole the domino line, contrary to what was intended.

We can overcome this potential objection and related complications (%
such as those arising from the possibility that parts of $E$ overlap with $D$%
) by modifying the definition of grounding.
In what follows we shall use the term `part' to include improper parts; accordingly, every event is part of itself.
Let us say that two events \emph{overlap} just if there is a part of one which is also part of the other.
More generally (this will be useful later),
two or more events \emph{overlap} just if there is a part of one of these events which is also a part of any of the other events.
Then let us revise Pietroski's definition by stipulating that:
%
\begin{quote}
\textbf{singular grounding revised} 
Event $D$ \emph{grounds} $E$  just if: $D$ and $E$ occur; 
$D$ is a  part of $E$; and 
$D$ causes every event that is a part of $E$ but does not overlap $D$.
\end{quote}
%
On the revised definition it is uncontroversial that the toppling of the first in a line of dominos grounds the toppling of the whole line.

The second part of Pietroski's proposal is along these lines (we ignore some complications not relevant for present purposes): for any event, whether or not it is an action, to be an agent of that event is to be an agent of an action which grounds it (p.\ 82).
(Note that since every event grounds itself, the proposal incorporates prior truths about agency.)
This provides an attenuated sense in which an individual can be an agent of an event even if the event is not an action.

To illustrate, suppose that Coralie shoots George.
As we saw, according to Davidson and others, none of Coralie's actions involve the movement of a bullet or the death of George.
But it is plausible that Coralie's action (whatever exactly it is)  grounds an episode that starts with her action and ends with George's death.
In this case her actions ground the episode and she is an agent of it. 

Given the standard views about what actions are discussed above (section \vref{section_first_objection}), there is a strong argument for accepting Pietroski's proposal.
The argument is simply that, in ordinary thinking about action, people do identify individuals as agents of events which are not actions, and their doing so appears to serve practical purposes.

One way to avoid the first objection to the simple definition of joint action is to revise it by adopting a notion of agency attenuated along the lines indicated by Pietroski.
There is an obstacle to doing this: Pietroski's proposal is limited to cases involving just one agent.
To get around the obstacle we first have to generalise his definition of grounding so that an event can be grounded by any number of events, not just one:
%
\begin{quote}
\textbf{plural grounding
	\label{df_plural_grounding}	
}
Events $D_1$, ...\ $D_n$ \emph{ground} $E$, if: $D_1$, ...\ $D_n$ and $E$ occur; 
$D_1$, ...\ $D_n$ are each part of $E$; and 
every event that is 
	a part of $E$
	but does not overlap $D_1$, ...\ $D_n$ 
is caused by some or all of $D_1$, ...\ $D_n$.
\end{quote}
%
For example, to return to the earlier illustration, Nora's and Olive's shootings ground Fred's killing, the event which starts with the shootings and ends with his death.

%The generalised definition of grounding has the consequence that if events $D_1$ and $D_2$ ground  $E$ and $D_1$ causes $D_2$, then, given that causation is a transitive relation, $D_1$ alone will also ground $E$.
%This and other ways in which the grounding relation may not be unique call for caution in generalising Pietroski's statement about agency.
We must be cautious in generalising Pietroski's statement about agency because the grounding relation is not unique; 
that is, it is possible for an event to be grounded by more than one set of events.
In particular, we need to allow for the possibility that more than one set of actions may ground an event.
This can be done as follows:
%
\begin{quote}
For an individual to be among the agents of an event is for there to be actions $a_1$, ...\ $a_n$ which ground this event where the individual is an agent of one or more of these actions.
\end{quote}
%
So where some actions ground an event, all the agents of those actions are agents of the event; and only agents of actions which ground the event are agents of the event.

How does this proposal apply to Nora and Olive's killing of Fred?
Consider the whole episode, the episode encompassing their two shootings and Fred's death.
Each shooting involves an action of which Nora or Olive is an agent
and these two shootings ground the whole episode.
So on this proposal Nora and Olive are agents of the whole episode, Fred's killing. 

The proposal is also consistent with the view that in paradigm cases of joint action, such as two people's painting a house together, there are events with two or more agents.
The proposal thus allows us to combine two claims about paradigm cases of joint action: first, that they do not involve \emph{actions} with more than one agent (as standard views about action require); and, second, that they do involve \emph{events} with more than one agent.

Accordingly one way of avoiding the first objection to the simple definition of joint action is to revise it in line with this proposal.  
On the revised simple definition, a joint action is an event with more than one agent.\footnote{
Ludwig appears not to distinguish between the claim that 
a joint action is `an \emph{event} of which there are multiple agents' \citep[p.\ 366]{ludwig_collective_2007}
and the claim that
a joint action is `an \emph{action} of which there are multiple agents'
\citep[p.\ 367]{ludwig_collective_2007}
(my emphasis).
However, his view is very close to what we are calling the revised simple definition (see pp. 375--6).
}


\section{Second Objection}
The revised simple definition of joint action clearly avoids the first objection, for it no longer fails to classify paradigm cases as joint actions.
But in revising the definition to avoid this objection we have left it open to a converse objection.
As we shall see, 
whereas the original simple definition was too narrow,
the revised definition appears to be too broad, classifying as joint actions events which arguably should not be so classified.

To illustrate, return to Nora and Olive's killing of Fred.
On the revised simple definition, this event is a joint action just because Nora and Olive are both agents of it.
Now suppose that Nora and Olive have no knowledge of each other, nor of each other's actions, and that their efforts are entirely uncoordinated.
We might even suppose that Nora and Olive are so antagonistic to each other that they would, if either knew the other's location, turn their guns on each other.
The event of their killing Fred is nevertheless a joint action on the revised simple definition.
But unless one thinks of the central event of \emph{Reservoir Dogs} \citep{Tarantino:1992fk} as a joint action, this is likely to seem counterintuitive.

The problem for the revised simple definition is general.
Whenever two or more agents' actions have a common effect and there is an event comprising the actions and their common effect,
the actions will ground this event.
And on the revised simple definition this is sufficient for the event to be a joint action.
This makes it plausible that the revised simple definition does not identify a notion of joint action that is appropriately linked to the paradigm cases we started with.

Note that the revised simple definition is not only too broad relative to notions linked to these paradigm cases.
The definition is also too broad in the sense that it counts as joint actions many cases which are not relevant to scientific inquiries about joint action.
For instance, the hypothesis that joint action grounds cognition \citep[p.\ 103]{Knoblich:2006bn} is clearly not a hypothesis about the kind of joint action (if any) exemplified by uncoordinated joint shootings.

But is the revised simple definition really too broad?
Perhaps rather than reject this definition, it would be better to reject the assumptions that shape our inquiry.
Since we cannot always rely on intuition, perhaps we should not insist that a notion of joint action be such that an implicit understanding of it can be gained by reflection on certain paradigm cases.
And since it may be a mistake to expect a \emph{definition} of joint action to pick out only cases which are theoretically significant relevant to certain explanatory projects,
perhaps we should not insist on this either.

We cannot claim to have shown that the revised definition is defective as a definition since it is arguable that the constraints we are assuming are not constraints on any adequate definition.
But, on the other hand, these constraints define our project and are appropriate given its motivation (see section \vref{section_motivation}).
What we can say, then, is that the revised simple definition, even if not defective as a definition, does fail to identify the notion or notions of joint action relevant to inquiries about joint action.
Further work is needed to show that there is an alternative to characterising joint action in terms of shared intention.
In what follows we offer ways of narrowing the revised simple definition.



\section{Goal-directed Joint Action}
According to the revised simple definition, a  joint action is an event with two or more agents.
What is missing from this definition?
All of the cases of joint action that serve as paradigms in  philosophy or psychology are \emph{goal-directed joint actions}.
That is, they are cases where the event taken as a whole is directed to a goal.

To illustrate, consider two people pulling handles in sequence to make a dog-puppet sing \citep[this example is from][]{Brownell:2006gu}.
One, asked about their action, might insist, `the goal of our actions was not to turn the light on but to make the dog sing'.
(Note that this sentence concerns the goal of an action and does not explicitly mention intentions.)
What is it for the goal of these agents' actions to be that of making the dog-puppet sing?
Among all the actual and possible outcomes of their actions, what distinguishes this as the goal to which their action was directed?
More generally, {what is the relation between a joint action and the goal (or goals) to which it is directed?}

In answering this question it is tempting to appeal to shared intention.
A shared intention  functions to coordinate agents' contributions and involves states which represent an outcome.
Where joint action involves shared intention, 
the goal represented by the shared intention coordinating the joint action is the goal to which the joint action is directed.\footnote{
Here and in what follows we simplify exposition by writing as if no action were directed to more than one goal.
}

As explained above, our aim is to avoid appeal to shared intention in characterising joint action.
Is there  a way of understanding how joint actions can be goal-directed which does not involve shared intention?
We shall answer this question with a series of increasingly elaborate notions.  
Each notion describes a relation between multiple agent's actions and an outcome to which those actions are directed.



\section{Distributive Goals}
\label{section_distributive}

The first notion in the series is that of a distributive goal.
An outcome is a \emph{distributive goal} of two or more agents' actions just if two conditions are met.
First, this outcome is a goal to which each agent's actions are individually directed.
Second, each agent's actions are related to the goal in such a way that it is possible for all the agents (not just any agent, all of them together) to succeed relative to this goal.

To illustrate, one dark night two communists each independently intend to paint a large bridge red.   
More exactly, each intends that her painting ground, or partially ground, the bridge's being painted red.\footnote{
Event $D$ \emph{partially grounds} event $E$ if there are events including $D$ which ground $E$.
(So any event which grounds $E$ thereby also partially grounds $E$; 
we nevertheless describe actions as `grounding or partially grounding' events for emphasis.)
See the definition of \emph{plural grounding} \vpageref{df_plural_grounding}.
}  
(These intentions ensure that it is possible for both communists to succeed in painting the bridge, as well as for either of them to succeed alone)
Because the bridge is large and they start from different ends, the two communists have no idea of each other's involvement until they meet in the middle.
Nor did they expect that anyone else would be involved in painting the bridge red.  
On almost any account, this implies that they were not acting on a shared intention.
Despite this, 
they both succeed in painting the bridge red. 
As this illustration suggests, 
it is possible for two or more agents' actions to have a distributive goal without the agents having any knowledge of, or intentions about, each other, and without the agents having a shared intention.

%Where to introduce this material?
%Objection insufficiently precise as it stands
%It may be objected that the intention which makes it possible for the communists' actions to have a distributive goal is of a kind not frequently found in everyday situations because it involves the somewhat technical notion of actions grounding events.
%We would argue, however, that this notion has counterparts in ordinary agents' thinking.
%develop using examples from Shared and Collective Intentions 2.tex?

As already mentioned, for two or more agents' actions to have a distributive goal it is necessary that each agent's actions are related to the goal in a way that allows all the agents to succeed relative to that goal.  
This condition would arguably not have been met if the bridge painters had each intended that she paint the bridge red.
Although this intention would have ensured that there was a single goal---the bridge's being painted red---to which each agent's actions were directed, it is arguable that neither agent's intention would have been be fulfilled because neither was the sole painter.
Distributive goals require intentions (or other ways of relating actions to goals) that are compatible with others' involvement.
This requirement is met by intentions which are explicitly neutral concerning others' success.  
For example, one of the bridge painters might have intended that she paint the bridge either alone or with others.
As illustrated in the above example, the requirement is also met by some intentions whose contents do not explicitly specify who might act, such as intentions to act in ways that ground, or partially ground, an outcome's occurrence (grounding is defined \vpageref{df_plural_grounding}).


Note that multiple agents' actions do not have a distributive goal just in virtue of their actions being directed to similar outcomes.  
In an example from Searle (\citeyear[p.\ 92]{Searle:1990em}), rain causes park visitors simultaneously to take cover under a central shelter.  
Suppose that each visitor's action is directed to a similar but distinct outcome, namely her own arrival at the shelter.  
These outcomes are so similar that each visitor could describe the outcome to which her actions are directed in the same words (`I reach the shelter').
Despite this, they are clearly distinct outcomes, for one could occur while another does not; one visitor might have made it to the shelter while another fell into the lake.
If each visitor's actions are directed only to her own arrival at the shelter, the visitors' actions lack a distributive goal just because there is no outcome to which they are all directed.
If, alternatively, each visitor's actions had been directed to their collective arrival at the shelter, then their actions would have had a distributive goal.

To recap, on the revised simple definition of joint action, a joint action is an event with two or more agents.
On this definition, many events which are perhaps not intuitively joint actions and apparently not relevant to philosophical or scientific questions about joint action, such as Nora and Olive's killing of Fred, are in fact joint actions.
As we saw, this raises the question of whether the revised simple definition can be narrowed. 
The general proposal under consideration is that the definition be narrowed to goal-directed joint action without appealing to shared intention.
The question of detail is how to explain the relation between a joint action and its goal without appeal to shared intention.
The notion of a distributive goal suggests one candidate answer to this question.
Where multiple agents' actions have a distributive goal there is a sense in which their actions are directed to a goal.  

But would invoking distributive goals enable us to suitably narrow the definition of joint action?
In their killing of Fred, Nora and Olive's actions might have a distributive goal.
After all, each agent's actions were individually directed to Fred's death.
Furthermore, it is consistent with the stipulations made about this scenario that these goal relations were compatible in the sense that both agents could succeed together.
(%
If the goal relations hold in virtue of Nora and Olive each acting on an intention that her shooting ground Fred's killing, then the goal relations are compatible in this sense; but if instead Nora and Olive each intended that she kill Fred, then arguably the goal relations are not compatible in this sense and so there is no distributive goal.%
)
So if we were to further revise the simple definition of joint action by invoking the notion of a distributive goal, we would barely improve on the revised simple definition.

Where multiple agents' actions have a distributive goal, it is true that there is a sense in which their actions are directed to a goal, 
but this may amount only to each agent's activities being individually directed to that goal.  
For significant cases of joint action we need a notion richer than that of a distributive goal, one that relates joint actions to goals without this being only a matter of each agent's activities being individually directed to the goal.





\section{Collective Goals}
\label{section_collective}

Let an outcome, possible or actual, be a \emph{collective goal \label{df_collective_goal}} of a joint action, or of any collection of goal-directed activities, where three conditions are met: 
	(a) this outcome is a distributive goal of the activities; 
	(b) the activities are coordinated; and 
	(c)  coordination of this type would normally  facilitate occurrences of outcomes of this type.  
Examples of actions  that typically have collective goals include two people jointly sawing a log with a two-handled saw and  
three people jointly lifting a heavy table.
The communist bridge painters (from section \vref{section_distributive}) are different: their actions do not have a collective goal because they are not coordinated.

The notion of a collective goal assumes that of coordination.  This should be understood in a broad sense.  
When two agents between them lift a heavy block by means of each agent pulling on either end of a rope connected to the block via a system of pulleys, their pullings count as coordinated in this broad sense.  
In this case, the agents' actions are coordinated by a mechanism in their environment, the rope, and not necessarily by any psychological mechanism.  
By invoking a broad notion of coordination 
and invoking coordination of actions rather than of agents,
the definition of collective goal avoids direct appeal to psychological states.

In characterising collective goals we have appealed to facts about what would \emph{normally} happen (in the third clause, (c), above).  
The relevant notion of normal paradigmatically features in statements like \emph{Birds can normally fly}.  
This notion is arguably teleological; certainly it is not  straightforwardly statistical or normative.\footnote{
Detailed discussion of the nature of the relevant notion of  \emph{normal} would take us too far from the present topic.
For teleological accounts, see 
	\citet[p.\ 33ff.]{Millikan:1984ib} and 
	\citet[p.\ 48ff.]{Price:2001hs}.
}
Conceptually it would be simpler to characterise collective goals by appeal only to what actually happens---that is, to replace the third clause with the requirement that the coordination actually facilitate the occurrence of the goal-outcome.  
Why is the appeal to what would normally happen necessary? 
Consider a case in which two agents' actions do have a collective goal and coordination of their activities actions facilitates the goal-outcome's occurrence: John and Anika fell a tree using a two-handled saw.  
Now imagine a case which is as similar as possible to this one except that John becomes exhausted and they have to give up half way through.  
In this modified case the coordination of John's and Anika's actions does not facilitate the occurrence of the outcome (the felling of the tree).
This is simply because the outcome does not occur.  
But the differences between the two cases are not the sorts of difference that generally determine facts about which goals an action is directed to.  
Whether actions succeed or fail does not generally play any role in determining what their goals were.
So it seems we must allow that agents' actions can have collective goals even where they fail.  
This is one reason for appealing to what would normally happen in characterising collective goals.  
A second, more direct but less obvious reason involves external factors which render coordination inefficacious.  For an illustration, suppose that Isabel and Rudi are in the habit of lifting heavy blocks by each pulling on a handle which is linked to the block by an intricate system of ropes and pulleys.  
Normally and on nearly all occasions either could lift any of the blocks alone but, providing their pullings are coordinated, the task is easier when done jointly;
and no matter how uncoordinated they are, the way the ropes are arranged means that it is never normally harder for them to lift a block jointly than alone.  
Normally, then, coordination facilitates the blocks being lifted.  
On one exceptional occasion John, a third person, intervenes.  John dislikes coordination between people and so, seeing the coordination of Isabel's and Rudi's activities, he grabs a rope and attempts to prevent the block being lifted.  Although John fails, he does make the joint lifting harder than it would have been for either Isabel or Rudi to lift the block alone.  
So in this exceptional case the coordination of their actions actually hinders rather than facilitates the blocks being lifted:
had their activities not been coordinated, it would have been easier for them to lift the block.
But this case, where John intervenes, does not differ from the normal cases in ways that are relevant to facts about the goals of the agents' actions.  
For this reason it seems that we must allow that Isabel's and Rudi's actions have a collective goal in the case where John intervenes as well as in the normal cases.
This is why appeal to what would normally happen is  necessary in characterising collective goals.



The word `collective' in `collective goal' should not be understood to imply that the agents  involved constitute a collective in any social sense.  Nor does having a collective goal imply that the agents think of themselves as having a collective goal.  The use of `collective' and `distributive' reflects  (but does not exactly match) the use of these terms in literature on plural quantification.\footnote{
On a widely accepted view,  the predication in `The goal of their actions was to lift this block' could be interpreted  either distributively and collectively.  On the distributive reading, the truth of the sentence is entailed by the truth of `For each of their actions, the goal of that action was to lift this block'.  On the collective reading this entailment does not necessarily hold.
See further the overview in Linnebo (\citeyear{Linnebo:2005ig}).  
}
The point is that for multiple agents' goal-directed activities to have a certain collective goal is not equivalent to each of their activities separately having that goal.


Where two or more agents' actions have a collective goal there is a sense in which, taken together, their actions are directed to the collective goal.  
It is not just that each agent individually pursues the collective goal; in addition, there is coordination among their activities which plays a role in bringing about the collective goal.  
We can put this in terms of the direction metaphor.  
Any structure or mechanism providing this coordination is directing the agents' activities to the collective goal.  
The notion of a collective goal provides one way of making sense of the idea that joint actions are goal-directed actions.

Recall how we came to introduce the notion of a  collective goal.
According to the revised simple definition of joint action, any event with two or more agents is a joint action.
This definition appears to be too broad.
A first idea about how to narrow it involved a restriction to goal-directed joint actions.
However, in making this idea more precise by appeal to the notion of a distributive goal, it emerged that the further narrowing was needed.
The notion of a collective goal does provide significant further narrowing.
Here, then, is a final attempt to characterise one notion of joint action.
Two or more agents' actions constitute a joint action when, taken together, 
	(a) there is an event which they collectively ground,\footnote{
	This condition is vacuous on some views of events; in particular, if any actions taken together constitute an event, then the actions will ground that event and so condition (a) can be met trivially.
	}
	 and
	(b) they have a collective goal.

This notion of joint action remains very broad, including cases that many philosophers would not recognise as joint actions.
Nevertheless we claim that it plausibly meets the conditions we gave for being a notion of joint action.
This notion is plausibly
central to a tangle of philosophical and scientific questions commonly taken to be questions about joint action.
And an implicit conception of this notion is plausibly available through reflection on most of the paradigm cases from our opening paragraph.



\section{Collective Goals vs.\ Shared Intentions}
Our aim in  introducing the notion of a collective goal was to understand how events with two or more agents could be goal-directed in the absence of shared intention.  
How can we be sure that multiple agents' actions having a collective goal does not amount to the agents having a shared intention?

Earlier we noted that, despite several divergent accounts of what shared intention is, there is broad agreement that three conditions are each individually necessary for shared intention (\vpageref{shared_intention_conditions}).
The existence of a collective goal does not require that any of these conditions are met.
Why not?
The requirements for agents' activities to have a collective goal can be met without the agents being aware that they are met 
(these requirements are given \vpageref{df_collective_goal}).  
This is because two agents' activities can be coordinated without either agent intending to coordinate, and even without the agents being aware that their activities are coordinated \citep{Sebanz:2003kf, schmidt_understanding_2010}.
It follows that two agents' activities might have a collective goal 
and that the agents might be engaged in a joint action together
even though each agent's knowledge states and beliefs are consistent with the possibility that she is acting alone.  
In particular, then, agents whose actions have a collective goal might lack awareness of joint-ness, awareness of each others' agency and awareness of others' states or commitments.
This shows not only that collective goals are conceptually distinct from shared intentions, but also that events  could involve collective goals without involving shared intentions. 

There are other, more intuitive, ways of contrasting shared intentions with collective goals.
Shared intentions often modelled on states of agents.  By contrast a collective goal is primarily an attribute of activities, not of agents.  Furthermore, a shared intention plays a role in the coordination of multiple agents' activities.  Appeal to a shared intention is therefore potentially a way of explaining how agents coordinate their activities.  By contrast, collective goals are not the sort of thing that can coordinate anything; their existence presupposes coordination.  Finally, shared intentions are supposed to be shared by agents in something resembling the sense in which conspirators can share a secret and not only in the sense in which two people can share a name.  
For two people to share a name it is sufficient that each is individually  so named.  
Collective goals are shared in this sense, but not in the stronger (although rarely explicated) senses associated with `shared intention'.
It would be a distortion to claim that the notion of a collective goal is the notion of something agents share.

So far we have shown that collective goals are conceptually distinct from shared intentions.
We have also argued that some events feature collective goals but not shared intentions.
We shall now argue that the converse is also possible.
That is, agents sometimes share intentions without their actions having any corresponding collective goal.  
Suppose that four merchants get together and agree to fix their prices with the shared intention that they will each become rich enough to retire by the end of the year.  
The year turns out to be good for trade and the merchants'  shared intention is realised.
As it happens these merchants are excellent traders but poor strategists.
Their profits would have been even larger if they had not colluded in fixing prices.
In fact their price-fixing strategy was so flawed that coordination of this type could not normally have a positive effect on profits.
So their attempts at coordination hindered rather than facilitated the realisation of their shared intention.  
Because the existence of a collective goal requires coordination of agents' activities to be of a type instances of which would normally facilitate occurrences of  goals of this type, no collective goal of the merchants' activities as here described corresponds to their shared intention.
(Of course their activities are directed to the goal of collective enrichment \emph{in some sense}, just not in the sense identified by the above characterisation of the notion of a collective goal.)

The possibility that shared intentions exist without there being corresponding collective goals is not a superficial feature of the way we have defined collective goal.  
Where there are shared intentions, one or more propositional attitudes serve to link the agents and their goals with any  means of coordination.  
For example, the merchants described above believe, falsely, that coordinating their pricing will facilitate fulfilment of their shared intention.  
Apart from their beliefs and other propositional attitudes, nothing appropriately connects their coordination to their activities and their goal.
In characterising collective goals our aim is to better understand how joint actions are related to their goals when such propositional attitudes are absent.  
In the case of collective goals, then, the link between a goal and a means of coordination can only involve facts, not beliefs or expectations, about its efficacy.
This is why, on standard accounts of shared intention, it is possible for agents to act on shared intentions without there being any corresponding collective goal.  

This matters if, as we claim, there is a notion of joint action which involves collective goals and not shared intentions.
The possibility of collective goals without shared intentions shows that a notion of joint action on which all joint actions involve shared intention fails to be maximally broad in this sense: there is another notion of joint action instances of which are not instances of this notion.
We also assume that shared intention is sufficient for joint action; more precisely, we assume that it is sufficient for two or more agents' actions to constitute a joint action that the agents and their actions are appropriately related to a shared intention.
The possibility of shared intentions with no corresponding collective goals shows that a notion of joint action characterised by collective goals also fails to be maximally broad.
So we cannot claim that collective goals enable us to characterise \emph{the} notion of joint action.
At most we can claim that collective goals enable us to characterise one of several notions of joint action.
The notion of joint action is heterogeneous or else there is a more fundamental approach to characterising it, one which relies neither on collective goals nor on shared intentions.
Either way, it follows that there is a notion of joint action which does not involve shared intention.

\section{Conclusion}
Differences in the ways that actions (joint or individual) relate to their goals can demarcate different kinds of action.
We saw that the notions of collective goal and shared intention are associated with two notions of joint action.
There is no straightforward hierarchical relation between these notions.  
For, as we saw, some joint actions involve collective goals but no shared intentions while others involve shared intentions without corresponding collective goals.  
There are, however, several ways in which these two notions of joint action may be related.
First, the contents of propositional attitudes which are or comprise shared intentions can refer to joint actions involving collective goals; because collective goals do not constitutively involve intentions, this involves no threat of circularity and raises no issues about well-foundedness.
Second, joint actions involving only collective goals may be proper parts of larger structures which do involve shared intention, much as (on some views) merely purposive activities can be components of actions that are intentional in a stronger sense.
Third, collective goals and the associated form of joint action may be a precursor, in evolution or development (or both), to the potentially more cognitively and conceptually demanding forms of joint action associated with shared intention.



Notions of joint action on which it is goal-directed but need not involve shared intentions have been neglected by philosophers, perhaps 
 partly because on some views it is tempting to assume that this combination of features is impossible, and 
partly because they are, or are thought to be, too simple to present conceptual puzzles.
This is a mistake.  
To understand the cognitive bases of abilities to engage in joint action or their evolution or development, the fact that a conception of joint action presents conceptual puzzles is no virtue.
It may be better to start with the simplest possible notions, such as those of distributive and collective goal, and add the minimum required to further demarcate the category of interest in any given inquiry.



\begin{comment}
This problem is that the definition applies to too broad a range of cases.  
To illustrate, imagine a section of stone wall which is very stable but has one weakness: if two small pieces are removed, the wall will collapse.
Now suppose that George removes one of these pieces and Philip removes the other, which causes the wall to collapse. 
George's and Philip's removals ground the event of the wall's collapse.
This is sufficient, on the revised simple definition, for George and Phillip to be agents of the wall's collapse.
And on the revised simple definition this in turn means that an event comprising their removal of the pieces and the wall's collapse is a joint action.
This is so even if  George's and Philips' actions were uncoordinated (other than by virtue of both involving the wall) and even if there was no contact at all between George and Phillip.
They may have no idea of each other's role; their actions may even have occurred hundreds of years apart.
Further, neither need have intended to destroy the wall.
Perhaps George needed a piece of stone and Philip's intention was to hide a ring in the wall.
Despite all this their destroying the wall would count as a joint action on the revised simple definition.
\end{comment}




\begin{comment}

Why is this a problem?
*Interesting questions about joint action focus on narrower range of cases.
*Good question, though, whether a \emph{definition} has to narrow all the way down to those cases.
*I will suggest that we can narrow down the range of cases in a variety of ways.
*Idea will be to argue that since there are multiple incompatible ways of narrowing down the definition (incompatible in the sense of there being no one way of  narrowing the definition which applies to all cases to which any other way of narrowing the definition applies) each of which has something theoretical going for it.
*So the conclusion will be that we should stick to the too-wide definition rather than attempt to narrow it further.


PLAN: In the light of this problem (the too-wideness of the definition), pursue the alternative resolution.
Different conception of action.
Still end up with the same problem ... the definition includes cases which are very distant from the paradigm cases.

\section{Alternative resolution of the first problem}

Second, reject standard views about action.

What about rejecting standard views of action?
There are alternative views about action on which the simple definition of joint action does not face the problems I have identified 
\citep[see][for discussion]{chant_special_2006,chant_unintentional_2007}.
Even so, it would be better to have a definition of joint action that is compatible with the most developed and best supported views about action, rather than one that works only given controversial alternative metaphysical assumptions.
For this reason it would be a mistake simply to reject standard views of action.

There is a complication.
Many of the arguments for the standard views of action which generate the problems depend on intuitions, and, in particular, intuitions about inferential relations among action sentences. 
So far only intuitions concerning individual actions have been considered in this context \citep[e.g.][]{Davidson:1971fz,hornsby_actions_1980,pietroski_actions_1998}.
There appears to be no good reason why, on these approaches, the focus should be on actions with just one agent or on sentences involving just one agent.
In ordinary thought and talk about action, cases involving two or more agents are not clearly demarcated from those involving just one.
Perhaps including cases of joint as well as individual action as inputs to the analysis would undermine some of the arguments.
If so, it may be that problem I have identified with the simplest definition of joint action is not a consequence of valid argument but merely a side-effect of failure by philosophers of action to consider the full range of relevant cases.
I do not believe that this is in fact the case.
But since a full discussion of this point requires evaluating arguments for claims central to the standard view of action, I do not want to pursue this here.
So I shall simply set aside the possibility that the existence of joint action requires revisions to parts of the standard view on which the above argument depends and which would cause that argument to fail.
If (as I suggest) no such revisions are required, we cannot accept the simplest definition of joint action.

\end{comment}





\bibliography{$HOME/endnote/phd_biblio}

\end{document}