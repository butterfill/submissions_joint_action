%!TEX TS-program = xelatex
%!TEX encoding = UTF-8 Unicode

\documentclass[12pt,a4paper]{extarticle}
% extarticle is like article but can handle 8pt, 9pt, 10pt, 11pt, 12pt, 14pt, 17pt, and 20pt text

\def \ititle {Joint Action Leipzig Talk:}
\def \isubtitle { Joint Action without Shared Intention}
\def \iauthor {Stephen A. Butterfill}
\def \iemail{s.butterfill@warwick.ac.uk}
%\date{}

\input{$HOME/Documents/submissions/preamble_steve_paper}

\begin{document}

\setlength\footnotesep{1em}

\bibliographystyle{newapa} %apalike

\maketitle
%\tableofcontents



\section{Orientation}
A joint action is a goal-directed action, or something resembling one, comprising two or more agents' goal-directed activities. 

Philosophers' paradigm cases of joint action include painting the house together (Michael Bratman), lifting a heavy sofa together (David Velleman), preparing a hollandaise sauce together (John Searle), going to Chicago together (Christopher Kutz), and walking together (Margaret Gilbert).


There are some inspiring claims about joint action in the literature ...
%
\begin{quote} 
`the unique aspects of human cognition ... were driven by, or even constituted by, social co-operation'
(Moll \& Tomasello 2007)
\end{quote}
%
\begin{quote} 
`perception, action, and cognition are grounded in social interaction'
(Sebanz \& Knoblich 2008)
\end{quote}
%
What concepts will help us to understand and test these ambitious claims about joint action?

In thinking about joint action most researchers, especially philosophers, have focused almost exclusively on a single concept, \textbf{shared intention}.  
I'll say more about what shared intentions are later.  For now, let's just say that shared intentions are supposed to stand to joint actions as ordinary intentions stand to individual actions.

My aims in this talk are to persuade you that it is a mistake to focus exclusively on shared intention, and to introduce some  additional conceptual tools.  
To understand joint action and its potential significance in development and evolution we do need shared intention, for sure, but we also need additional conceptual tools.

In some ways it surprises me that this is a controversial claim.
Compare individual action ... goal-directed action requires several notions, not all of it is intentional ... can help us to understand the structure of intentional action (Velleman).  
My claim is just that something approximately parallel to this is true for the case of joint action.

To understand joint action and its potential role in evolution and development we need multiple conceptual tools.  This is not to deny that shared intention is important, just that it can't do everything.

This is a talk with two halves.  

In the first half I shall explain why, in addition to shared intention, we need a further notion, which I call the collective goal.  The hard part is to see why collective goals are needed.  Once you see the need for them, it's not too difficult to say what collective goals are.

In this first half we will be concerned with forms of joint action that are probably too primitive to be terribly exciting here.  
I want to consider the primitive cases because it seems to me useful to take a constructive approach to theorising about joint action. 
We should start with the minimal case and only gradually introduce more complexity \citep{vesper_minimal_2010}.   

In the second half we shall turn to richer, more interesting cases of joint action.

I should thank my collaborators Cordula Vesper, Natalie Sebanz and Guenther Knoblich---some of what follows is based on joint work.

\section{First Half}



\subsection{The Standard Story}

As I said, a joint action is a goal-directed action, or something resembling one, comprising two or more agents' goal-directed activities. 

A basic question about joint action is, 
\textbf{What is the relation between a joint action and the goal (or goals) to which it is directed?}  
%(Here and throughout we ignore for simplicity the possibility that joint actions may be directed to multiple goals.  The answers considered are consistent with this possibility.)

To illustrate, suppose Ayesha and Beatrice between them lift a table thereby releasing a cat whose paws were trapped and,  simultaneously, breaking a glass.  On standard theories of events, the event which initiated the cat's release is identical to the event which initiated the glass' destruction \citep{Davidson:1969ie}.  But observing this joint action we might wonder whether its goal was the release of the cat or the destruction of the glass (or both); and if we are in doubt about this then in an important sense we don't yet know which action Ayesha and Beatrice performed.  To identify their action we need not only a concrete event, something with temporal and spatial properties, but also an abstract outcome such as that of the cat's release \citep{Davidson:1971fz}.

Ayesha, asked about the table lifting episode mentioned above, might insist, `The goal of our intervention wasn't to smash the glass but to free the cat.'  (Note that this concerns the goal of the joint action and not---or not directly---the agents' intention.)  In terms of this example, we could put the question like this.  \textbf{What makes it true that the goal of Ayesha and Beatrice's action is that of releasing the cat rather than that of breaking the glass?}

Answering this question is necessary for saying what joint actions are and for understanding in what senses, if any, they are actions.
%*better sentence on significance of question?

The term `goal' has been used both for outcomes (as in `the goal of our struggles') and, perhaps improperly, for psychological states (it is in this sense that agents' goals might cause their actions).  The terms `goal-outcome' and `goal-state' make the distinction explicit.  Our question is about how joint actions are related to their goal-outcomes.

For individual (that is, not joint) action, the counterpart of this question has a standard answer.  In barest outline, an individual action consists in an agent acting on an intention (a goal-state).  The intention has a content which specifies an outcome.
%, namely the outcome whose occurrence would consist in the propositional content's being true.
This is the goal-outcome of the action.
It is the goal-outcome of the action in virtue of being specified by the intention (or goal-state) the agent is acting on. 

In essence, then, on the standard theory, an individual action is related to its goal-outcome in virtue of the agent's acting on one or more intentions or goal-states.


How do things stand in the case of joint action?  Can we extend the standard theory from individual to joint action?  


\subsubsection{Shared Intention}


The usual way of thinking about joint action starts with the premise that all significant cases of joint action involve shared intention (dissenters are mentioned below).  For instance:  
%
\begin{quote} 
`I take a collective action to involve a collective intention.'  \citep[p.\ 5]{Gilbert:2006wr}.
\end{quote}
%
\begin{quote}
`the key property of joint action lies in its internal component [...] in the participants’ having a “collective” or “shared” intention.' \citep[pp. 444-5]{alonso_shared_2009}.
\end{quote}
%
\begin{quote}
`Shared intentionality is the foundation upon which joint action is built.' \citep[p.\ 381]{Carpenter:2009wq}
\end{quote}


But what is shared intention?

In barest outline, shared intentions are supposed to do for multiple agents some of what ordinary intentions do for individuals.  So, like an ordinary intention, a shared intention coordinates plans and goal-directed activities---with the difference that these are the plans and goal-directed activities of multiple agents performing a joint action \citep{Bratman:1993je}.

The notion of shared intention provides a straightforward way to extend the standard theory about how actions are related to their goal-outcomes to cases of joint action.

In brief, the standard theory is extended by substituting shared for individual intentions and otherwise unchanged.  

So on the usual way of thinking about joint action, shared intentions have two functions.  They coordinate multiple agents' activities; and they determine which goals their actions are directed to.


\subsubsection{My aims}
I want to argue, not that this extension of the standard theory is incorrect, but just that it is not a full account of the relation between joint actions and their goals.

For I claim that there are significant cases of joint action without shared intention.

So I think the following propositions can both be true:

\begin{enumerate}

\item The goal of Ayesha and Beatrice's action is to free the cat.  [That is, their joint action has a goal-outcome.]

\item It is not true that Ayesha and Beatrice have any kind of  shared goal-state.  

\end{enumerate}

In what follows I shall first argue for the consistency of these propositions by giving you some examples of joint actions without shared intentions.  I then offer an account of how joint actions are related to their goal-outcomes which does not involve anything like shared intention.  


\subsection{Preliminary: Shared Intention}

But before any of that I need to fill in some background on shared intention.

My immediate aim, as I mentioned, is to show that there are cases of joint action without shared intention.  An immediate obstacle is \textbf{lack of agreement on what shared intentions are}.  

Some hold that shared intentions differ from individual intentions with respect to the attitude involved (\citealp{Kutz:2000si}; \citealp{Searle:1990em}). 
Others have explored the notion that shared intentions differ with respect to their subjects, which are plural \citep{Gilbert:1992rs}, 
or that they differ from individual intentions in the way they arise, namely through team reasoning \citep{Gold:2007zd}, 
or that shared intentions involve distinctive obligations or commitments to others (\citealp{Gilbert:1992rs}; \citealp{Roth:2004ki}).
Opposing all such views, \citet{Bratman:1992mi,Bratman:2009lv} argues that shared intentions can be realised by multiple ordinary individual intentions and other attitudes whose contents interlock in a distinctive way. 

Given that almost no two philosophers agree on what shared intentions are, how can we say that shared intention is not involved in any given case?  

On all or most leading accounts of shared intention, each of the following is a necessary condition:

\begin{idescription}
\label{conditions-for-shared-intention}

\item[awareness of joint-ness] at least one of the agents knows that they are not acting individually; she or they have `a conception of themselves as contributors to a collective end.'\footnote{
	\citet[p.\ 10]{Kutz:2000si}.  Compare \citet[p.\ 361]{Roth:2004ki}: `each participant ... can answer the question of what he is doing or will be doing by saying for example ``We are walking together'' or ``We will/intend to walk together.''' 
Relatedly, \citet[p. 56]{miller_social_2001} requires that each agent believes her actions are interdependent with the other agent's.
}

\item[awareness of others' agency]  at least one of the agents is aware of at least one of the others as an intentional agent.\footnote{
	Compare \citet[p.\ 333]{Bratman:1992mi}: `Cooperation ... is cooperation between intentional agents each of whom sees and treats the other as such'.  See also \citet[p.\ 105]{Searle:1990em}: `The biologically primitive sense of the other person as a candidate for shared intentionality is a necessary condition of all collective behavior' 
}

\item[awareness of others' states or commitments] at least one of the agents who are F-ing is aware of, or has individuating beliefs about, some of the others' intentions, beliefs or commitments concerning F.\footnote{
This condition is necessary for shared intention even on what \citet[p.\ 40]{tuomela_collective_2000} calls `the weakest kind of collective intention'.  But it may not be necessary if, as \citet{Gold:2007zd} suggest, shared intentions are constitutively intentions formed by a certain kind of reasoning.
% "if the distinctive feature of collective intentions is to be found in the reasoning by which they were formed, then an analysis that focuses on the intentions themselves will miss the feature that makes collective intentions collective. " 
}

\end{idescription}

There are philosophers who deny that shared intention is necessary for joint action,\footnote
{
\citet[p.\ 407]{Roth:2004ki} and \citet{Searle:1990em}  hold that the intentions required for joint action need not be shared; \citet{miller_social_2001} also denies that shared intention is necessary for joint action.
*Should say something about Bratman and others.
*Should possibly also mention Kutz on participatory intentions.
}
but even they hold that one or more of these conditions is individually necessary for joint action (see footnotes above).

What follows assumes that where one or more of these three conditions is not met, there is no shared intention. 

Some of what follows also makes use of the further assumption that these conditions express causal conditions on shared intention.  That is, where joint action involves shared intention, the agents act in part \emph{because} they have awareness of joint-ness, of others' agency or of others' states or commitments.




\subsection{Preliminary: Joint Action [skip?]}

Now we have necessary conditions for shared intention.  
Since I want to give examples of joint action without shared intention, I also need to give some sufficient conditions for joint action.  

The sufficient condition is already implicit in my project.

For there to be an interesting question about how a joint action is related to its goal-outcome, there has to be a sense in which all the agents' activities taken together have a goal where this isn’t simply a matter of each agent's activities individually having that goal. 

So the condition is this: 
\begin{quote}
Multiple agents' activities taken together have a goal-outcome which isn't just a matter of each agent's activities individually having that goal-outcome.
\end{quote}
I'm not saying that this a necessary condition for joint action.  I'm only saying that where this condition is met, there is an interesting question about how joint actions are related to their goals.

To see what sort of case this rules out, suppose that a large, brightly lit package is dropped from a helicopter.  Several people on the ground below attempt to catch the package.  These people are unaware of each other because there the helicopter has generated a mini dust storm and there is no coordination of their activities. 
Each person's goal is not that \emph{she} should catch the package but only that the package should be caught (perhaps they want to avoid its contents from breaking).  The package is large enough so that they could all have a hand in catching it, so it's possible for them all to succeed; but the package is also light enough so that one person could catch it alone, so the others are not necessary.   

As it happens, though, only one person does catch it.

In this case it is true that the people's activities were directed to the goal of catching the package.  But there is nothing more to this than the fact that each person's activities were individually directed to the goal of catching the package.



\subsection{Case Studies: Joint Action Without Shared Intention}


\subsubsection{Case Study---the environment coordinates joint actions}

Two ropes hanging over either side of a high wall are connected to a heavy block via a system of pulleys.  Ayesha and Beatrice each individually intend to raise the block.  
The positions of the walls mean that they can each see the block but they can't see each other.
Ayesha and Beatrice each know that, in addition to the rope they can pull, there is another rope and that force has to be exerted on that as well.  
But they have no idea about what will exert this force; for all they know it might be something mechanical or a fluke of nature rather than another agent.
They hold their own rope so that it's possible to feel when additional force is exerted on the other.  As soon as they feel such force, they attempt to lift the block.\footnote{
Compare the `Me plus X' notion from Vesper et al *ref
} 
This ensures that Ayesha and Beatrice pull the ropes simultaneously, causing the heavy block to rise as a common effect of their actions. 

Intuitively this is joint action, perhaps because the ropes and pulleys bind the agents' actions together and ensure a common effect.  And it also counts as a case of joint action by the criterion I offered above.  For there is a mechanism---the rope---which coordinates Ayesha and Beatrice's action and ensures that their individual pullings will normally cause the block to rise.  So their action's being directed to the goal of raising the block is not just a matter of each of their actions individually being directed to this goal.  

None of the above necessary conditions for shared intention are met: there is no awareness of joint-ness, no awareness of others' agency and no awareness of others' states or commitments either.  

We could make this example less bizarre by allowing that Ayesha and Beatrice can see each other.  As long as they are not aware of each other as agents, this still won't count as a case of shared intention.  

This artificial case indicates that in joint action much of the coordination can be taken care of by what objects afford multiple agents rather than by intentions. 



\subsubsection{Case Study---Kissing}
Kissing seems to be a good candidate for joint action because, like salsa dancing, it's not the sort of thing you can do alone.

But if the kiss is sufficiently spontaneous, it might not involve awareness of jointness or awareness of others' states or commitments in advance of success.  In this case, instead of shared intentions providing coordination, there are likely to be chemical and emotional means of coordination.



\subsubsection{Case Study---Motor Simulation [cut]}

There are other cases of where joint action involves meshing motor coordination rather than shared intentions.  Discussion of the details would take us too far from the main theme, but I would be happy to come back to these cases in discussion.

Overall, then, my suggestion is that coordination of two or more agents' activities can be provided by mechanisms in agents' environments, by emotional chemistry, and by meshing motor cognition.  
In each of these cases, the action's being directed to a goal involves more than each agent's activities individually being directed to this goal---in addition, there is an element of coordination.



\subsection{Collective Goals}
\label{section_collective}

So far I have argued that there are goal-directed joint actions without shared intentions.  
In the examples I offered, necessary conditions for shared intention are not met---there is no awareness of joint-ness, no awareness of others' agency or no beliefs about others' states or commitments.  But the examples do involve joint action.  For there is a sense in which multiple agents' activities taken collectively are directed to a goal, where this is not just a matter of each agent's activities being individually directed to that goal.
My question is, How can we characterise the relation between joint actions and their goal-outcomes without invoking shared intentions?  

In many (perhaps all) cases of joint action there is a single outcome to which each of the agent's activities are individually directed, where each agent's activities could succeed even were all other agents to play a role in realising the outcome.  
Let a \emph{distributive goal} of a joint action, or of a collection of goal-directed activities, be any such outcome.  
To illustrate, suppose that many revellers at a village festival jump into a single boat each with the aim of sinking it, where each reveller's aim would be met if the boat were to sink under their collective weight.  
That the boat sink is a distributive goal of their activities.


Where multiple agents activities have a distributive goal there is a sense in which their activities are directed to a goal.  But this may be only a matter of each agent's activities being individually directed to that goal.  The question about how joint actions relate to their goals could only be interestingly different from the parallel question about individual action where all the activities composing a joint action are directed to a goal and this is not only a matter of each agent's activities individually being directed to that goal.
What, in addition to a distributive goal, could be involved in two or more agents' activities being directed to a goal?

Let an outcome be a \emph{collective goal} of a joint action, or of any collection of goal-directed activities, where three conditions are met: (a) this outcome is a distributive goal of the activities; (b) the activities are coordinated; and (c) the outcome occurs, or would normally occur, partly as a consequence of this coordination.  Examples of activities that would typically have collective goals include uprooting a small tree together and kissing a baby together.

Where a joint action or collection of goal-directed activities has a collective goal there is a sense in which, taken together, the activities are directed to their collective goal.  It is not just that each agent individually pursues the collective goal; in addition, there is coordination among their activities which plays a role in bringing about the collective goal.  We can put this in terms of the direction metaphor.  Any structure or mechanism providing this coordination is directing the agents' activities to the collective goal.  The notion of a collective goal therefore already provides one way of making sense of the idea that joint actions are goal-directed actions.

The approach to explaining the relation between a joint action and its goal without invoking shared intention can now be stated.  A joint action is related to the goal to which it is directed in virtue of this goal  being the collective goal of the joint action.  The goal of a joint action is its collective goal.\footnote{  
Just as it is possible that some joint actions involve multiple shared intentions, so also might some involve involve multiple collective goals.  This minor complication is ignored only for ease of exposition.
}


\subsection{Shared intentions}
Collective goals are conceptually distinct from shared intentions.  Shared intentions are, or resemble, states of agents.  By contrast a collective goal is primarily an attribute of activities not of agents.  Furthermore, a shared intention is something that coordinates multiple agents' activities.  Appeal to a shared intention is therefore potentially a way of explaining how agents coordinate their activities.  By contrast, collective goals are not the sort of thing that can coordinate anything; their existence presupposes coordination.  Finally, shared intentions are supposed to be shared by agents in something resembling the sense in which conspirators can share a secret, not only in the sense in which two people can share a name.  It would be a distortion to claim that the notion of a collective goals is the notion of something agents share.







\section{Second Half}

So far I have suggested that understanding the relation between joint actions and their goals requires the notion of collective goal in addition to that of shared intention.

This is a modest claim.  It can be accommodated with at most small modification to any good account of joint action.  No revolution here.

The forms of joint action we have so far been considering are quite primitive and can be observed in many species, including some insect species.

They are much simpler than those found in humans and even in young children.  

Joint actions that one-year-old children engage in include tidying up the toys together (Behne, Carpenter and Tomasello 2005), cooperatively pulling handles in sequence to make a dog-puppet sing (Brownell, Ramani and Zerwas 2006), bouncing a ball on a large trampoline together (Tomasello and Carpenter 2007) and pretending to row a boat together. 

So from near the start of their second year or earlier, children engage in joint actions which are characterised by:
%
\begin{itemize}
\item sensitivity to others' mental states \citep{Buttelmann:2009gy}
\item dispositions to help partners \citep{Warneken:2006qe}
\item *role-swap (suggests awareness of the activity)
\item partner-specific projects \citep{Liebal:2010lr}
\item ...
\end{itemize}
%
In short, developmental and comparative research suggests that children are highly cooperative collaborators.  
In particular, their joint actions:
%
\begin{enumerate}
\item are voluntary with respect to their joint-ness
\item involve potentially novel goals
\item rely (at least in part) on psychological mechanisms for coordination
\end{enumerate}
%
This means that if we want to understand the potential significance of these joint actions in development and evolution, then we cannot adequately characterise using the notion of collective goal alone.

I'm not saying that these joint actions don't involve collective goals, only that they involve more than that.

A natural thought, then, is that we should characterise these joint actions by appeal to shared intention.

This is not straightforward because shared intentions are neither literally intentions or literally shared.  
You can't share an intention in the sense that you can share a bottle of wine.  
The term `shared intention' is just a colourful metaphor.  
In appealing to shared intention, we must avoid relying on \textbf{romantic ideas about sharing}.
Instead we have to say what we mean by shared intention.

As I mentioned earlier, there are many different and incompatible approaches to characterising shared intention.  
For now I will simply adopt Michael Bratman's account of shared intention.  
This account is generally taken as a point of departure by philosophers and some psychologists.
I don't claim that this is the whole story about shared intention.
But I do claim it is theoretical coherent, 
that no one has succeeded in identifying an objection to it in print, 
and that captures one  concept that is important for understanding joint action.

\subsection{Bratman's account of shared intention}

Bratman's account of shared intention has two parts, a specification of the functional role shared intentions play and a substantial account of what shared intentions could be.  On the first part, Bratman stipulates that the functional role of shared intentions is to: 
%
\begin{quote}
(i) coordinate activities; (ii) coordinate planning; and (iii) provide a framework to structure bargaining \citep[p.\ 99]{Bratman:1993je}
\end{quote}
%
To illustrate: if we jointly intend that we paint a house, this shared intention will (iii) structure bargaining insofar as we may need to decide what colours to paint it on the assumption that we are painting it together; the shared intention will also require us to (ii) coordinate our planning by each bringing complementary paints and tools, and to (i) coordinate our activities by painting different parts of the house without getting in each others’ way.

Given this claim about what shared intentions are for, Bratman argues that the following three conditions are jointly sufficient\footnote{
In (1993), Bratman offers the following as sufficient and necessary conditions; the retreat to merely sufficient conditions occurs in Bratman (1999 [1997]) where he notes that “for all that I have said, shared intention might be multiply realizable.”
}  
for you and I to have a shared intention that we J.  This is his substantial account of what shared intentions could be:
%
\begin{quote}
`1. (a) I intend that we J and (b) you intend that we J

`2. I intend that we J in accordance with and because of la, lb, and meshing subplans of la and lb; you intend that we J in accordance with and because of la, lb, and meshing subplans of la and lb

`3. 1 and 2 are common knowledge between us' \citep[View 4]{Bratman:1993je}
\end{quote}
%
In favourable circumstances the attitudes specified in these conditions are capable of playing the three roles shared intentions are supposed to play.  This is what it means to say that they constitute a shared intention.



\subsection{Shared Intention is cognitively demanding}
On the substantial account given by Bratman, sharing intentions requires intentions about intentions (see Condition 2 in the quote above).\footnote{
Bratman emphasises this feature of the account: “each agent does not just intend that the group perform the […] joint action. Rather, each agent intends as well that the group perform this joint action in accordance with subplans (of the intentions in favor of the joint action) that mesh” (Bratman 1992: 332).
}

Furthermore, each agent must know that the others have intentions about her own intentions; and this knowledge must be mutual (see Condition 3 above).  So sharing an intention involves knowing that someone else knows that I have intentions concerning subplans of their intentions.  
   
This exposes Bratman's view to the objection that it is \textbf{too cognitively demanding}.  
%
\begin{quote}
`philosophers ... postulate complex intentional structures that often seem to be beyond human cognitive ability in real-time social interactions.'
\citep[p.\ 2022]{Knoblich:2008hy}
\end{quote}
%
This objection needs careful handling.  
In his recent papers, Bratman doesn't say that shared intention necessary requires knowledge of others' knowledge of intentions about intentions.  He only says that this would be \emph{sufficient} for shared intention.

So the objection can't be based on the need for need for meta-meta-meta-representations.

The key thing to note here is that one of the functions of shared intentions is to coordinate planning.  

In Bratman’s account, the term `planning' is used in a narrow sense.  Planning in this narrow sense concerns the coordination of an agent’s various activities over relatively long intervals of time; it involves practical reasoning and forming intentions which may themselves require further planning, generating a hierachy of plans and subplans.  Paradigm cases include planning a birthday party or planning to move house.   

To share intentions is to be disposed to coordinate plans; because this requires recognising oneself and others as planning agents, it involves sophisticated insights into the nature of minds.  
Sharing intentions is cognitively demanding because coordinating plans is.
So even if sharing intentions didn’t require multiple levels of metarepresentation, it would still be cognitively demanding.


\subsection{Shared Intention is conceptually demanding}

***MOVE THIS TO THE CONCLUSION?

In addition to being cognitively demanding, shared intention is also conceptually demanding, and in particular requires sophisticated theory of mind cognition.  

Some philosophers (notably Deborah Tollefsen) have suggested that this would mean that young children cannot share intentions because they lack the conceptual sophistication to meta-meta-meta-represent.  I don't think this is a good line of objection.  Instead I want to offer a different consideration.

Suppose we think that possessing and exercising capacities for joint action might play a role in explaining the development or evolution of social cognition and in particular of theory of mind cognition.

This sort of view has been defended by Claire Hughes and Moll and Tomasello, amongst others.  In a paper on the Vygotskian Intelligence Hypothesis, Moll and Tomasello say:
%
\begin{quote}
`regular participation in cooperative, cultural interactions during ontogeny leads children to construct uniquely powerful forms of cognitive representation.'
\citep[pp.\ 2-3]{Moll:2007gu}
\end{quote}
%
If `cooperative, cultural interactions' involved shared intention, the hypothesis would be false.  For sharing intentions already presupposes the `powerful forms of cognitive representation' whose acquisition is supposed to be explained by participating in cooperative interaction.
So if the Vygotskian intelligence hypothesis is correct, it must be false that joint action necessarily involves shared intentions.
By the way, Moll and Tomasello explicitly adopt Bratman’s account of joint action:
%
\begin{quote}
`As in previous theoretical work […], we use here a modified version of Bratman’s (1992) definition of `shared cooperative activities'.'
\citep[p.\ 3]{Moll:2007gu}
\end{quote}
%
I think appeal to shared intention as characterised by Bratman is not compatible with the hypothesis that possessing or exercising capacities for joint action might explain development of sophisticated forms of social cognition, particularly theory of mind cognition.

The problem, once again, is that sharing intentions means coordinating one's planning with others' planning.
Social cognition doesn't get much more conceptually sophisticated than this.
So shared intention already presupposes too much social cognition to explain much about its development or evolution.


\subsection{Summary of the objections to Bratman's account}

Sharing intentions is both conceptually and cognitively demanding because it involves coordinating planning.

This is not an objection to Bratman's account of shared intention.  I am assuming that Bratman is right about what shared intentions are.

My claim is just that shared intention is not involved in simple cases of joint action where the agents are not making used of abilities to coordinate their plans.

In simple cases of joint actions, agents coordinate their activities but not their planning.

There is no need for shared intention here.

This leaves us with \textbf{a puzzle} about the cases of joint action mentioned earlier. 
Since these forms of joint action don’t seem to be especially cognitively demanding or to require deep insight into minds, they cannot involve shared intentions.  On the other hand, coordination is to an extent voluntary in these cases, so we cannot characterise them by appeal to collective goals only.

To resolve the puzzle I suggest that we need another entity, something which, unlike a collective goal, can account for joint actions where the joint-ness is voluntary and the ends potentially novel, and unlike a shared intention in that it does not impose the sort of cognitive demands involved in intentionally coordinating planning.

In short we need something intermediate between collective goals and shared intention.  I call it a shared goal.


\subsection{How to share goals}
Following the general model provided by Bratman’s account of shared intentions, my account of shared goals has two parts: a specification of their function role and a substantial description of states that could realise them.

Shared goals function is to coordinate activities. That is, they exist in order to coordinate multiple agents’ purposive activities around an outcome to be achieved as a common effect of their efforts. 

The next step is to characterise states capable of realising this function.

Here I want to start with the notion of collective goal and add the minimum necessary.  

Suppose two or more agents have a common goal.  What do we need to add to characterise joint actions which are voluntary with respect to jointness, directed to potentially novel goals and involve psychological states which coordinate the activities?

What we need to suppose is just that the agents are aware of the distributive goal and expect that their actions will succeed only in combination with others' efforts.

More explicitly we need to suppose that:
%
\begin{enumerate}
\item Each agent expects each of the other agents to perform activities directed to the goal.
\item Each agent expects the goal to occur as a common effect of all their goal-directed actions.
\end{enumerate}
%
In favourable circumstances this simple pattern of goals and expectations would be sufficient to coordinate the agents’ activities in bringing about this outcome. Since `shared goal' was defined in terms of this coordinating function, meeting these three conditions is sufficient for possessing shared goals.

To illustrate, Amin’s goal is to put a large barrel into a boat, Amin anticipates that Bertram has this goal, and Amin expects the barrel’s moving into the boat to occur as a common effect of his own goal-directed actions and Bertram’s; Bertram has some goal or other which also requires the same barrel to move into the same boat, and Bertram has expectations mirroring Amin’s. Their activities could be coordinated around moving the barrel into the boat as a common effect in virtue of this interlocking pattern of goals and expectations. Accordingly, meeting these conditions is sufficient for Amin and Bertram to have shared goals.

\subsection{shared goals are not shared intentions}

Shared goals are not shared intentions for two reasons.

First, on most accounts of shared intention, having a shared intention involves knowing that you have a shared intention, and therefore understanding what shared intentions are.  We'll see how this works in practice in a moment.

By contrast, shared goals do not have this property.  Having a shared goal does not require 


\subsection{*NOTES DELETE}
*CITE TOMASELLO ET AL FROM SLIDE FROM SOMEWHERE ?BHX

***Follow 'grounding social cognition' talk : introduced shared intentions ... idea is to set up a dilemma for explaining these cases, neither Bratman-shared-intention nor collective goal is enough.  We need something else.

Why not Bratman shared intention?  Moll \& Tomasello + Claire Hughes; also Sebanz \& Knoblich point about processing costs (for adults).

Shared goals ... done



\subsection{Why we need shared goals}

Here is a wild conjecture about joint action, one that is entirely at odds with the way that many, perhaps most, philosophers and psychologists think about joint action and theory of mind cognition:

\begin{quote}
Human social cognition, including full-blown theory of mind cognition, is built on capacities for joint action and their exercise.
\end{quote}
%
This is not the same as Moll and Tomasello's Vygotskian intelligence hypothesis,\footnote{
`the unique aspects of human cognition ... were driven by, or even constituted by, social co-operation'
\citep[p.\ 1]{Moll:2007gu}.
}
 but it is related.  The wild conjecture is also distantly related to Guenther Knoblich and Natalie Sebanz' hypothesis that human cognition generally is grounded in social interaction.\footnote{
`perception, action, and cognition are grounded in social interaction … functions traditionally considered hallmarks of individual cognition originated through the need to interact with others' \citep[p.\ 103]{Knoblich:2006bn}.
 }


There are at least two reasons why the wild conjecture, despite its imprecise formulation, seems completely untennable.  The first is empirical, the second conceptual.
%
\begin{enumerate}
\item The ability to ascribe false beliefs is a hallmark of full-blown theory of mind cognition; we now know that this ability appears at around 14 months or even earlier; and we may  find the ability to ascribe false beliefs in non-human primates or corvids as well.  So the objection is that, in both development and evolution, significant joint actions probably occur only after full-blown theory of mind cognition is established.

\item All joint actions involve shared intention; shared intention presupposes the ability to ascribe you knowledge of my intentions about your intentions; and this sophisticated capacity is already distinctive of human social cognition.  So the objection is that, once capacities for joint action appear, there is not much social cognition left for joint action to explain.
\end{enumerate}
%
I do have things to say about the first point but here I want to focus on the second.

[*go through it slowly; introduce Bratman here?]

So I'm not trying to establish the wild conjecture, only to remove one obstacle to its acceptance.

But the sorts of joint action I have focused on so far seem to be much too simple to serve this end.  To see why, consider some features of joint actions identified early in children's second year ...

\begin{itemize}
\item sensitivity to others' mental states \citep{Buttelmann:2009gy}
\item dispositions to help partners \citep{Warneken:2006qe}
\item partner-specific projects \citep{Liebal:2010lr}
\item ...
\end{itemize}
%
In short, developmental and comparative research suggests that children are highly cooperative collaborators.  By contrast, the sort of coordinated joint action I have described is common in many species, even some insect species.


\subsection{The Dilemma}

I can summarise the story so far by offering a dilemma for the wild conjecture.  If we accept the standard account of joint action which appeals to shared intention, then in appealing to joint action we are presupposing the existence of the social cognition which was to be explained.  If, on the other horn, we appeal only to collective goals in characterising joint action, then we end up with a form of joint action too basic to be a plausible basis for social cognition.

The way out of the dilemma is to identify intermediate forms of joint action, forms which involve more than collective goals but less than shared intentions.

The right approach to take here is a constructive one.  The notion of collective goal is useful not because it provides for an interesting notion of joint action but because it makes possible a constructive approach to characterising joint action and the psychological mechanisms that make it possible.

The notion of a collective goal is the minimum needed for joint action; it is the starting point for a constructive approach to joint action.  



\section{A constructive approach to characterising joint action}

goal-states vs. intentions


\section{Conclusion}
The notion of joint action involves extending theories from the individual, single-agent case to cases involving multiple agents.  There is no straightforward way to so extend these theories, any extension will be truer to some features of individual action at the expense of abandoning others.  Further, no single extension distinguishes itself as uniquely correct.

 

\bibliography{$HOME/endnote/phd_biblio}

\end{document}