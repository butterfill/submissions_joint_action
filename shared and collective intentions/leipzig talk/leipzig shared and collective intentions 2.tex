%!TEX TS-program = xelatex
%!TEX encoding = UTF-8 Unicode

\documentclass[14pt,a4paper]{extarticle}
% extarticle is like article but can handle 8pt, 9pt, 10pt, 11pt, 12pt, 14pt, 17pt, and 20pt text

\def \ititle {Joint Action Leipzig Talk:}
\def \isubtitle { Joint Action without Shared Intention}
\def \iauthor {Stephen A. Butterfill}
\def \iemail{s.butterfill@warwick.ac.uk}
%\date{}

\input{$HOME/Documents/submissions/preamble_steve_paper}

\begin{document}

\setlength\footnotesep{1em}

\bibliographystyle{newapa} %apalike

\maketitle
%\tableofcontents


\section{Leipzig questions}


\subsection{Mike Tomasello}
The expectations I refer to in defining shared goals are statistical, not normative?  Yes!

Second.  If what I say is true, how is it that chimps can't solve the Hare and Tomasello pointing paradigm but can solve the reaching paradigm?  Possibly because they don't think of the experimenter as engaged in joint action with them, either because they fail to pick up on any social cues or because they really don't expect cooperation (especially in the context of food---might be helpful what Alica Melis said in conversation, that chimps won't cooperate at all when food is at stake, but they will help another (by opening a door) when they are at play and there is no chance of them getting a reward for themselves).  It is also possible that chimpanzees (and children for that matter) do not naturally think that 'to get the reward' is the sort of goal that two people can have in the non-exclusive sense.

it's hard to deny that, in theory, capacity to share goals would be sufficient for success in this task.  And it's hard to deny that chimps can share goals (because they will do things together---I think Daniel Haun mentioned a study in which chimpanzees choose to cooperate about half the time---but generally choose not to).

AFTER I wonder if we could get chimps to pass the Hare and Tomasello 2004 paradigm.  Perhaps if the reward was not food but some means to getting food (Alicia Melis' bowl trick might help, although that didn't work in her study) and perhaps if the human had previously demonstrated willingness to help in other tasks (e.g.\ on the rope pulling, *** apparatus), and perhaps if the human had imitated the chimpanzee (which in Daniel Haun's studies didn't result in more helping but did result in the chimpanzee moving closer to the human, which might have been because imitation was a sign of submission).


\subsection{with Mike Tomasello after}
Shared intentions and planning.  The key is to understand that shared intentions are defined with respect to a special kind of planning, inter-goal planning.  Illustrate with respect to the individual case.  Planning in Bratman's sense involves sensitivity to constraints on the execution of one intention that arise from the fact of having to execute another intention as well.  (Simple example is timing constraints.  More complex example is where I need a car only because I intend both to X and to Y --- for either X or Y alone I could take the bus.)  

A good example of inter-goal planning in joint action comes from the tool-selection experiment (name of investigator is given in notes from Kathrina Hamann).  I need a clearer way of distinguishing coordinating activities from coordinating planning.  (E.g., consider that when we lift the table together I take one side whereas when lifting it alone I hold in the middle.  Why isn't this inter-goal planning?  Perhaps just because it doesn't involve practical reasoning---so it could be, but it is more natural to suppose that it no more involves practical reasoning than does any decision (in individual action) about how to grasp an object?)


\subsection{more with Mike Tomasello after}
How should we understand mutual knowledge?  Combine two ideas.  

First idea.  When we look out at those people shovelling snow in the street it is natural to talk about what they are doing, what they want, believe and intend.  This arguably doesn't involve ascribing states to individual agents---to test whether it does, we could ask whether adding agents increases the cognitive load involved in talk about what the group is doing, since there should be some cost per individual if we have to think about individuals.

This might arise as a consequence of being able to do the recursive thing.  Or it might arise as a consequence of over-willingness to ascribe mental states and actions to things which are neither thinkers or agents.

Second idea.  Sugden's team reasoning.  It's just like ordinary decision theory except that I have a set of preferences and expectations which represent the group.  This is not based on any sort of mutual knowledge (indeed it might not be based on any sort of knowledge at all).  It's also more powerful in that it can explain success on a wider range of problems than mutual knowledge (nb mutual knowledge isn't helpful for the solution to the Footballers' problem).

Third idea.  The sort of fiction that we engage in when observing joint action: we can construct the same sort of fiction when engaging in joint action.  We can shift from \emph{they} to \emph{us}.  In doing this we are effectively using team reasoning or something like it.  It is not necessary to look for justifications for the shift but we do need explanations.  Part of the explanation is surely abilities to identify opportunities (which presupposes abilities to engage in joint actions characterised by collective goals or by shared goals).  Part of the explanation might be group affiliation.  Part of the explanation might be social cues like the `shared look' characteristic of shared attention.



\subsection{[keep evolution and development separate]}
Have to keep evolution and development separate.  Someone born with limited abilities to engage in joint action might still acquire rich social cognition!  Mike put this kindly by saying (after, in discussion) that the story works better for evolution than development.


\subsection{Elena Lieven}
In what sense can the adult's goal be to find the object?  

Consider two cases in which my action is directed to mowing the grass.  There's a way in which I can be hooked up to this goal that allows others' participation consistent with my succeeding.  Perhaps there's also a way in which I can be hooked up to this goal that doesn't allow others' participation---as soon as you do any of the mowing, I have failed.

Btw, the adult's actual goal might be to help the child find the object; but it doesn't matter for the purpose of finding the object if the child mistakenly things the goal is to find the .


\subsection{Harriet Over}

How could this work if the child's is mistaken about the adult's goal?  How could the child ever come to a true understanding of communicative intentions?

The misunderstanding creates the need for an understanding of communicative intention.  The child knows \emph{that} holding up the replica is a way of getting the object.  Having seen this succeed, she might be in a position to ask \emph{how} that works.  And the answer is that she needs a communicative intention.

TODO: Isn't the replica condition about holding up a replica that determines which object is to be selected?


\subsection{Amrisha Vaish}
I claim that social cues indicate that we are about to engage in joint action.  Where do these social cues come from?  Isn't it plausible that humans only acquire the ability to recognise these cues through already having succeeded in identifying communicative intent in a sense at least as complex that you are trying to explain?

What seems right is that, potentially, the social cues are as hard to interpret as the communicative actions abilities to interpret which I am trying to explain.  The problem of obscure means arises in both cases.

In discussion after we talked about the possibility that these social cues might come to be recognised through more early binary and triadic interactions.
If I had a story along these lines, that would have the added advantage of explaining why understanding communicative intent only emerges against the background of rich forms of social interaction.
Not because understanding communicative intent constitutively involves these kinds of sharing.  But because this background makes possible the your-goal-is-my-goal route to knowledge by making it possible to identify social cues to joint action.

But of course these cues could just be hard wired.





\section{Orientation}
A joint action is a goal-directed action, or something resembling one, comprising two or more agents' goal-directed activities. 

Philosophers' paradigm cases of joint action include painting the house together (Michael Bratman), lifting a heavy sofa together (David Velleman), preparing a hollandaise sauce together (John Searle), going to Chicago together (Christopher Kutz), and walking together (Margaret Gilbert).


Some researchers have suggested that joint action might play a key role in the evolution or development of human cognition ...
%
\begin{quote} 
`the unique aspects of human cognition ... were driven by, or even constituted by, social co-operation'
\citep[p.\ 1]{Moll:2007gu}.
\end{quote}
%
\begin{quote} 
`perception, action, and cognition are grounded in social interaction%
% … functions traditionally considered hallmarks of individual cognition originated through the need to interact with others
' \citep[p.\ 103]{Knoblich:2006bn}.
\end{quote}
%
As a philosopher, it's not my job to work out whether these are true hypotheses.  I want to pursue a simpler question.  
What concepts will help us to understand and test these ambitious claims about joint action?

In thinking about joint action most researchers, especially philosophers, have focused almost exclusively on a single concept, \textbf{shared intention}.  
I'll say more about what shared intentions are later.  For now, let's just say that shared intentions are supposed to stand to joint actions as ordinary intentions stand to individual actions.

My aims in this talk are to persuade you that it is a mistake to focus exclusively on shared intention, and to introduce some  additional conceptual tools.  
To understand joint action and its potential significance in development and evolution we do need shared intention, for sure, but we also need additional conceptual tools.

%In some ways it surprises me that this is a controversial claim.  Compare individual action ... goal-directed action requires several notions, not all of it is intentional ... can help us to understand the structure of intentional action (Velleman).  My claim is just that something approximately parallel to this is true for the case of joint action.

%To understand joint action and its potential role in evolution and development we need multiple conceptual tools.  This is not to deny that shared intention is important, just that it can't do everything.

This is a talk with two halves.  

In the first half I shall explain why, in addition to shared intention, we need a further notion, which I call the collective goal.  The hard part is to see why collective goals are needed.  Once you see the need for them, it's not too difficult to say what collective goals are.

In this first half we will be concerned with forms of joint action that are probably too primitive to be terribly exciting here.  
I want to consider the primitive cases because it seems to me useful to take a constructive approach to theorising about joint action. 
We should start with the minimal case and only gradually introduce more complexity \citep{vesper_minimal_2010}.   

In the second half we shall turn to richer, more interesting cases of joint action.

I should thank my collaborators Cordula Vesper, Natalie Sebanz and Guenther Knoblich---some of what follows is based on joint work.





\section{First Half}


\subsection{The Standard Story}

As I said, a joint action is a goal-directed action, or something resembling one, comprising two or more agents' goal-directed activities. 

A basic question about joint action is, 
\textbf{What is the relation between a joint action and the goal (or goals) to which it is directed?}  
%(Here and throughout we ignore for simplicity the possibility that joint actions may be directed to multiple goals.  The answers considered are consistent with this possibility.)

To illustrate, suppose Ayesha and Beatrice between them lift a table thereby releasing a cat whose paws were trapped and,  simultaneously, breaking a glass.  On standard theories of events, the event which initiated the cat's release is identical to the event which initiated the glass' destruction \citep{Davidson:1969ie}.  But observing this joint action we might wonder whether its goal was the release of the cat or the destruction of the glass (or both); and if we are in doubt about this then in an important sense we don't yet know which action Ayesha and Beatrice performed.  To identify their action we need not only a concrete event, something with temporal and spatial properties, but also an abstract outcome such as that of the cat's release \citep{Davidson:1971fz}.

Ayesha, asked about the table lifting episode mentioned above, might insist, `The goal of our intervention wasn't to smash the glass but to free the cat.'  (Note that this concerns the goal of the joint action and not---or not directly---the agents' intention.)  In terms of this example, we could put the question like this.  \textbf{What makes it true that the goal of Ayesha and Beatrice's action is that of releasing the cat rather than that of breaking the glass?}

Answering this question is necessary for saying what joint actions are and for understanding in what senses, if any, they are actions.
%*better sentence on significance of question?

The term `goal' has been used both for outcomes (as in `the goal of our struggles') and, perhaps improperly, for psychological states (it is in this sense that agents' goals might cause their actions).  The terms `goal-outcome' and `goal-state' make the distinction explicit.  Our question is about how joint actions are related to their goal-outcomes.

For individual (that is, not joint) action, the counterpart of this question has a standard answer.  In barest outline, an individual action consists in an agent acting on an intention (a goal-state).  The intention has a content which specifies an outcome.
%, namely the outcome whose occurrence would consist in the propositional content's being true.
This is the goal-outcome of the action.
It is the goal-outcome of the action in virtue of being specified by the intention (or goal-state) the agent is acting on. 

In essence, then, on the standard theory, an individual action is related to its goal-outcome in virtue of the agent's acting on one or more intentions or goal-states.


How do things stand in the case of joint action?  Can we extend the standard theory from individual to joint action?  


\subsubsection{Shared Intention}


The usual way of thinking about joint action starts with the premise that all significant cases of joint action involve shared intention (dissenters are mentioned below).  For instance:  
%
\begin{quote} 
`I take a collective action to involve a collective intention.'  \citep[p.\ 5]{Gilbert:2006wr}.
\end{quote}
%
\begin{quote}
`the key property of joint action lies in its internal component \ldots \ in the participants’ having a ``collective'' or ``shared'' intention.' \citep[pp. 444-5]{alonso_shared_2009}.
\end{quote}
%
\begin{quote}
`Shared intentionality is the foundation upon which joint action is built.' \citep[p.\ 381]{Carpenter:2009wq}
\end{quote}
%
\begin{quote}
`it is precisely the meshing and sharing of psychological states \ldots \ that holds the key to understanding how humans have achieved their sophisticated and numerous forms of joint activity'
\citep[p.\ 369]{Call:2009fk}
\end{quote}

The notion of shared intention provides a straightforward way to extend the standard theory about how actions are related to their goal-outcomes to cases of joint action.

In brief, the standard theory is extended by substituting shared for individual intentions and otherwise unchanged.  

So on the usual way of thinking about joint action, shared intentions have two functions.  They coordinate multiple agents' activities; and they determine which goals their actions are directed to.


\subsubsection{My aims}
I want to argue that this extension of the standard theory is not a full answer to the question.  It is not a full account of the relation between joint actions and their goals.

For I claim that there are significant cases of joint action without shared intention.

So I think the following propositions can both be true:

\begin{enumerate}

\item The goal of Ayesha and Beatrice's action is to free the cat.  [That is, their joint action has a goal-outcome.]

\item It is not true that Ayesha and Beatrice have any kind of  shared goal-state.  

\end{enumerate}

In what follows I shall first argue for the consistency of these propositions by giving you some examples of joint actions without shared intentions.  I then offer an account of how joint actions are related to their goal-outcomes which does not involve anything like shared intention.  


\subsection{Preliminary: Shared Intention}

But before any of that I need to fill in some background on shared intention.

My immediate aim, as I mentioned, is to show that there are cases of joint action without shared intention.  An immediate obstacle is \textbf{lack of agreement on what shared intentions are}.  

Some hold that shared intentions differ from individual intentions with respect to the attitude involved (\citealp{Kutz:2000si}; \citealp{Searle:1990em}). 
Others have explored the notion that shared intentions differ with respect to their subjects, which are plural \citep{Gilbert:1992rs}, 
or that they differ from individual intentions in the way they arise, namely through team reasoning \citep{Gold:2007zd}, 
or that shared intentions involve distinctive obligations or commitments to others (\citealp{Gilbert:1992rs}; \citealp{Roth:2004ki}).
Opposing all such views, \citet{Bratman:1992mi,Bratman:2009lv} argues that shared intentions can be realised by multiple ordinary individual intentions and other attitudes whose contents interlock in a distinctive way. 

Given that almost no two philosophers agree on what shared intentions are, how can we say that shared intention is not involved in any given case?  

On all or most leading accounts of shared intention, each of the following is a necessary condition:

\begin{idescription}
\label{conditions-for-shared-intention}

\item[awareness of joint-ness] at least one of the agents knows that they are not acting individually; she or they have `a conception of themselves as contributors to a collective end.'\footnote{
	\citet[p.\ 10]{Kutz:2000si}.  Compare \citet[p.\ 361]{Roth:2004ki}: `each participant ... can answer the question of what he is doing or will be doing by saying for example ``We are walking together'' or ``We will/intend to walk together.''' 
Relatedly, \citet[p. 56]{miller_social_2001} requires that each agent believes her actions are interdependent with the other agent's.
}

\item[awareness of others' agency]  at least one of the agents is aware of at least one of the others as an intentional agent.\footnote{
	Compare \citet[p.\ 333]{Bratman:1992mi}: `Cooperation ... is cooperation between intentional agents each of whom sees and treats the other as such'.  See also \citet[p.\ 105]{Searle:1990em}: `The biologically primitive sense of the other person as a candidate for shared intentionality is a necessary condition of all collective behavior' 
}

\item[awareness of others' states or commitments] at least one of the agents who are F-ing is aware of, or has individuating beliefs about, some of the others' intentions, beliefs or commitments concerning F.\footnote{
This condition is necessary for shared intention even on what \citet[p.\ 40]{tuomela_collective_2000} calls `the weakest kind of collective intention'.  But it may not be necessary if, as \citet{Gold:2007zd} suggest, shared intentions are constitutively intentions formed by a certain kind of reasoning.
% "if the distinctive feature of collective intentions is to be found in the reasoning by which they were formed, then an analysis that focuses on the intentions themselves will miss the feature that makes collective intentions collective. " 
}

\end{idescription}

There are philosophers who deny that shared intention is necessary for joint action,\footnote
{
\citet[p.\ 407]{Roth:2004ki} and \citet{Searle:1990em}  hold that the intentions required for joint action need not be shared; \citet{miller_social_2001} also denies that shared intention is necessary for joint action.
*Should say something about Bratman and others.
*Should possibly also mention Kutz on participatory intentions.
}
but even they hold that one or more of these conditions is individually necessary for joint action (see footnotes above).

\textbf{What follows assumes that where one or more of these three conditions is not met, there is no shared intention.}

\subsection{Case Studies: Joint Action Without Shared Intention}


\subsubsection{Case Study---the environment coordinates joint actions}

Two ropes hanging over either side of a high wall are connected to a heavy block via a system of pulleys.  Ayesha and Beatrice each individually intend to raise the block.  
The positions of the walls mean that they can each see the block but they can't see each other.
Ayesha and Beatrice each know that, in addition to the rope they can pull, there is another rope and that force has to be exerted on that as well.  
But they have no idea about what will exert this force; for all they know it might be something mechanical or a fluke of nature rather than another agent.
They hold their own rope so that it's possible to feel when additional force is exerted on the other.  As soon as they feel such force, they attempt to lift the block.\footnote{
Compare the `Me plus X' notion from Vesper et al *ref
} 
This ensures that Ayesha and Beatrice pull the ropes simultaneously, causing the heavy block to rise as a common effect of their actions. 

Intuitively this is joint action, perhaps because the ropes and pulleys bind the agents' actions together and ensure a common effect.  
And I think this intuition is right.  For Ayesha and Beatrice's action's being directed to the goal of raising the block is not just a matter of each of their actions individually being directed to this goal.
In addition, there is a mechanism---the rope---which coordinates Ayesha and Beatrice's action and ensures that their individual pullings will normally cause the block to rise.  

None of the above necessary conditions for shared intention are met: there is no awareness of joint-ness, no awareness of others' agency and no awareness of others' states or commitments either.  

This artificial case indicates that in joint action much of the coordination can be taken care of by what objects afford multiple agents rather than by intentions. 



\subsubsection{Case Study---Kissing}
Kissing seems to be a good candidate for joint action because, like salsa dancing, it's not the sort of thing you can do alone.

But if the kiss is sufficiently spontaneous, it might not involve awareness of jointness or awareness of others' states or commitments in advance of success.  In this case, instead of shared intentions providing coordination, there are likely to be chemical and emotional means of coordination.



\subsubsection{Case Study---Motor Simulation [cut]}

There are other cases of where joint action involves meshing motor coordination rather than shared intentions.  Discussion of the details would take us too far from the main theme, but I would be happy to come back to these cases in discussion.

Overall, then, my suggestion is that coordination of two or more agents' activities can be provided by mechanisms in agents' environments, by emotional chemistry, and by meshing motor cognition.  
In each of these cases, the action's being directed to a goal involves more than each agent's activities individually being directed to this goal---in addition, there is an element of coordination.



\subsection{Collective Goals}
\label{section_collective}

So far I have argued that there are goal-directed joint actions without shared intentions.  
This means that we need an account of the relation between joint actions and their goals that doesn't invoke shared intention.
What is this account? 

The examples of joint action without shared intention each have \textbf{three features}.

First, there is a single outcome to which each of the agent's activities are individually directed.  
Let's use the term \emph{distributive goal} for this sort of goal.  For an outcome to be a distributive goal of a joint action just means that each agent's activities are indirectly directed to that goal.
For example, Ayesha's and Beatrice's activities were each directed to the goal of lifting the heavy block.

Second, the activities are coordinated

And third, the outcome occurs, or would normally occur, partly as a consequence of this coordination.  

These features constitute what I call a \emph{collective goal}.  Any outcome with these three features is a collective goal of the joint action.

Where a joint action has a collective goal there is a sense in which, taken together, the activities are directed to the collective goal.  It is not just that each agent individually pursues the collective goal; in addition, there is coordination among their activities which plays a role in bringing about the collective goal.  We can put this in terms of the direction metaphor.  Any structure or mechanism providing this coordination is directing the agents' activities to the collective goal.  The notion of a collective goal therefore already provides one way of making sense of the idea that joint actions are goal-directed actions.

The approach to explaining the relation between a joint action and its goal without invoking shared intention can now be stated.  A joint action is related to the goal to which it is directed in virtue of this goal  being the collective goal of the joint action.  The goal of a joint action is its collective goal.\footnote{  
Just as it is possible that some joint actions involve multiple shared intentions, so also might some involve involve multiple collective goals.  This minor complication is ignored only for ease of exposition.
}


\subsection{collective goals are not Shared intentions}
Collective goals are conceptually distinct from shared intentions.  Shared intentions are, or resemble, states of agents.  By contrast a collective goal is primarily an attribute of activities not of agents.  Furthermore, a shared intention is something that coordinates multiple agents' activities.  Appeal to a shared intention is therefore potentially a way of explaining how agents coordinate their activities.  By contrast, collective goals are not the sort of thing that can coordinate anything; their existence presupposes coordination.  Finally, shared intentions are supposed to be shared by agents in something resembling the sense in which conspirators can share a secret, not only in the sense in which two people can share a name.  It would be a distortion to claim that the notion of a collective goals is the notion of something agents share.







\section{Second Half}

So far I have suggested that understanding the relation between joint actions and their goals requires the notion of collective goal in addition to that of shared intention.

This is a modest claim.  It can be accommodated with at most small modification to any good account of joint action.  No revolution here.

The forms of joint action we have so far been considering are quite primitive and can be observed in many species, including some insect species.

They are much simpler than those found in humans and even in young children.  

Joint actions that one-year-old children engage in include tidying up the toys together (Behne, Carpenter and Tomasello 2005), cooperatively pulling handles in sequence to make a dog-puppet sing (Brownell, Ramani and Zerwas 2006), bouncing a ball on a large trampoline together (Tomasello and Carpenter 2007) and pretending to row a boat together. 

So from near the start of their second year or earlier, children engage in joint actions which are characterised by:
%
\begin{itemize}
\item sensitivity to others' mental states \citep{Buttelmann:2009gy}
\item dispositions to help partners \citep{Warneken:2006qe}
\item *role-swap (suggests awareness of the activity)
\item partner-specific projects \citep{Liebal:2010lr}
\item ...
\end{itemize}
%
In short, developmental and comparative research suggests that children are highly cooperative collaborators.  
In particular, their joint actions \textbf{involve potentially novel goals and are voluntary with respect to their jointness.  This means that the must involve psychological mechanisms for coordination}

So if we want to understand the potential significance of these joint actions in development and evolution, then we cannot adequately characterise them using the notion of collective goal alone.

I want to take a constructive approach to understanding joint action, one that starts with collective goals and adds, as explicitly as possible, the minimum necessary for a specified end.

What is the minimum we can add to the notion of collective goal in order to capture some joint actions which are voluntary with respect to joint-ness, may involve novel goals and depend on psychological mechanisms for coordination?


\subsection{How to share goals}
I want to start with the notion of collective goal and add the minimum necessary.  

Suppose two or more agents have a common goal.  What do we need to add to characterise joint actions which are voluntary with respect to jointness, directed to potentially novel goals and involve psychological states which coordinate the activities?

What we need to suppose is just that the agents are aware of the distributive goal and expect that their actions will succeed only in combination with others' efforts.

More explicitly we need to suppose that:
%
\begin{enumerate}
\item Each agent expects each of the other agents to perform activities directed to the goal.
\item Each agent expects the goal to occur as a common effect of all their goal-directed actions, or to be partly constituted by all of their goal-directed actions.
\end{enumerate}
%
In favourable circumstances this simple pattern of goals and expectations would be sufficient to coordinate the agents’ activities in bringing about this outcome. 

To illustrate, the goal of Ayesha's actions is to free the cat, Ayesha anticipates that Beatrice's actions will be directed to this goal, and Ayesha expects that the cat will be freed as a common effect of her own goal-directed actions and Ayesha’s; Beatrice's goals and expectations mirror Ayesha's. 
Their activities could be coordinated around the cat's release in virtue of this interlocking pattern of goals and expectations. The pattern of goals and expectations is a psychological means of coordination.

I'm going to call this pattern of goals and expectations a shared goal.  I'm nervous about invoking the term `sharing' because this has lots of romantic associations.  And of course shared goals are not literally shared.  You can't share a goal---or an intention, for that matter---in the sense that you can share a parent with a sibling.  So talk about sharing is just a colourful metaphor; what it amounts to in this case is just that each agent has expectations about others' goals and the efficacy of their actions.

The primary reason for labelling this a `shared goal' is that it fits into the standard story about the relation between actions and their goals that I mentioned earlier.

\textbf{
Shared goals have two basic features in common with ordinary, individual intentions.  
Like an ordinary, individual intention, a shared goal both specifies an outcome to which the action is directed and coordinates the activities which make up that action.
}


\subsection{shared goals are not shared intentions}

Shared goals are not shared intentions for two reasons.

First, on most accounts of shared intention, having a shared intention involves knowing that you have a shared intention, and therefore understanding what shared intentions are.  We'll see how this works in practice in a moment.

By contrast, shared goals do not have this property.  Having a shared goal does not require knowing that you have a shared goal, and it doesn't require understanding what shared goals are.  It requires only knowing that you have a distributive goal.


\textbf{Second, on most accounts of shared intention, having a shared intention involves representing others' intentions.  Having a shared goal does not involve this.}

Intentions are psychological states, and among the most complex of the propositional attitudes.  By contrast, to share goals it is only necessary to represent relations between actions and their goals.  This is similar to representing the functions of tools and need not involve representing psychological states.

\begin{comment}
Having a shared goal involves being able to represent actions as goal-directed.  Let me take a moment to spell out what this means.  
A goal-directed action is a sequence of object-directed behaviours, such as reaching, pulling and twisting, where the sequence as a whole has a function which isn't a function of any of the parts.
Note especially that the goal can be thought of as a property of the action rather than as a state of the agent.
To represent actions as goal directed, then, involves being able to represent object-directed behaviours and being able to represent functions.  
It need not involve being able to represent intentions or indeed any kinds of goal state.

By contrast, intentions are ...
%
\begin{list}{*}{}

\item attitudes to propositions

\item conclusions of practical reasoning

\item elements of plans

\item distinguished from other attitudes, such as desire, by the norms of consistency that constitutively govern intending
\end{list}
%
\end{comment}
%
So representing intentions is likely to be cognitively and conceptually demanding in ways that representing goal-directed actions is not.

This is why shared goals are distinct from shared intentions.  
\textbf{Having a shared goal doesn't involve knowing that you have it, or even knowing what a shared goal is; and having a shared goal doesn't require representing intentions.}


\subsection{Bratman on shared intention}
To distinguish more clearly between shared goals and shared intentions, I want briefly to turn to shared intention.

It is not easy to say what shared intentions are for two reasons.  First, shared intentions are neither literally intentions or literally shared.  
You can't share an intention in the sense that you can share a bottle of wine.  
The term `shared intention' is just a colourful metaphor.  
Second, as I mentioned earlier, almost no philosophers agree with any others on what shared intentions are.

For now I will simply adopt Michael Bratman's account of shared intention.  
This account is generally taken as a point of departure by philosophers and some psychologists.
It is theoretical coherent, 
no one has succeeded in identifying a valid objection to it in print, 
and I believe it captures one concept that is important for understanding joint action.

\subsubsection{Bratman's account of shared intention}

Bratman's account of shared intention has two parts, a specification of the functional role shared intentions play and a substantial account of what shared intentions could be.  On the first part, Bratman stipulates that the functional role of shared intentions is to: 
%
\begin{quote}
(i) coordinate activities; (ii) coordinate planning; and (iii) provide a framework to structure bargaining \citep[p.\ 99]{Bratman:1993je}
\end{quote}
%
To illustrate: if we jointly intend that we paint a house, this shared intention will require us to (iii) structure our bargaining insofar as we may need to decide what colours to paint it on the assumption that we are painting it together; the shared intention will also require us to (ii) coordinate our planning by fixing a day on which to paint and agreeing to bring complementary paints and tools, and to (i) coordinate our activities on the day by painting different parts of the house without getting in each others’ way.

Given this claim about what shared intentions are for, Bratman argues that the following three conditions are jointly sufficient\footnote{
In (1993), Bratman offers the following as sufficient and necessary conditions; the retreat to merely sufficient conditions occurs in Bratman (1999 [1997]) where he notes that “for all that I have said, shared intention might be multiply realizable.”
}  
for you and I to have a shared intention that we J.  This is his substantial account of what shared intentions could be:
%
\begin{quote}
`1. (a) I intend that we J and (b) you intend that we J

`2. I intend that we J in accordance with and because of la, lb, and meshing subplans of la and lb; you intend that we J in accordance with and because of la, lb, and meshing subplans of la and lb

`3. 1 and 2 are common knowledge between us' \citep[View 4]{Bratman:1993je}
\end{quote}
%
In favourable circumstances the attitudes specified in these conditions are capable of playing the three roles shared intentions are supposed to play.  This is what it means to say that they constitute a shared intention.



\subsection{Shared Intention is cognitively demanding}
On the substantial account given by Bratman, sharing intentions requires intentions about intentions (see Condition 2 in the quote above).\footnote{
Bratman emphasises this feature of the account: “each agent does not just intend that the group perform the […] joint action. Rather, each agent intends as well that the group perform this joint action in accordance with subplans (of the intentions in favor of the joint action) that mesh” (Bratman 1992: 332).
}

Furthermore, each agent must know that the others have intentions about her own intentions; and this knowledge must be mutual (see Condition 3 above).  So sharing an intention involves knowing that someone else knows that I have intentions concerning subplans of their intentions.  
   
This exposes Bratman's view to the objection that it is \textbf{too cognitively demanding}.  
%
\begin{quote}
`philosophers ... postulate complex intentional structures that often seem to be beyond human cognitive ability in real-time social interactions.'
\citep[p.\ 2022]{Knoblich:2008hy}
\end{quote}
%
This objection needs careful handling.  

Knoblich are right, I think, that shared intention is generally (although perhaps not always) unable to explain real-time coordination in spontaneous, small-scale joint actions.

But we need to be careful about the reasons for this.  It isn't really because Bratman's substantial account specifies complex representations.  For in his recent papers these conditions (the substantial account) are given as \emph{sufficient}  conditions for shared intention only.

The real reason why shared intention is too cognitively demanding for spontaneous, real-time coordination has to do with one of the functions of shared intentions: to coordinate planning.  

In Bratman’s account, the term `planning' is used in a narrow sense.  Planning in this narrow sense concerns the coordination of an agent’s various activities over relatively long intervals of time; it involves practical reasoning and forming intentions which may themselves require further planning, generating a hierachy of plans and subplans.  Paradigm cases include planning a birthday party or planning to move house.   

To share intentions is to be disposed to coordinate plans; because this requires recognising oneself and others as planning agents, it involves sophisticated insights into the nature of minds.  
Sharing intentions is cognitively demanding because coordinating plans is cognitively demanding.
So even if that states that realise sharing intentions didn’t require multiple levels of metarepresentation, sharing an intention would still be cognitively demanding.

Knoblich and Sebanz are right that shared intention is cognitively demanding.  But is this an objection?  I don't think it is an objection to the claim that Bratman is right about what shared intention is.  But it is a good objection to the claim that spontaneous joint action involving real-time coordination always involves shared intention.

My conclusion, then, is that \textbf{we need shared intention in addition to shared goals}.  These are conceptually distinct notions with empirically distinct conceptual and cognitive demands and are implicated in different sorts of joint action.

\textbf{Shared intentions are necessary for the sorts of joint action where you might need to consult your diary before agreeing to be involved.}

To return to the examples of children's joint action considered earlier.  For all I know, actual children who do these things might shared intentions.  But as the actions do not involve planning in Bratman's sense, they do not need to have shared intentions.  Their joint actions might in principle be powered by shared goals.
Or, more likely, by something intermediate in psychological complexity between shared goals and shared intentions.



\section{Summary so far (emergency conclusion)}

To sum up so far, I have suggested that it is a mistake to think that all significant cases of joint action involve shared intention.  
While shared intention might be important in lots of ways, it is a feature only of relatively sophisticated forms of joint action, those where coordinated planning in the `diary' sense is required.

My question was about the concepts we need to understand joint action and its potential significance in development and evolution.

In addition to shared intention, at least two further concepts are needed.

\textbf{We need collective goals because there are goal-directed actions involving multiple agents who do not share intentions or any other kind of goal states.}

\textbf{And we need shared goals because there are joint actions which are both voluntary with respect to their jointness and also spontaneous. 
Their voluntary nature means that they involve some form of shared psychological states.
But spontaneity means that coordination has to happen in real-time and so cannot involve shared intention.}


This is not supposed to be an exhaustive list of concepts.  The constructive approach I have outlined is extensible.

One last thing ... I want to look at an application of the notion of shared goals


\section{Joint Action and Knowing Others' Minds}
In a paper on the Vygotskian Intelligence Hypothesis, Moll and Tomasello say:
%
\begin{quote}
`regular participation in cooperative, cultural interactions during ontogeny leads children to construct uniquely powerful forms of cognitive representation.'
\citep[pp.\ 2-3]{Moll:2007gu}
\end{quote}
%
How does this work?
How does participation in cooperative interactions lead children to construct powerful forms of representation?

Moll and Tomasello explicitly adopt Bratman’s account of shared intention in characterising cooperative interactions:
%
\begin{quote}
`As in previous theoretical work […], we use here a modified version of Bratman’s (1992) definition of `shared cooperative activities'.'
\citep[p.\ 3]{Moll:2007gu}
\end{quote}
%
I don't think this will work.  
For sharing intentions already presupposes `powerful forms of cognitive representation', and in particular it presupposes sophisticated theory of mind cognition. 

The problem, once again, is that sharing intentions means coordinating one's planning with others' planning, and therefore representing their knowledge of your knowledge of their intentions about your intentions.
\textbf{Social cognition doesn't get much more conceptually sophisticated than this.}

Shared intention already presupposes too much social cognition to explain much about its development or evolution.

Can we do better with the notion of shared goals that I elaborated?

My suggestion concerns identifying the goals of actions.  
Some of the most plausibly unique aspects of human cognition depend on our abilities to recognise the goals of novel behaviours involving tools and gestures.  
\textbf{In particular, communicative actions depend on our abilities to recognise as the goals of behaviours goals which the behaviours can serve only because we recognise the behaviours as serving those goals [the Gricean circle].}
It is here that I think joint action can play a role in explaining aspects of human cognition.  
My suggestion will be, roughly, that abilities to engage in joint action provide a route to knowledge of others’ goals which is distinct from ordinary third-person interpretation.  

To explain this suggestion in detail I first need to identify a problem …



\subsection{The Problem of Opaque Means}

Suppose you cannot gain knowledge of the current goals of another's actions through linguistic communication.

Suppose the goal is relatively novel (there are no stereotypical indications).

In this situation we cannot generally do better to work backwards from her behaviours to her goals: we determine which outcomes her behaviour is likely to bring about and then suppose that her goal is to bring about one or more of these outcomes (Dennett 1991).  

But this method doesn’t work when: 
%
\begin{list}{*}{}
\item we don’t know which outcomes the observed behaviour is likely to bring about;
\item we, or the agent under observation, have false beliefs relevant to which outcomes this behaviour is likely to bring about; or
%\item there are many possible outcomes.
\end{list}
%
Cases where the method is particularly likely to fail include:
%
\begin{list}{*}{}
\item use of novel tools which we don’t recognise as tools or don’t know the functions of;
\item multi-step activities where the steps can occur in various orders and do not form a natural sequence 
%(it takes young children months to work out the need to move a switch in order to enable buttons to cause sounds by themselves [useful example because they pick this up right away if you show them how it works]);
\item communicative actions where there is no apparent connection between an act of meaning and its intended outcome (e.g. Tomasello, Call and Gluckman 1997; Leekam forthcoming)
\end{list}
%
Here’s how I propose that joint action bears on the problem of opaque means …


\subsection{The your-goal-is-my-goal route to knowledge}

How could abilities to engage in joint action provide us with knowledge of others’ goals?   The intuitive idea I started with was this: if you’re engaged in joint action with me, it’s easy for me to know what your goal is … because your goal is my goal.  

This intuitive idea isn’t quite right as it stands.  For to be engaged in joint action requires that I already have expectations about the goals of your behaviours.  
So engaging in joint action presupposes rather than explains knowledge of others’ goals.  Or so it seems.

But there is a way around this.  For there are various cues that you can give me which signal that you are about to engage in joint action with me.  Seeing me struggling to get my twin pram on to the bus, you grab the front wheels and make eye contact, raising your eyebrows and smiling.  In this way you signal that you both disposed to help and are about to engage in joint action with me.  This makes it trivial for me to know what the goal of your behaviour is: your goal is my goal, to get the pram onto the bus.

My suggestion, then, is that the following inference characterises a route to knowledge of others’ goals:
%
\begin{enumerate}
\item We are about to engage in some joint action\footnote{
*What notion of joint action is needed here?  Any will do as long as it involves distributive goals.
}
or other (for example, because you have made eye contact with me while I was in the middle of attempting to do something).

\item I am not about to change my goal.

\end{enumerate}
%
Therefore:
%
\begin{enumerate}[resume]
%
\item The others will each individually perform actions directed to my goal.
\end{enumerate}
%
Call this the ‘your-goal-is-my-goal’ route to knowledge.  To say that this inference characterises a route to knowledge implies two things.  First, in some cases it is possible to know the three premises, 1–2, without already knowing the conclusion, 3.  Second, in some cases knowing the two premises puts one in a position to know the conclusion.  I take both points to be true.

The your-goal-is-my-goal route to knowledge is characterised by an inference.  However, exploiting this route to knowledge may not require actually making the inference or knowing the premises.  Depending on what knowing requires, it may be sufficient to believe the conclusion because one has reliably detected a situation in which the premises of the inference are true without necessarily being able to think of this as a situation where the premises are true.


\subsection{Application}
I want to suggest that your-goal-is-my-goal might give us a way to understand how joint action facilitates a transition from a simple understanding of goals to an understanding of communicative intent.

\subsubsection{Step 1}
Hare and Call (\citeyear{hare_chimpanzees_2004}) contrast pointing with a failed reach as two ways of indicating which of two closed containers a reward is in.  Chimps can easily interpret a failed reach but are stumped by the point to a closed container.

In discussing this experiment, Moll and Tomasello say:
%
\begin{quote}
`to understand pointing, the subject needs to understand more than the individual goal-directed behaviour. She needs to understand that ... the other attempts to communicate to her ...  and ... the communicative intention behind the gesture'
(Moll \& Tomsello 2007)
\end{quote}
%
Of course I don't want to question this assertion.
But I do want to suggest that in the context of joint action there is a way to respond reliably to informative pointing without understanding pointing at all.
For if one knows that one is engaged in joint action with the person producing the point, one already knows what the (long-term) goal of the pointing action.  
The goal of the pointing is my goal, which is to find the reward.
So in the context of a joint action, it should be no harder to understand the point than it is to understand the failed reach.
Both are attempts to get the reward.

The pointing action, unlike the failed reach, is an \textbf{opaque means} of getting the object.  But in the context of joint action this doesn't matter because the your-goal-is-my-goal tells you that the goal of the point is to get the object.

\textbf{The `my goal is your goal' inference enables you to treat pointing as having the same goal as the failed reach.}

\subsubsection{Step 2}
Now consider a recent study by Sue Leekam and her colleagues.  Subjects---who were 2- and 3-year-old children---had to retrieve an object from one of three containers.  In some cases the experimenter indicated the correct container by holding a replica of the target object above that container.
Participants found it difficult to interpret this action.  Compared with a simple point to the correct location, participants given the replica as a cue to the correct location were less likely to succeed in finding the target.

In Leekam's experiment the key manipulation involved social cues.  In one case the experimenters' engaging face was visible to the child.  In the other case, the face was concealed.  Children (both 2- and 3-year-olds) did significantly better when the face was present.

Leekam and colleagues consider two interpretations of these findings, including this one:
%
\begin{quote}
`the adult’s social cues conveyed her communicative intent, which in turn encouraged the child to 'see through the sign'.'
\citep[p.\ 118]{leekam_adults_2010}
\end{quote}
%
In other words, the social cues somehow enhance children's abilities to understand communicative intent.

I want to suggest a different possibility.  The smiling face is a cue to joint action.  And in joint action it is possible to interpret the experimenter's action, holding up the replica, without first identifying communicative intent.

Let me explain.  Outside of the context of joint action, the child has to work out whether the experimenter's action, holding up the replica, is relevant to her current goal or not.  Since the child can't readily identify outcomes of this action, she can't identify its relevance.

But in the context of joint action, your-goal-is-my-goal means that children can know the experimenter's (distal) goal.
The experimenter's goal is to find the target.
And this enables children to solve the task without understanding communicative intent.  
It makes the replica like a failed reach: the child knows that the experimenter is attempting to retrieve the reward.

So Leekam's conjecture was that social cues help children to uncover communicative intentions.
An alternative is that the social cues are cues to joint action, and in the context of joint action it is not necessary to see the experimenter's action as communicative.

I want to stress that I'm not offering this as a plausible explanation of what is happening in these experiments.  And I'm not suggesting that any children don't understand communicative intent.  Rather my suggestion is more abstract.

\textbf{The suggestion is that capacities to share goals might play a role in explaining how humans come to understand communicative intent.
The idea is that in the context of joint action, communicative actions can be understood as ordinary goal-directed actions.
But once an action has been given a function in joint action, it can be used to serve that function outside joint action contexts.  And so it becomes genuinely communicative.}

Joint action may explain how individuals starting with a simple understanding of goals end up understanding communicative intentions.\footnote{
Contrast Csibra's `two stances' idea. The referential action understanding involves a “stance” (p. 455); teleological and referential action interpretation “rely on different kinds of action understanding' \citep[p.\ 456]{Csibra:2003kp}; they are initially two distinct `action interpretation systems' (although of course they come together later in development)  \citep[p.\ 456]{Csibra:2003kp}.
} 

This is one illustration of how capacities for joint action, even very simple forms of joint action, might be relevant to explaining the development or evolution of richer forms of cultural cognition.



\section{Conclusion}
In conclusion I have suggested that understanding what joint action is, and its potential roles in development and evolution, requires more than shared intentions.  It also requires collective goals, shared goals and perhaps more.

These notions presuppose less conceptual sophistication than the notion of shared intention and so have greater potential for explaining how humans acquire sophisticated forms of cognition.  If we think that cognition might be grounded in social interaction, or that unique aspects of human cognitive might be driven by social interaction, then we need to recognise that not all joint action presupposes sophisticated theory of mind reasoning.

One more way of putting it ... standard view allows just two ways of understanding `Ayesha and Beatrice lifted the table' ...





\section{*BIN}

\subsection{Why we need shared goals}

Here is a wild conjecture about joint action, one that is entirely at odds with the way that many, perhaps most, philosophers and psychologists think about joint action and theory of mind cognition:

\begin{quote}
Human social cognition, including full-blown theory of mind cognition, is built on capacities for joint action and their exercise.
\end{quote}
%
This is not the same as Moll and Tomasello's Vygotskian intelligence hypothesis,\footnote{
`the unique aspects of human cognition ... were driven by, or even constituted by, social co-operation'
\citep[p.\ 1]{Moll:2007gu}.
}
 but it is related.  The wild conjecture is also distantly related to Guenther Knoblich and Natalie Sebanz' hypothesis that human cognition generally is grounded in social interaction.\footnote{
`perception, action, and cognition are grounded in social interaction … functions traditionally considered hallmarks of individual cognition originated through the need to interact with others' \citep[p.\ 103]{Knoblich:2006bn}.
 }


There are at least two reasons why the wild conjecture, despite its imprecise formulation, seems completely untennable.  The first is empirical, the second conceptual.
%
\begin{enumerate}
\item The ability to ascribe false beliefs is a hallmark of full-blown theory of mind cognition; we now know that this ability appears at around 14 months or even earlier; and we may  find the ability to ascribe false beliefs in non-human primates or corvids as well.  So the objection is that, in both development and evolution, significant joint actions probably occur only after full-blown theory of mind cognition is established.

\item All joint actions involve shared intention; shared intention presupposes the ability to ascribe you knowledge of my intentions about your intentions; and this sophisticated capacity is already distinctive of human social cognition.  So the objection is that, once capacities for joint action appear, there is not much social cognition left for joint action to explain.
\end{enumerate}
%
I do have things to say about the first point but here I want to focus on the second.

So I'm not trying to establish the wild conjecture, only to remove one obstacle to its acceptance.

 

\bibliography{$HOME/endnote/phd_biblio}

\end{document}
