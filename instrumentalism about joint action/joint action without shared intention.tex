
\documentclass[12pt]{amsart}
\usepackage{geometry} % see geometry.pdf on how to lay out the page. There's lots.
\geometry{a4paper} % or letter or a5paper or ... etc
% \geometry{landscape} % rotated page geometry

\usepackage{natbib}

\usepackage{palatino} %font

\renewcommand{\baselinestretch}{1.5} %line spacing

%%% DOC PROPERTIES

\title{Joint Action without Shared Intention}
\author{Stephen Butterfill}
% \date{} % delete this line to display the current date

%%% BEGIN DOCUMENT
\begin{document}

\bibliographystyle{apalike}

\maketitle
% \tableofcontents



\section{Introduction}

Humans perhaps among other animals regularly engage in small-scale joint actions together such as lifting a heavy object together, painting a house together, or kissing a baby sister together.  This paper is concerned with cases of joint action which are relatively simple in that only two agents are involved and cooperation is not dependent on institutional structures, promises or relations of authority.  They are a subset of what Michael Bratman calls `modest sociality' \citep{Bratman:2009lv}. 

Philosophers generally agree that joint actions of this sort all involve shared intentions.\footnotemark \ \   While there is much debate on what shared intentions are, there is almost no discussion of whether shared intention is necessary for joint action.  I shall argue that it is not.
\footnotetext{
For example, \citep[p. 5]{Gilbert:2006wr}: 'I take a collective action to involve a collective intention.' 

Searle, Collective Intentions and Actions p. 105: `The biologically primitive sense of the other person as a candidate for shared intentionality is a necessary condition of all collective behavior' (*nb this doesn't quite say that shared intentions are necessary, and I'm not certain what the 'biologically primitive' qualifier is doing)

*others
}

Here, then, is the claim that concerns me:

\begin{quote}
Whenever two or three agents jointly F and their doing so does not depend on institutional structures, promises or relations of authority, then the agents all share an intention that they F.
\end{quote}

I shall argue against this claim by giving counterexamples, that is cases of joint action not involving shared intention.  



\section{Merely terminological?}

Before turning to the counterexamples I want to say a little more about the claim I am opposing.

You might think this claim is just a terminological stipulation.  Philosophers just choose to reserve the term `joint action' for cases involving shared intention.  And it's silly to quibble over terminology.

But the claim that all joint actions involve shared intentions is not just terminological because it has been given a central role in answering questions about joint action.  Let me explain.

Joint actions raise a tangle of scientific and philosophical questions.  Psychologically we want to know which mechanisms make it possible to engage in and understand different sorts of joint actions.  Developmentally we want to know when joint action emerges, what it presupposes and whether abilities to engage in it somehow facilitate socio-cognitive, pragmatic or symbolic development.  Conceptually, we want a principled way of distinguishing joint from individual actions which supports investigation of mechanisms and development.  Phenomenologically we want to characterise what (if anything) is special about experiences of action and agency when the actions are joint actions.  Metaphysically we want to know what kinds of entities and structures are implied by the recognition that some actions are joint actions.  And normatively we want to know what kinds of commitments (if any) are imposed on participants in joint actions or how these commitments arise [*Roth].\footnotemark \ \ 

\footnotetext{
On these questions, cf. Knoblich, Butterfill and Sebanz forthcoming *ref: 'What are the perceptual, cognitive, and motor processes that enable individuals to coordinate their actions with others, and how can the seemingly irreducible components of joint actions [...] be characterized?'.

*Pacherie on phenomenology
*V.I.H on development

See also Bratman's `three main concerns: conceptual, metaphysical, and normative. We seek an articulated conceptual framework that adequately supports our theorizing about modest sociality; we seek to understand what in the world constitutes such modest sociality; and we seek an understanding of the kinds of normativity�the kinds of ��oughts���that are central to modest sociality.'  \citep[p. 150]{Bratman:2009lv}. 
} 

The claim that joint actions all involve shared intentions, if true, would permit one to appeal to shared intention in answering these and other questions.  And this is frequently what happens.\footnotemark  This is why the joint actions all involve shared intentions is not merely terminological.  I claim that there are joint actions not involving shared intentions and that some of the philosophical and scientific questions about joint action apply to these joint actions as well as to those involving shared intentions.  Focussing exclusively on joint actions involving shared intentions means failing to fully answer some of the questions.
\footnotetext{For examples, see *Tomasello, *Bratman MSDI, *Tollefsen}


\section{Necessary conditions for shared intention}

As I said, my aim is to show that this claim is false by giving counterexamples:

\begin{quote}
*claim
\end{quote}

An immediate problem is that no philosopher agrees with any other on what shared intentions are.  How, then, can we say that shared intention is not involved in any given case?  On all or most leading accounts of shared intention, each of the following is an individually necessary condition for shared intention:


\begin{description}
\item[awareness of joint-ness] Agents who share an intention know that they are not acting individually; they have `a conception of themselves as contributors to a collective end.' \citep[p. 10]{Kutz:2000si}\footnotemark
\item[awareness of others' agency]  When agents share an intention, each is aware of at least one of the others as an intentional agent; `Cooperation ... is cooperation between intentional agents each of whom sees and treats the other as such' \citep[p. 333]{Bratman:1992mi}.
\item[knowledge of others' intentions] When two agents share an intention that they F, each knows that the other has an intention concerning F 
\end{description}


\footnotetext{Compare \citep[p. 361]{Roth:2004ki}: `each participant ... can answer the question of what he is doing or will be doing by saying for example ``We are walking together'' or ``We will/intend to walk together.''' }

(Strictly speaking we don't need the second condition because it is entailed by the third.)

These are individually necessary conditions for shared intention.  So where one or more of these conditions is not met, there is no shared intention.  


\section{Counterexample---the environment coordinates our actions}

There are two levers on either side of a box which are connected to a heavy block via a system of ropes and pulleys.

Ayesha and Beatrice pull the levers simultaneously, causing the heavy block to rise as a common effect of their actions.  Each individually intends to raise the weight.  They  are not aware of each other, nor even that they are participants in a joint action.

Intuitively this case counts as joint action even though the simultaneity of their pullings is a coincidence.  It counts as joint action because the mechanism they are acting on binds their actions together, ensuring a common effect.  

Now you might say that this is merely coordinated and cooperative behaviour; even ants are capable of joint action in this attenuated sense.  And I agree that, as it stands, this is not a good counterexample to the claim that all joint action involves shared intention.  

But consider a small elaboration.  Ayesha and Beatrice each know that, in addition to the lever they can see, there is another lever and that force has to be exerted on the other lever as well.  They each hold their own lever down so that it's possible to feel when additional force is exerted on the other lever.  As soon as they feel such force, they attempt to lift the weight.\footnotemark
\footnotetext{Compare the `Me plus X' notion from Vesper et al *ref}

Now Ayesha and Beatrice recognise that something additional is needed, so this elaborated case is distinct from what ants do.  But it's still clear that no shared intention is involved because the agents are indifferent to whether the additional factor involves agency or not.  
 
This case can be gradually elaborated.  For example, we could elaborate it so that Ayesha selects a parter for the weight lifting.  She recognises that some additional force is needed and selects Beatrice over another candidate to provide that force.\footnotemark  \ \ Notice that this doesn't imply that Ayesha is thinking of Beatrice as an intentional agent: she may be treating Beatrice just as a tool for exerting additional force.  So she is not necessarily sharing an intention with Beatrice.
\footnotetext{Compare Hare et al's experiment on partner selection *ref}

The ultimate elaboration would have Ayesha phoning Beatrice to plan when and how they will lift the weight; this certainly requires shared intention.  

What I take this counterexample to indicate, then, is that joint action without shared intention can involve varying degrees of psychological sophistication because there are varying degrees to which coordination relies on psychological mechanisms.  At one extreme are cases of merely cooperative and coordinated behaviour such as are found in ants; at the other extreme are paradigm cases of shared intentional activity such as painting a house together (everyone agrees about this); between these there are many further cases where coordination involves various psychological mechanisms but not shared intentions (this is news).

What's striking about this case is that much of the coordination can be taken care of by what the object affords for action rather than by intentions.  It's tempting to dismiss this sort of case as borderline or just plain lucky.  But there is some evidence that humans perceive affordances differently when acting in ways which require coordination between multiple agents (*ref to planks).  Joint action may be as much about how we perceive objects and events as it is about how we perceive other minds.  This is why   cases where coordination depends in part on shared affordances rather than shared intentions are significant cases of joint action.


\section{Counterexample---Kissing}

Stan and Mavis are having dinner together on their first date.  After drinking some wine they kiss each other.  Neither of them quite knew what was happening until it had already happened, and neither of them could say why they did it.

Why think that this is a joint action at all?  It involves an emotionally charged common effect and the agents are not acting alone.

But in this example two necessary conditions for shared intention are not met either before or during the kissing (although they may be met afterwards):

\begin{enumerate}
\item Neither Stan nor Mavis knew whether they were acting individually
\item Neither Stand nor Mavis knew what the other intended
\end{enumerate}

So here we have another case of joint action without shared intention.



\section{Counterexample---Shared Task Representations}

Sam and Ahmed are sitting have a pile of blocks in front of them.  They each individually intend to create a stack from all the blocks.  They are indifferent to the presence and activities of each other; as long as the blocks all end up in a single stack they will be satisfied.  Acting on their individual intentions, they pile the blocks into a single stack.

So far I think this is not obviously a case of joint action.  But if we add in shared task representations, I think it is very intuitive to think of it as joint action.  So let me explain what shared task representations are ...

Here is are two rules for action:

\begin{enumerate}
\item If a green ring appears, press the right button
\item If a finger pointing left appears, press the left button
\end{enumerate}

Sebanz and colleagues implemented two tasks based on these rules:
\begin{description}
\item[individual version] One person sees a series of stimuli.  She has to implement both rules.
\item[joint version] Two people see the same series of stimuli; each has to implement one of the rules. \citep{Sebanz:2005fk}
\end{description}

In the individual version, the agent has to inhibit tendencies to press the button to which the finger is pointing, which slows her responses and increases her errors.  Sebanz and colleagues showed that in the joint version of the task, the partners' performance resembles that of the single agent: the slowing of their responses indicates that they are inhibiting tendencies to move the other agent's hand.

On the basis of this and other findings, Sebanz et al hypothesise that there are shared task representations.  A shared task representation occurs when an agent represents the rules which characterise her own part in a task and also represents rules characterising the other agent�s role as if they characterised her own role.\footnotemark \ \   So having a shared task representation means that an agent will have tendencies to perform another�s prescribed actions as well as her own actions.  To perform only her own part in the task, she will have to inhibit these tendencies.  This is illustrated in the figure below.
 
*figure (see reply to Knoblich, Does Eve need Adam?)

\footnotetext{In general, an agent has a shared task representation when:
(i) she represents one or more rules characterising her own part in a task;
(ii) she represents one or more rules characterising another agent�s part in the task; and 
(iii) all rules are represented in a functionally equivalent way: that is, the representations of rules in (i) and (ii) alike dispose her to act on these rules.} 

Shared task representations are unlike shared intentions.  First, a shared task representation is not shared in the sense that two people necessarily represent the same task.  Rather, what's shared is the type of vehicle that a single person uses to represent both her own and her partner's role in a task.   Second, shared task representations are automatic in the sense that they occur even when agents are not supposed to be coordinating their activities and even when such coordination harms their performance [*ref].

In this task, shared task representations degrade performance because there is no need for the participants to coordinate their responses.  But in other cases where coordinated responses are needed, shared task representations could facilitate coordination.  Speaking loosely, the idea is that an agent with a shared task representation can coordinate her own actions with those of other agents because she plans for their actions in addition to planning for her own.  

This brings me back to the counterexample, to Sam and Ahmed who are stacking blocks.  Since they are in fact adding blocks to the same stack, Sam and Ahmed have to coordinate their actions with each other.  For instance, they have to time their actions to avoid colliding with each other, and they have to anticipate where the other will place a block when planning their own move.  Shared task representations make such coordination possible.\footnotemark
\footnotetext{See (Sebanz, Bekkering and Knoblich 2006; summarising Flanagan and Johansson 2003): ``when individuals observed a person stacking blocks, their gaze preceded the action and predicted a forthcoming grip, just like when they performed the block-stacking task themselves''.}   

In this case there is no shared intention because Sam and Ahmed need not think of themselves as contributing to a collaborative end and because they need not think of each other as intentional.  This is all irrelevant to the activity as I have described it.  But this does count as a case of joint action, I think, because the stack is created as a common effect of their actions and because the presence of shared task representations ensure that the two agent's actions resemble those of a single agent in some respects.  In these respects Sam and Ahmed are acting as one.  

The first counterexample I gave illustrated how joint action sometimes relies on shared affordances.  This counterexample illustrates how, in other cases, joint action relies on the meshing of motor cognition rather than the meshing of intentions.  To insist that joint action always involves shared intention means neglecting other psychological mechanisms involved in the coordination of joint action.



\section{Counterexample---The Imaginary We}

This counterexample depends on attributing states to an imaginary agent and executing actions on behalf of the agent.  This can seem like a strange idea.  So think first about how two- and three-year-old children interact with their teddies and dolls.  It's quite common for children to describe these imaginary agents as doing and feeling things, and as trying to do things.  Given what the imaginary agent is supposed to be doing, children are able to identify how they need to act on its behalf and to execute those actions on behalf of the imaginary agent.  

Note that children sometimes do this in order to bring about their own real-world objectives (especially when they risk getting into trouble).  So using an imaginary agent is compatible with achieving a real-world objective. 

I want to suggest that joint action sometimes involves imaginary agents as well.

Lucina and Charlie each want to complete a multi-step puzzle.  Neither has the skill or motivation to complete it alone, but they do have complementary skills which are jointly sufficient to complete the puzzle.  Lucina:

\begin{enumerate}
\item imagines that she and Charlie are a single agent
\item attributes to the imaginary agent the intention of completing the puzzle
\item identifies and executes part of what must be done on behalf of the imaginary agent in order for the puzzle to be completed
\end{enumerate}
	
Charlie does the same, ascribing the same intention to the imaginary agent and identifying and executing complementary parts of what must be done.  That Charlie hits on the same intention is more a matter of luck than reliable coordination; similarly, that he executes parts of the imaginary agent's behaviour which complement those Lucina executes is more a matter of luck than reliable coordination.

Of course I don't think that very much joint action is so thoroughly based on mindless optimism.  But I do think it sometimes is, and I think that joint action probably involves more mindless optimism---and more imagination---than philosophers typically allow.

This counts as joint action because two agents work together to achieve an outcome which occurs as a common effect of their actions, and they themselves construe their actions as if they were the actions of a single agent.

The necessary conditions for shared intention are not met in this case: the agents need not think of themselves as contributing to a collective end and they need not know what the other intends before the joint action.



\section{Conclusion}

As I said at the start, most philosophers claim that all joint action involves shared intention.  I have given a series of counterexamples to this claim.  There are multiple kinds of joint action and many do not involve shared intention.  This matters because many (but not not all) of the scientific and philosophical questions about joint action apply in cases not involving shared intention.  Focusing exclusively on shared intentions means failing to fully answer questions about joint action.

[That some joint actions do not involve shared intention also matters for questions about what is involved in recognising and reasoning about joint actions as an observer.  Adult humans, at least, readily recognise and reason about joint actions.  Intuitively, it seems that in plenty of cases it is no harder to reason about what \emph{they} are doing than it does to reason about what \emph{she} is doing.  For example, consider the case of two men lifting a heavy barrel into a boat together.  What is involved in understanding this?  Does one have to attribute a shared intention to the agents?  In my view we have a variety of ways of understanding joint action as observers---and this variety of ways of understanding joint action is a reflection of the many ways in which joint actions can come about.  One way of understanding joint action as an observer is probably to attribute shared intentions.  But another, perhaps less cognitively demanding way, may be simply to imagine that the actions are those of a single agent and attribute ordinary individual intentions to this pretend agent.]

I am not suggesting that we need an alternative definition of joint action, one that does not involve shared intention.  In my view it is probably a mistake to try to give a single account which applies to every case of joint action.  The various cases may not have any deep unity.  Instead I am drawn to instrumentalism about joint action.  

According to instrumentalism, for a sequence of events to be a joint action it is sufficient that:

\begin{enumerate}
\item the events include purposive actions done by more than one agent; and 
\item some observers (or participants) can usefully construe the events as if they constituted a single purposive action performed by only one agent.
\end{enumerate}

As broad as it is, even this condition probably fails to be necessary for joint action (perhaps it excludes the kissing considered earlier).  But I don't think it matters very much whether we can define joint action in general.  The point of instrumentalism is to make us open to the possibility that there could be many different forms of joint action. There may be no essential features that all joint actions have to share and so no reason to think that there is a single substantive characterisation of joint action which captures every case.  If so, answering the really deep questions about joint action requires understanding shared intentions, certainly, but it also requires understanding all of the many other states and structures which make joint actions possible. 

*The next, more radical, step is to question the notion of shared intention itself.  Since shared intentions are neither literally shared nor literally intentions, it is a reasonable conjecture that there is more than one useful way of cashing out the metaphor of shared intention.  


\bibliography{$HOME/endnote/phd_biblio}

%\subsection{}

\end{document}