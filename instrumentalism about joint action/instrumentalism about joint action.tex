
\documentclass[12pt]{amsart}
\usepackage{geometry} % see geometry.pdf on how to lay out the page. There's lots.
\geometry{a4paper} % or letter or a5paper or ... etc
% \geometry{landscape} % rotated page geometry

\usepackage{natbib}

\usepackage{palatino} %font

\renewcommand{\baselinestretch}{1.5} %line spacing

%%% DOC PROPERTIES

\title{Instrumentalism about Joint Action}
\author{Stephen Butterfill}
% \date{} % delete this line to display the current date

%%% BEGIN DOCUMENT
\begin{document}

\bibliographystyle{apalike}

\maketitle
% \tableofcontents

\section{Introduction}
Philosophers generally agree that joint actions are actions involving shared intentions.\footnotemark \ \   While there is a rich debate on what shared intentions are, there is almost no discussion of whether shared intention is essential for joint action.  In this paper I argue that it is not.

\footnotetext{For example, \citep[p. 5]{Gilbert:2006wr}: 'I take a collective action to involve a collective intention.' *others}

This matters  because philosophers have focussed on notions of joint action that are too narrow and ignored states and things other than shared intentions which also serve to make joint action possible.  Of course there are good reasons why philosophers might be primarily concerned with those forms of joint action which involve shared intentions.  But just as a full understanding of individual intentions requires understanding how they interact with other psychological states involved in control of action, so fully understanding shared intention will require understanding its interactions with other states involved in interpersonal coordination of action.

It is hardly controversial that there are many cases of individual action which do not involve intention, such as response behaviours \citep{Dickinson:1993oy} and arrational actions \citep{Hursthouse:1991rd}.  Given this, it should not have been surprising that there are also kinds of joint action which do not involve shared intentions.  Investigation of non-intentional forms of joint action can also shed light on forms which do involve shared intention.  



\section{Instrumentalism about joint action}
Instrumentalism about joint action is a useful starting point.  For a sequence of events to be a joint action it is sufficient that:

\begin{enumerate}
\item the events include purposive actions done by more than one agent; and 
\item some observers (or participants) can usefully construe the events as if they constituted a single purposive action performed by only one agent.
\end{enumerate}

As broad as it is, this characterisation leaves out some cases which are intuitively joint actions.  Consider kissing---consider an event in virtue of which `Ayesha and Beatrice kissed each other' is true.  Construing this event as if only one agent is involved seems to conflict with construing it as a kissing, or at least as any ordinary kind of kissing.  *What to do about this?!  Perhaps kissing each other is different from kissing a kiss (cf. dancing a dance), and the genuinely joint cases are those in which Ayesha and Beatrice kiss a kiss.

In offering instrumentalism as a starting point, I am suggesting that there is not much to be said for asking whether something is \emph{really} joint action and that the term `joint action' does not pick out a natural kind.  The point of instrumentalism is to make us open to the possibility that there could be many different forms of joint action.  There are no essential features that joint actions have to share and so no reason to think that there is a single substantive characterisation of joint action which captures every case.  

According to Christopher Kutz, `all forms of collective action share a common element in the form of overlapping, individual participatory intentions' \citep[4]{Kutz:2000si}.  This is supposed to be a substantive claim, not merely terminological.  Taken as a substantive claim, even this minimal condition fails to be necessary.  To have participatory intentions involves agents having `a conception of themselves as contributors to a collective end.' \citep[10]{Kutz:2000si}.  Consider a child attempting to fasten a button; after several failed attempts, an uncle surreptitiously supports the child's efforts.  The fastening of the button is a common effect of the purposive actions of two agents (neither agent's actions would have been individually sufficient), and it is conceivably useful to construe this event as if it involved only one agent.  This is sufficient reason to regard this as a case of joint action.  But the child, who is unaware of being assisted, lacks any participatory intention.  So Kutz' necessary condition for joint (or, as he calls it, `collective') action is not met.  

*Even more radical case: two agents engaged in plural activity but neither realises this.  Coordination is achieved by the affordances offered by the object they are acting on.
 
I am proposing instrumentalism as a starting point only.  Starting with instrumentalism is consistent with holding that there may be much to be gained from characterising particular forms of joint action.  So instrumentalism is consistent with Michael Bratman's claim, in characterising shared cooperative activity, to have identified a notion that `many of us see as important in out lives' \citep[327]{Bratman:1992mi}.  If (as I think) there are deep questions about joint action, substantial accounts of particular forms of joint action will be needed to answer them.  But it may be that no single account can answer all of the deep questions.



\section{Instrumentalism about shared intention}
You and I can literally share a bottle of wine: it can be true that the bottle is yours and, simultaneously, true that the wine is mine.   By contrast, we cannot literally share an intention.  There is (in my view at least) no psychological state which is, simultaneously, my intention and your intention.\footnotemark  \ \ So, in the absence of a substantive account, talk about shared intentions is metaphorical.

\footnotetext{*refs to others -- Smith?  Roth? Gilbert?}
I propose that we should start with instrumentalism about shared intentions.  For two or more agents to share an intention it is sufficient that some thinkers can usefully construe these agents as if they were a single agent with that intention.

 *use this to get Gold \& Sugden -- pointless to suggest that they are replacing Bratman's -- both accounts valuable insofar as they can shed light on different sorts of case.


\bibliography{$HOME/endnote/phd_biblio}

%\subsection{}

\end{document}