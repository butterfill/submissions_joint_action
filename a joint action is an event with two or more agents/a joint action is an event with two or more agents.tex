%!TEX TS-program = xelatex
%!TEX encoding = UTF-8 Unicode

\documentclass[12pt,a4paper]{extarticle}
% extarticle is like article but can handle 8pt, 9pt, 10pt, 11pt, 12pt, 14pt, 17pt, and 20pt text

\def \ititle {A Joint Action Is an Event with Two or More Agents}
\def \isubtitle {}
\def \iauthor {Stephen A. Butterfill}
\def \iemail{s.butterfill@warwick.ac.uk}
%\date{}

\input{$HOME/Documents/submissions/preamble_steve_paper}

\begin{document}

\setlength\footnotesep{1em}

\bibliographystyle{newapa} %apalike

\maketitle
%\tableofcontents

\begin{abstract}
Two conceptions of joint action ... The second kind of account raises a series of puzzles which, I argue, can be overcome.
The advantages of this approach for analysis and scientific research, are described.
\end{abstract}


\section{Introduction}
Some types of action, such as painting a house, tidying up, moving a table and tickling a baby, can be done jointly with others as well as individually.

The existence of joint action raises a tangle of empirical and philosophical questions.
These include 
 questions about its role in development \citep{Moll:2007gu,Hughes:2004zj,Brownell:2006gu}, 
 questions about the embodied and cognitive bases of joint action \citep{schmidt_understanding_2010,vesper_minimal_2010}, 
  as well as 
  questions about phenomenological characteristics of joint actions \citep{Pacherie:2010fk,seemann_why_2009}
  and questions about the distinctive sorts of commitment, if any, they involve \citep{gilbert_walking_1990,roth_shared_agency}.
An analysis of joint action should provide a conceptual framework which facilitates investigation of these questions \citep[p.\ 150]{Bratman:2009lv}.

The standard approach to providing this analysis starts with the notion of shared intention.
As we explain in detail below, on this approach the characteristic feature of joint action is either a complex of interlocking psychological states or commitments or else a special kind of psychological attitude or subject.
On this approach the analysis of joint action is primarily concerned with identifying the states, commitments or subjects which characterise joint action.
When such analyses are considered as attempts to provide a conceptual framework for investigation of the tangle of questions about joint action, a key objection is that no single analysis applies to every case of interest (as we explain below).
Instead we have a plethora of analyses, each of which only partly overlaps with others and several of which seem to capture some of the cases to which the tangle of questions apply.

This paper therefore examines an alternative approach derived from recent work in semantics.
Here the key idea is that a joint action is an event with two or more agents.
This idea faces a series of objections which, we argue, can be overcome.
While some objections remain, the alternative approach differs from the standard approach in providing a simpler, more general and more unifying conceptual framework; in these respects it is better suited to facilitating investigation of the tangle of questions about joint action.

Our aim, then, is to articulate and defend an analysis of joint action on which a joint action is simply an event with two or more agents, and to explain the benefits of this analysis as a conceptual framework for investigating scientific and philosophical questions about joint action.
In what follows we first describe the standard approach for readers unfamiliar with philosophical research on joint action.



\section{Shared Intention}
To standard way of analysing joint action starts with the premise that all significant cases of joint action involve shared intention.  This is taken for granted by Alonso:
%
\begin{quote}
`the key property of joint action lies in its internal component [...] in the participants’ having a “collective” or “shared” intention.' \citep[pp. 444-5]{alonso_shared_2009}
\end{quote}
%
Gilbert is also explicit on this point:  
%
\begin{quote} 
`I take a collective action to involve a collective intention.'  \citep[p.\ 5]{Gilbert:2006wr}\footnote{
For the purposes of this paper we can treat `collective intention' as synonymous with `shared intention'.  
In introducing the term `collective action', Gilbert stipulates that collective action involves more than multiple agents performing concurrent actions while leaving open what more is involved (p.\ 4).   
As this matches the sufficient condition for joint action we gave above, we assume that joint actions of the sort we are concerned with are collective actions.
}
\end{quote}
%
And the same view is taken by Tomasello:
%
\begin{quote}
`The sine qua non of collaborative action is a joint goal [shared intention] and a joint commitment’ 
\citep[p.\ 181]{tomasello:2008origins}\footnote{
The context makes it clear that a `joint goal' involves a shared intention in approximately Bratman's (\citeyear{Bratman:1993je}) sense.
}
\end{quote}
%
On this approach, a joint action is an event appropriately related to a shared intention.
The main difficulty is then to explain what shared intention is.
So what is shared intention?  
In barest outline, shared intentions are supposed to do for multiple agents some of what ordinary intentions do for individuals.  
So, like an ordinary intention, a shared intention's function is to coordinate plans and activities---with the difference that these are the plans and activities of multiple agents performing a joint action \citep{Bratman:1993je}.

This tells us something about what shared intentions are not.
Suppose there is a house which Ayesha and Lucinda each intend to paint today.
Their intentions are of the same type.
But it is clear that their having these intentions does not amount to their having a shared intention.
For these intentions alone will not function to coordinate their plans and activities.
Although each intends to paint the house it may be that neither knows of the other's intention and, approaching the house from different sides, they each paint in ignorance of the other's activities.  
So having a shared intention amounts to more than having an intention of the same type as another agent.

Beyond this point there is almost no agreement on what shared intentions are.
*attitude, content, commitment ...






\bibliography{$HOME/endnote/phd_biblio}

\end{document}